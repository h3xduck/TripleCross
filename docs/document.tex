

%----------
% DOCUMENT CONFIGURATION
%----------

\documentclass[12pt]{report} %font 12pt

% MARGINS
\usepackage[
a4paper,
vmargin=2.5cm,
hmargin=3cm
]{geometry}

\renewcommand{\baselinestretch}{1.15}
\parskip=6pt

\usepackage{pdflscape}

% COLORS for cover and code
\usepackage[table]{xcolor}
\definecolor{azulUC3M}{RGB}{0,0,102}
\definecolor{gray97}{gray}{.97}
\definecolor{gray75}{gray}{.75}
\definecolor{gray45}{gray}{.45}

% Soporte para GENERAR PDF/A --es importante de cara a su inclusión en e-Archivo porque es el formato óptimo de preservación y a la generación de metadatos, tal y como se describe en http://uc3m.libguides.com/ld.php?content_id=31389625. En la carpeta incluímos el archivo plantilla_tfg_2017.xmpdata en el que puedes incluir los metadatos que se incorporarán al archivo PDF cuando lo compiles. Ese archivo debe llamarse igual que tu archivo .tex. Puedes ver un ejemplo en esta misma carpeta.
\usepackage[a-1b]{pdfx}

% LINKS
\usepackage{hyperref}
\hypersetup{colorlinks=true,
	linkcolor=black,
	urlcolor=blue}

% MATH
\usepackage{amsmath,amssymb,amsfonts,amsthm}

\usepackage{txfonts} 
\usepackage[T1]{fontenc}
\usepackage[utf8]{inputenc}

\usepackage[english]{babel} 
\usepackage[babel, english=american]{csquotes}
\AtBeginEnvironment{quote}{\small}

\usepackage{fancyhdr}
\usepackage{tikz}
% FOOTER
\pagestyle{fancy}
\fancyhf{}
\renewcommand{\headrulewidth}{1pt}
\fancyhead{}         
\fancyhead[LO]{CHAPTER \thechapter}     
\fancyhead[RO]{\rightmark}  
\rfoot{\thepage}
\fancypagestyle{plain}{\pagestyle{fancy}}
\fancypagestyle{plainnofancy}%
{%
\fancyhf{}
\renewcommand{\headrulewidth}{0pt}
\rfoot{\thepage}
}

%LANDSCAPE PAGES
\fancypagestyle{lscape}{% 
\fancyhf{} % clear all header and footer fields 
\fancyfoot{%
    \tikz[remember picture,overlay]
      \node[outer sep=1cm,above,rotate=90] at (current page.45) {\thepage};}
\renewcommand{\headrulewidth}{0pt} 
\renewcommand{\footrulewidth}{0pt}
}

% TITLES
\usepackage{titlesec}
\usepackage{titletoc}
\titleformat{\chapter}[block]
{\large\bfseries\filcenter}
{\thechapter.}
{5pt}
{\MakeUppercase}
{}
\titlespacing{\chapter}{0pt}{0pt}{*3}
\titlecontents{chapter}
[0pt]                                               
{}
{\contentsmargin{0pt}\thecontentslabel.\enspace\uppercase}
{\contentsmargin{0pt}\uppercase}                        
{\titlerule*[.7pc]{.}\contentspage}                 

\titleformat{\section}
{\bfseries}
{\thesection.}
{5pt}
{}
\titlecontents{section}
[5pt]                                               
{}
{\contentsmargin{0pt}\thecontentslabel.\enspace}
{\contentsmargin{0pt}}
{\titlerule*[.7pc]{.}\contentspage}

\titleformat{\subsection}
{\normalsize\bfseries}
{\thesubsection.}
{5pt}
{}
\titlecontents{subsection}
[10pt]                                               
{}
{\contentsmargin{0pt}                          
	\thecontentslabel.\enspace}
{\contentsmargin{0pt}}                        
{\titlerule*[.7pc]{.}\contentspage}  


% TABLE DESIGN
\usepackage{multirow}
\usepackage{caption}
\usepackage{floatrow} 
\usepackage{array}
\newcolumntype{P}[1]{>{\centering\arraybackslash}p{#1}}
\DeclareCaptionFormat{upper}{#1#2\uppercase{#3}\par}

\captionsetup[table]{
	%format=upper,  UPPER??? Set by the template, but it looks really weird, I got this off
	justification=centering,
	labelsep=period,
	width=.75\linewidth,
	labelfont=small,
	font=small,
}


% FIGURES DESIGN
\usepackage{graphicx}
\graphicspath{{images/}}

\captionsetup[figure]{
	format=hang,
	name=Fig.,
	singlelinecheck=off,
	labelsep=period,
	labelfont=small,
	font=small,
	%THE FOLLOWING WAS ADDED BY ME, is this ok? I think it was missed on the template
	justification=centering		
}


% FOOT NOTES
\usepackage{chngcntr}
\counterwithout{footnote}{chapter}

% CODE LISTINGS
\usepackage{listings}

\lstdefinestyle{estilo}{ frame=Ltb,
	framerule=0pt,
	aboveskip=0.5cm,
	framextopmargin=3pt,
	framexbottommargin=3pt,
	framexleftmargin=0.4cm,
	framesep=0pt,
	rulesep=.4pt,
	backgroundcolor=\color{gray97},
	rulesepcolor=\color{black},
	%
	basicstyle=\ttfamily\footnotesize,
	keywordstyle=\bfseries,
	stringstyle=\ttfamily,
	showstringspaces = false,
	commentstyle=\color{gray45},     
	%
	numbers=left,
	numbersep=15pt,
	numberstyle=\tiny,
	numberfirstline = false,
	breaklines=true,
	xleftmargin=\parindent
}


\captionsetup[lstlisting]{font=small, labelsep=period}

\lstset{style=estilo}
\renewcommand{\lstlistingname}{\uppercase{Code}}

% IEEE BIBLIOGRAPHY
\usepackage[backend=biber, style=ieee, isbn=false,sortcites, maxbibnames=5, minbibnames=1]{biblatex}

\addbibresource{bibliography/bibliography.bib}


%-------------
%	DOCUMENT
%-------------

\begin{document}
\pagenumbering{roman}
	
%----------
%	COVER
%----------	
\begin{titlepage}
	\begin{sffamily}
	\color{azulUC3M}
	\begin{center}
		\begin{figure}[H] %university logotype
			\makebox[\textwidth][c]{\includegraphics[width=16cm]{Portada_Logo.png}}
		\end{figure}
		\vspace{2.5cm}
		\begin{Large}
			University Degree in Computer Science and Engineering\\			
			Academic Year 2021-2022\\
			\vspace{2cm}		
			\textsl{Bachelor Thesis}
			\bigskip
			
		\end{Large}
		 	{\Huge ``An analysis of offensive capabilities of eBPF and implementation of a rootkit''}\\
		 	\vspace*{0.5cm}
	 		\rule{10.5cm}{0.1mm}\\
			\vspace*{0.9cm}
			{\LARGE Marcos Sánchez Bajo}\\ 
			\vspace*{1cm}
		\begin{Large}
			Juan Manuel Estévez Tapiador\\
			Leganés, 2022\\
		\end{Large}
	\end{center}
	\vfill
	\color{black}
	% Creative Commons license
	\includegraphics[width=4.2cm]{images/creativecommons.png}\\  
	This work is licensed under Creative Commons \textbf{Attribution – Non Commercial – Non Derivatives}
	\end{sffamily}
\end{titlepage}

\newpage
\thispagestyle{empty}
\mbox{}

%----------
%	SUMMARY & KEYWORDS
%----------	
\renewcommand\abstractname{\large\uppercase{Summary}}
\begin{abstract}
\thispagestyle{plainnofancy}
\setcounter{page}{3}
	
	% So I read that acronyms are not allowed in abstracts and I should write the full name. At the same time, the official ebpf page says it is not an acronym anymore...

eBPF is a technology introduced in the 3.18 version of the Linux kernel that allows running code in the kernel without the need of loading a kernel module. Although originally intended for filtering packets, eBPF programs can be used for network monitoring, accessing kernel-exclusive resources and tracing activities at the user and kernel space. This has positioned eBPF as a leading environment for the development of network, security and observability tools. During the last years, however, eBPF has been found to be at the heart of the latest innovation on the development of rootkits.

This work identifies the offensive capabilities of eBPF that could be weaponized by a threat actor. Based on them, we have developed an 
eBPF-based rootkit that uses these capabilities to showcase multiple malicious use cases. Our rootkit, named TripleCross, incorporates (1) a
library injection module to execute malicious code by writing at processes virtual memory; (2) an execution hijacking module that modifies data passed to the kernel to execute malicious programs; (3) a local privilege escalation module that allows for running malicious programs with root privileges; (4) a backdoor with C2 capabilities that can monitor the network and execute commands sent from a remote rootkit client, incorporating multiple backdoor triggers so that these actions are transmitted with stealth in mind; (5) a rootkit client program that allows an attacker to establish 3 different types of shell-like connections for sending commands and actions that control the rootkit state
remotely; (6) a persistence module that ensures the
rootkit remains installed maintaining full privileges even after a reboot event; and (7) a stealth module that hides rootkit-related files and directories from the user.

TripleCross demonstrates the existing danger when running eBPF programs, a
technology also available by default in most distributions. It is intended for being used in pentesting and red teaming exercises.

%Apparently I must not repeat those appearing in the title
	\textbf{Keywords: Backdoor; Berkeley Packet Filter; Implant; Command and Control; Linux kernel; Malware; Computer security}
	% TODO KEYWORDS
	
	\vfill
\end{abstract}
	\newpage
	\thispagestyle{empty}
	\mbox{}
	
	
%----------
%	DEDICATION
%----------	
\chapter*{Dedication}

\setcounter{page}{5}
	\thispagestyle{plainnofancy}
	\vfill
	
	\newpage
	\thispagestyle{empty}
	\mbox{}
	

%----------
%	INDEXES
%----------	

%--
%General indexes
%-
\tableofcontents



\newpage
\thispagestyle{empty}
\mbox{}

%--
% List of figures
%-
\listoffigures
\thispagestyle{fancy}

\newpage
\thispagestyle{empty}
\mbox{}

%--
% List of tables
%-
\listoftables
\thispagestyle{fancy}

\newpage % página en blanco o de cortesía
\thispagestyle{empty}
\mbox{}

%----------
%	INTRODUCTION
%----------	

\clearpage
\pagenumbering{arabic}

% This prevents the underscores going out of the margins
\renewcommand\_{\textunderscore\allowbreak}
\chapter{Introduction}
\section{Motivation} \label{section:motivation}
%M-> SA bit long, but it summarizes and presents the ideas and background needed to understand the topic in order:
% Main idea: Malware keeps evolving -> 
% -> Relevance of innovating and researching on the new techniques ->
% -> Relevance of stealth software in targeted attacks-> 
% -> Introduce eBPF as the logical step of innovation in the field ->
% -> There is a need to research on this topic now.

As the efforts of the computer security community grow to protect
increasingly critical devices and networks from malware infections, the
techniques used by malicious actors become more sophisticated.  Following
the incorporation of ever more capable firewalls, Endpoint Detection and
Response (EDR), and Intrusion Detection Systems (IDS), cybercriminals have
in turn sought novel attack vectors and exploits in common software, taking
advantage of an inevitably larger attack surface that keeps growing due to
the continued incorporation of new programs and functionalities into modern
computer systems.

In contrast with ransomware incidents, which remained the most significant
and common cyber threat faced by organizations in 2021
\cite{ransomware_pwc}, a powerful class of malware called rootkits is found
considerably more infrequently, yet it is usually associated to
high-profile targeted attacks that lead to greatly impactful consequences.

A rootkit is a piece of computer software characterized for its advanced
stealth capabilities. Once it is installed on a system it remains invisible
to the host, usually hiding its related processes and files from the user,
while at the same time performing the malicious operations for which it was
designed. Common operations include storing keystrokes, sniffing network
traffic, exfiltrating sensitive information from the user or the system, or
actively modifying critical data at the infected device. The other
characteristic functionality is that rootkits seek to achieve persistence
on the infected hosts, meaning that they keep running on the system even
after a system reboot, without further user interaction or the need of a
new compromise. The techniques used for achieving both of these
capabilities depend on the type of rootkit developed. One of the most
commmon classifications is based on the level of privileges on which the
rootkit operates in the system \cite{rootkit_ptsecurity}:
\begin{itemize}
\item \textbf{User-mode} rootkits run at the same level of privilege as
common user applications. They usually work by hijacking legitimate
processes on which they may inject code by preloading shared libraries,
thus modifying the calls issued to user APIs, on which malicious code is
placed by the rootkit. Although easier to build, these rootkits are exposed
to detection by common anti-malware programs and other simple system
inspection techniques.
%I am mentioning the kernel panic part because that could be considered an advantage for eBPF, there is less worry about crashing the system
\item \textbf{Kernel-mode} rootkits run at the same level of privilege as
the operating system, thus enjoying unrestricted access to all system
resources. These rootkits usually come as kernel modules or device drivers
and once loaded, they reside in the kernel. This implies that special
attention must be taken to avoid programming errors since they could
potentially corrupt user or kernel memory, resulting in a fatal kernel
panic and a subsequent system reboot, which goes against the original
purpose of maintaining stealth.

Common techniques used for the development of their malicious activities
include hooking system calls made to the kernel by user applications (on
which malicious code is then injected) or modifying data structures in the
kernel to change the data of user programs at runtime. Therefore, trusted
programs on an infected machine can no longer be trusted to operate securely.

Kernel-mode rootkits are usually the most attractive (and difficult to
build) option for a malicious actor, but their installation requires a
complete previous compromise of the system, meaning that administrator or
root privileges must have been already achieved by the attacker, commonly
by the execution of an exploit or a local installation of a privileged user.
\end{itemize}

Historically, kernel-mode rootkits have been tightly associated with
espionage activities on governments, research centers, or key industry
actors by Advanced Persistent Threat (APT) groups
\cite{rootkit_ptsecurity}---state-sponsored or criminal organizations
specialized on long-term operations to gather intelligence and gain
unauthorized persistent access to computer systems. Although rootkits'
functionality is tailored for each specific attack, a common set of
techniques and procedures can be identified being used by these
organizations.

%Yes, I am not mentioning that eBPF comes from "Extended Berkeley Packet %Filters here since apparently it is no longer considered an acronym, we'll %tackle that on the history section
During the last years, a new technology called eBPF has been found to be at
the heart of the latest innovation on the development of rootkits.  eBPF is
a technology incorporated in the 3.18 version of the Linux kernel
\cite{ebpf_linux318} that allows running code in the kernel without the
need of loading a kernel module. Programs are created in a restrictive
version of the C language and compiled into eBPF bytecode, which is loaded
into the kernel via a new bpf() system call. After a mandatory step of
verification by the kernel in which the code is checked to be safe to run,
the bytecode is compiled into native machine instructions. These programs
can then get access to kernel-exclusive functionalities including network
traffic filtering, system calls hooking or tracing.

Although eBPF has built an outstanding environment for the creation of
networking and tracing tools, its ability to run kernel programs without
the need to load a kernel module has attracted the attention of multiple
APT groups. On February 2022, the Chinese security team Pangu Lab reported
about a NSA backdoor that remained unnoticed since 2013. This implant used
eBPF for its networking functionality and infected military and
telecommunications systems worldwide \cite{bvp47_report}. Also on 2022, PwC
reports about a China-based threat actor that has targeted
telecommunications systems with a eBPF-based backdoor \cite{bpfdoor_pwc}.

Current official efforts are focused on porting the eBPF technology to
Windows \cite{ebpf_windows} and Android systems \cite{ebpf_android}, which
could spread the mentioned risks to new platforms.  Therefore, we can
confidently claim that there is a growing interest in researching the
capabilities of eBPF in the context of offensive security, in particular
given its potential to become a common component for modern rootkits and
other offensive tools. This knowledge would be valuable to the computer
security community, both in the context of pen-testing and for analysts
which need to know about the latest trends in malware to prepare their
defenses.


\section{Project objectives} \label{section:project_objectives}
The main objective of this project is to investigate and demonstrate the capabilities of
the eBPF technology that could be weaponized by a malicious actor. In
particular, we will focus on functionalities present in the Linux platform,
given the maturity of eBPF on these environments and which therefore offers
a wider range of possibilities. We will be approaching this study from the
perspective of a threat actor, meaning that we will develop an eBPF-based
rootkit which shows these capabilities live in a current Linux system,
including proof of concepts (PoC) showing specific features, and also by
building a realistic rootkit system which leverages these PoCs and
integrates them into a fully operational implant.

%According to the library guide, previous research should be around here. %Is it the best place tho?
Before narrowing down our objectives and selecting a specific list of
rootkit capabilities to provide using eBPF, we analyze previous research in
this area. The work by Jeff Dileo from NCC Group at DEFCON 27
\cite{evil_ebpf} is particularly relevant, as it discusses for the first
time the ability of eBPF to overwrite userland data, highlighting the
possibility of overwriting the memory of a running process and executing
arbitrary code on it.
%
Subsequent talks on 2021 by Pat Hogan at DEFCON 29 \cite{bad_ebpf}, and by
Guillaume Fournier and Sylvain Afchain from Datadog at DEFCON 29
\cite{ebpf_friends}, research deeper on eBPF's ability to support rootkit
capabilities. In particular, Hogan shows how eBPF can be used to hide the
rootkit's presence from the user and to modify data at system calls, while
Fournier and Afchain built the first instance of an eBPF-based backdoor
with command-and-control (C2) capabilities, enabling to communicate with
the malicious eBPF program by sending network packets to the compromised
machine.

Taking these previous research works into account, and considering the
common functionality usually to be incorporated into a rootkit, the
objectives of our research on eBPF are summarized in the following points:
\begin{itemize}
\item Analyze eBPF's possibilities to hook system calls and kernel
functions.
\item Explore eBPF's potential to read/write arbitrary memory.
\item Research networking capabilities with eBPF packet filters.
\end{itemize}

The knowledge gathered by the previous three pillars will be then used as a
basis for building our rootkit. We will present attack vectors and
techniques different than the ones presented in previous research, although
inevitably we will also tackle common points, which will be clearly
indicated and on which we will try to perform further research. In essence,
our eBPF-based rootkit aims at:
\begin{itemize}
\item Hijacking the execution of user programs while they are running, injecting libraries and executing malicious code, without impacting their normal execution.
\item Featuring a command-and-control module powered by a network backdoor, which can be operated from a remote client. This backdoor should be controlled with stealth in mind, featuring similar mechanisms to those present in rootkits found in the wild.
\item Tampering with user data at system calls, resulting in running malware-like programs and for other malicious purposes.
\item Achieving stealth, hiding rootkit-related files from the user.
\item Achieving rootkit persistence, the rootkit should run after a complete system reboot.
\
\end{itemize}

The rootkit will work in a fresh-install of a Linux system with the following characteristics:
\begin{itemize}
%Maybe a table for this?
\item Distribution: Ubuntu 21.04.
\item Kernel version: 5.11.0-49.
\end{itemize} 


\subsection{Social and economic environment}\label{sec:social_econ_env}
%M-> Mentioned talking about community outreach and its role under pentesting
%TODO Talk about the difference between having always on BPF and always on kernel modules, BPF is consider "safe" in production while it's almost as dangerous (I think this might fit here)

Our world has a growing dependency on digital systems. From the use of
increasingly complex computer systems and networks in business environments
to the thriving industry of consumer devices, the use these digital systems
has shaped today's society and will likely continue to do so in the future. 

As discussed in our project motivation, this has also implied an increasing
relevance of the cybersecurity industry, particularly as a consequence of a
growing number of cyber incidents. The use of malware and, in particular,
ransomware attacks currently stands as one of the major trends among threat
actors, which has impacted both the private and public sector with infamous
attacks. Moreover, during the last decade there has been a steady influx of
targeted high-impact attacks featuring increasingly complex techniques and
attack vectors, which raises the need to stay up to date with the latest
discovered vulnerabilities.

As a response for this growing concern, the computer security community has
proposed multiple procedures and frameworks with the aim of minimizing
these cyber incidents, setting a series of fundamental pillars on which
cyber protection activities on organizations shall be based. As a summary,
these pillars are often defined to revolve around the following actions
\cite{nist_cyber}:
\begin{itemize}
\item Identifying security risks.
\item Protecting computer systems from the identified security risks.
\item Detecting attacks and malicious activity.
\item Responding and taking action when a cyber incident is detected.
\item Recovering after the cyber incident, reducing the impact of the attack.
\end{itemize} 

Focusing our view on the identification of security risks, we can find the
use of adversary simulation exercises, whose purpose is to test the
security of a system or network by emulating the role of a threat actor,
thus trying to find vulnerabilities and exploit them in this controlled
environment so that these security flaws can be patched. There exist two
main types of assessments \cite{pentest_redteam}:
\begin{itemize}
\item Penetration testing (pentesting) exercises, whose aim is mainly to discover which known unpatched vulnerabilities are present in the computer system, attempting to exploit them. These exercises are focused on uncovering as many vulnerabilities as possible and, therefore, in many ocassions the stealth which a real attacker would need while performing such process is disregarded.
\item Red teaming exercises, whose aim is to uncover vulnerabilities as in pentesting, but this process is done quietly (with stealth in mind) and using any resource available, often crafting targeted attacks emulating the procedures which a real threat actor such as an APT would follow. Therefore, the goal is not to find as many vulnerabilities as possible, but rather these exercises take place in already security-audited environments to be further protected from targeted cyber attacks.
\end{itemize}

Our efforts to better understand the offensive capabilities offered by eBPF
are relevant to both pentesters and red teamers. For the security
professionals performing these exercises, it is essential not only to know
about the latest security trends being used by threat actors, but also to
have expertise on the techniques and attack vectors employed in these cyber
attacks. Therefore, a research on last-generation rootkits using eBPF is
useful and relevant for the security community, which will benefit
positively from having an open-source rootkit showcasing the offensive
capabilities of the eBPF technology.

Consequently, given the growing importance of eBPF for offensive security,
it also undertakes a positive impact in the area of defensive security. In
particular, it presents a clear example on how eBPF may be weaponized for
malicious purposes, thus inspiring system administrators and other
professionals to consider eBPF programs as a possible attack vector. As we
will show during this research work, our rootkit can achieve similar
capabilities compared to classic rootkits based on kernel modules. However,
while kernel modules are usually considered a risk and might be restricted
(or its activity, particularly loading a new one, easy to flag), in many
environments eBPF remains as a technology often available by default and
not considered in the security framework of most organizations. Therefore
our project aims to raise awareness on this regard.


\section{Regulatory framework}
As discussed in Section \ref{sec:social_econ_env}, this project is tightly
related to both cybersecurity in general and to offensive tools in
particular. We will now analyze the most relevant frameworks that regulate
or are related to both activities with the purpose of studying how they can
be applied to the development of our rootkit.

\subsection{NIST Cybersecurity Framework}
In the case of activities related to cybersecurity, multiple standards and
frameworks regulate the best practices and guidelines to follow for
managing cyber risks. One of the most relevant is the Framework for
Improving Critical Infrastructure Cybersecurity by the National Institute
of Standards and Technology (NIST) \cite{nist_cyber}. This is the framework
that regulates the 5 pillars of cyber risk mamagement which we have
discussed in Section \ref{sec:social_econ_env}, describing the needs
originated by each pillar (in the framework named as 'Categories') and the
actions needed for meeting the requirements of each of these needs
('Subcategories'). In particular, we can identify the following procedures
on each of these pillars relevant in our context:
\begin{itemize}
\item With respect to the 'Identify' pillar, the framework highlights the need for Asset Management and Risk Assessment between others:
	\begin{itemize}
	\item Asset Management refers to the identification and management of data, devices and systems, so as to consider their relative importance in the organization objectives and cyber risk strategy. This involves inventorizing all software platforms and applications used in the organization. In our case, maintaining strict control over the software present on each system reduces the risk of an infection.
	\item Risk Assessment refers to the identification of the vulnerabilities of each of the organization assets, receiving intelligence about cyber threat from external forums and sources, and identifying the risks, likelihook and impact of an specific risk. In the case of eBPF, it relates to the identification of devices and systems supporting this technology and assessing the risk of malicious eBPF programs using, for instance, this research work as one of the external sources described in the framework.
	\end{itemize}
\item With respect to the 'Protect' pillar, it describes the need for Identify Management, Authentication and Access Control, together with the use of Protective Technologies, between others:
	\begin{itemize}
	\item With respect to Identify Management, Authentication and Access Control, the framework describes the need to use the principle of least privilege and management of access permissions, that is, assigning the least permissions possible to each system component. In the case of our rootkit, this is particularly relevant given that it needs to be executed as root or by an user with specific capabilities, as we will explain in Section \ref{section:ebpf_security}.
	\item Protective Techniques are solutions with the aim of managing the security of systems and organization assets. In this category we can find the storage of log records about activity on the system, and the protection of communication in the organization network. In the case of our rootkit, maintaining logs and non-plaintext connection means the rootkit shall increase its complexity and invest some resources into stealth functionalities.
	\end{itemize}
\item With respect to the 'Detection' pillar, the framework describes the need for an Anomalies and Events policy and Security Continuous Monitoring, between others.
	\begin{itemize}
	\item An Anomalies and Events policy relates to detecting and analysing the risk of suspicious events in the network and at systems. This includes gathering information about each of the events in the system using multiple sensors, analysing the data and the origin of each, and analysing the impact of them. In the context of our rootkit, a proper management of system events could disclose the rootkit activities (e.g.: when it is loaded or when it executes user process) although this can be mitigated by the use of stealth techniques.
	\item Security Continuous Monitoring relates to the monitoring of information systems and organization assets with the purpose of identifying cybersecurity-related events. Some actions described in this regard by the framework include monitoring the network for events, scanning programs for malicious code, and implementing monitoring policies for detecting unauthorized software and network connections. In our case, these all belong to recommended steps an organization shall take to prevent and early detect an infection by a rootkit (and therefore the rootkit will attempt to circumvent these actions by means of stealth techniques).
	\end{itemize}
\item With respect to the 'Respond' pillar, the framework describes the need for Analysis, between others:
	\begin{itemize}
	\item Analysis refers to conducting response processes after the detection of a cyber attack, analysing the causes to support recovery activities. This includes analysing the events gathered in log traces and other sensors, performing a forensic investigation on the cyber attack. In our case, an organization infected by an eBPF rootktit needs to analyse the source of the compromise and analyse its functioning so as to know the extent of the infection.
	\end{itemize}
\item Finally, with respect to the 'Recover' pillar the NIST framework shows the need for Recovery Planning and Improvements policies between others:
	\begin{itemize}
	\item Recovery Planning relates to the process of restoring the original state of systems and assets impacted by a cyber incident. In the case of our rootkit, previous conduced analysis should unveil the rootkit persistence capabilities, so that in this step these are nullified and the eBPF programs belonging to the rootkit are unloaded.
	\item Improvements relates to the need to incorporate the new knowledge and leasons learned after the cyber incident into existing organization policies. In the case of an organization infected by an eBPF rootkit, it would proceed to adopt protective measures for mitigating a similar attack, such as blocking its use. 
	\end{itemize}
\end{itemize}


\subsection{MITRE ATT\&CK}
MITRE Adversarial Tactics, Techniques, and Common Knowledge (MITRE ATT\&CK) is a framework collecting knowledge about adversarial techniques, that is, techniques, methodologies and offensive actions followed by threat actors that can be used against computer systems. This is an useful framework for red teaming or pentesting activities performing adversary emulation exercises, since it details adversary behaviours and the techniques being used, which can help to build multiple attack scenarios. Moreover, it is also relevant for professionals in charge of defending a system, since they can prepare and design mitigations for the techniques described in the framework \cite{mitre_blog} \cite{mitre_blog_2}.

A relevant aspect of the MITRE ATT\&CK framework is the MITRE ATT\&CK Matrix, which contains all the adversarial techniques organized as 'tactics'. These tactics are the objective of the adversary, which it aims to achieve by using each corresponding technique. Therefore, the MITRE ATT\&CK Matrix shows a list of columns, where each column is one tactic (one adversary objective), and each row on that column shows the techniques associated to that tactic, explaining how that objective can be achieved. Additionally, different matrices exist depending on the platform. In this project, we will consider the Linux Matrix \cite{mitre_matrix_linux}.

Using the Linux MITRE ATT\&CK matrix, red teamers and pentesters can evaluate the techniques incorporated in our rootkit according to the tactics described in the framework. These tactics range from an initial access step (which usually preceeds the adversary attack) to the description of the impact that the attack has on the normal functioning of the system. In summary, these tactics are the following:
\begin{itemize}
\item \textbf{Initial access}, comprising techniques to gain a foothold in the system or network, such as spear-phising attacks, with which the adversary may obtain credentials that can be used to achieve access to a machine.
\item \textbf{Execution}, comprising techniques used to execute code in the target system. This includes exploiting vulnerabilities that lead to Remote Code Execution (RCE).
\item \textbf{Persistence}, comprising techniques that enable the adversary to maintain access at the system after the initial foothold, independently on the actions performed by the target machine (which may reboot or change passwords). One of these techniques is using scheduled jobs (such as Cron, which will be used in our rootkit).
\item \textbf{Privelege escalation}, consisting on techniques used to achieve privileged access in a machine from an original unprivileged access position. This includes techniques that abuse the elevation control mechanisms of a system, such as sudo, which will be used in our rootkit.
\item \textbf{Defense evasion}, comprising techniques to avoid detection after a machine infection. This includes hiding processes, files and directories, or network connections related to the adversary activities.
\item \textbf{Credential access}, comprising techniques to steal passwords and other credentials from the system. An example of such a technique is sniffing the network for credentials transmitted in plaintext.
\item \textbf{Discovery}, comprising techniques used by the adversary to gather knowledge about the target system and the available actions it may engage with (once it has access to the system, e.g. execution of commands). This includes techniques such as listing running processes or scanning the network.
\item \textbf{Lateral movement}, comprising techniques allowing for pivoting through systems from the internal network after having compromised the original target machine. An example of a technique acomplishing this is the exploitation of vulnerabilities that can only be exploited from the local network.
\item \textbf{Collection}, comprising techniques to gather critical information in the compromised system, with the purpose of, often, leaking them. In contrast to the discovery tactic, collection techniques do not search for possible targets in the compromised system, but rather use this knowledge to locate key data and exfiltrate it.
\item \textbf{Command and control}, comprising techniques that enable an attacker to communicate with the compromised machine, usually issuing commands and actions to be executed by it. Since network traffic belonging to the adversary activities should remain hidden, techniques belonging to this category include encoding or obfuscating data so that they can be transmitted secretly.
\item \textbf{Exfiltration}, containing the techniques used for exfiltrating the data collected during the Collection step, transmitting this data outside of the compromised network. The use of C2 encrypted channels is a recurrent technique. Our rootkit will use this and other communication means for sending data from the infected to the attacker machine.
\item \textbf{Impact}, comprising techniques used by the adversary to manipulate or destroy data, and to disrupt the normal services at the compromised machine. A common technique belonging to this tactic is the modification of system files, which we will use to implement multiple of the rootkit functionalities.
\end{itemize}

\subsection{Software licenses}
Finally, it must be noted that this project uses the libbpf library
\cite{libbpf_github}, as described in Section \ref{subsection:libbpf}, for
the development of our eBPF rootkit. This library is licensed under dual
BSD 2-clause license and GNU LGPL v2.1 license. 
%Should I say something else? I usually license my own projects under GPLv3 because I don't like corporations taking the code, but I guess I am restricted to use the Creative Commons license.


\section{Structure of the document}
This section details the structure of this document and the contents of each chapter with the aim of offering a summarized view and improving its readibility.

\textbf{Chapter 1: Introduction} describes the motivation behind the project and the purposes it aims to achieve, presenting the functionalities expected to be implemented in our rootkit. It also discusses the regulatory frameworks and the environmental issues related to the development of the research work.

\textbf{Chapter 2: Background} presents all the concepts needed for the later discussion of offensive capabilities. It includes an in-depth description of the eBPF system, a brief discussion of its security features and multiple alternatives for developing eBPF programs. It also discusses networking concepts and an offers an overview on the memory architecture at Linux systems, showing basic attacks and techniques that are the basis of those later incorporated to the rootkit.

\textbf{Chapter 3: Analysis of offensive capabilities of eBPF} discusses the possible capabilities of a malicious eBPF program, describing which features of the eBPF system could be weaponized and used for offensive purposes.

\textbf{Chapter 4: Design of a malicious eBPF rootkit} describes the architecture of the rootkit we have developed, offering a comprehensive view of the different techniques and attacks designed and implemented on each of the rootkit modules and components.

\textbf{Chapter 5: Evaluation} analyses whether the rootkit developed meets the expected functionality proposed in the project objectives by testing the rootkit capabilities in a simulated testing environment. We will prepare a virtualized network consisting of two connected machines, where one is the infected host and the other belongs to the attacker, proceeding to test every rootkit functionality.

\textbf{Chapter 6: Related work} includes a comprehensive review of previous work on UNIX/Linux rootkits, their main types and most relevant examples. We also offer a comparison in terms of techniques and functionality with previous families. In particular, we highlight the differences of our eBPF rootkit with respect to others that rely on traditional methods, and also to those already built using eBPF.

\textbf{Chapter 7: Budget} describes the costs associated to the development of this project, including personnel, hardware and software related costs.

\textbf{Chapter 8: Conclusions and future work} revisits the project objectives, discusses the work presented in this document, and describes possible future research lines.

\section{Code availability}
%Is it ok to reference the repo as a cite? Maybe it's better writing the link directly?
All the source code belonging to the rootkit development can be visited publicly at the GitHub repository \url{https://github.com/h3xduck/TripleCross} \cite{triplecross_github}. The most important folders and files of this repository are described in Table \ref{table:triplecross_dirs}.

%I can go with more detail if needed. Is it needed?
\begin{table}[htbp]
\begin{tabular}{|>{\centering\arraybackslash}p{4cm}|>{\centering\arraybackslash}p{10cm}|}
\hline
\textbf{DIRECTORY} & \textbf{DESCRIPTION}\\
\hline
\hline
src/client & Source code of rootkit client.\\
\hline
src/client/lib & RawTCP\_Lib shared library.\\
\hline
src/common & Constants and configuration for the rootkit. It also includes the implementation of elements common to the eBPF and user space side of the rootkit, such as the ring buffer.\\
\hline
src/ebpf & Source code of the eBPF programs used by the rootkit.\\
\hline
src/helpers & Includes programs for testing rootkit modules functionality, and the malicious program and library used at the execution hijacking and library injection modules respectively.\\
\hline
src/libbpf & Contains the libbpf library, integrated with the rootkit.\\
\hline
src/user & Source code of the user land programs used by the rootkits.\\
\hline
src/vmlinux & Headers containing the definition of kernel data structures (this is the recommended method when using libbpf).\\
\hline
\end{tabular}
\caption{Relevant directories at TripleCross repository.}
\label{table:triplecross_dirs}
\end{table} 

Additionally, the source code of the RawTCP\_Lib library can be visited publicly at its own GitHub directory \url{https://github.com/h3xduck/RawTCP_Lib} \cite{rawtcp_lib}.


\chapter{Background}
This chapter is dedicated to a study of all the background needed for our research into offensive eBPF applications. Although our rootkit has been developed using a library that will provide us with a layer of abstraction over the underlying operations, this background is needed to understand how eBPF is embedded in the kernel and which capabilities and limits we can expect to achieve with it.

Firstly, we will analyse the origins of the eBPF technology, understanding what it is and how it works, and discuss the reasons why it is a necessary component of the Linux kernel today. Afterwards, we will cover the main features of eBPF in detail and discuss the security features incorporated in the system, together with an study of the currently existing alternatives for developing eBPF applications.

Finally, we will offer an overview into multiple aspects of the Linux system (memory, networking and executable files), which will be critical during the design of the offensive techniques incorporated in our rootkit.

\section{BPF}
% Is it ok to have sections / chapters without individual intros?
In this section we will detail the origins of eBPF in the Linux kernel. By offering us background into the earlier versions of the system, the goal is to acquire insight on the design decisions included in modern versions of eBPF.

\subsection{Introduction to the BPF system}
Nowadays eBPF is not officially considered to be an acronym anymore \cite{ebpf_io}, but it remains largely known as "extended Berkeley Packet Filters", given its roots in the Berkeley Packet Filter (BPF) technology, now known as classic BPF.

BPF was introduced in 1992 by Steven McCanne and Van Jacobson in the paper "The BSD Packet Filter: A New Architecture for User-level Packet Capture" \cite{bpf_bsd_origin}, as a new filtering technology for network packets in the BSD platform. It was first integrated in the Linux kernel on version 2.1.75 \cite{ebpf_history_opensource}.


\begin{figure}[htbp]
	\centering
	\includegraphics[width=12cm, keepaspectratio=true]{classic_bpf.jpg}
	\caption{Functionality of classic BPF. Based on the figure at the original paper \cite{bpf_bsd_origin_bpf_page2}.}
	\label{fig:classif_bpf}
\end{figure}

Figure \ref{fig:classif_bpf} shows how BPF was integrated in the existing network packet processing by the kernel. After receiving a packet via the Network Interface Controller (NIC) driver, it would first be analysed by BPF filters, which are programs directly developed by the user. This filter decides whether the packet is to be accepted by analysing the packet properties, such as its length or the type and values of its headers. If a packet is accepted, the filter proceeds to decide how many bytes of the original buffer are passed to the application at the user space. Otherwise, the packet is redirected to the original network stack, where it is managed as usual.


\subsection{The BPF virtual machine} \label{subsection:bpf_vm}
In a technical level, BPF comprises both the BPF filter programs developed by the user and the BPF module included in the kernel which allows for loading and running the BPF filters. This BPF module in the kernel works as a virtual machine \cite{bpf_bsd_origin_bpf_page1}, meaning that it parses and interprets the filter program by providing simulated components needed for its execution, turning into a software-based CPU. Because of this reason, it is usually referred as the BPF Virtual Machine (BPF VM). The BPF VM comprises the following components:
\begin{itemize}
\item \textbf{An accumulator register}, used to store intermediate values of operations.
\item \textbf{An index register}, used to modify operand addresses, it is usually incorporated to optimize vector operations \cite{index_register}.
\item \textbf{A scratch memory store}, a temporary storage.
\item \textbf{A program counter}, used to point to the next machine instruction to execute in a filter program.
\end{itemize}


\subsection{Analysis of a BPF filter program} \label{subsection:analysis_bpf_filter_prog}
As we mentioned in section \ref{subsection:bpf_vm}, the components of the BPF VM are used to support running BPF filter programs. A BPF filter is implemented as a boolean function:
\begin{itemize}
\item If it returns \textit{true}, the kernel copies the packet to the application.
\item If it returns \textit{false}, the packet is not accepted by the filter (and thus the network stack will be the next to operate it).
\end{itemize}

Figure \ref{fig:cbpf_prog} shows an example of a BPF filter upon receiving a packet. In the figure, green lines indicate that the condition is true and red lines that it is evaluated as false. Therefore, the execution works as a control flow graph (CFG) which ends on a boolean value \cite{bpf_bsd_origin_bpf_page5}. The figure presents an example BPF program which accepts the following frames:
\begin{itemize}
\item Frames with an IP packet as a payload directed from IP address X.
\item Frames with an IP packet as a payload directed towards IP address Y.
\item Frames belonging to the ARP protocol and from IP address Y.
\item Frames not from the ARP protocol directed from IP address Y to IP address X.
\end{itemize}

\begin{figure}[ht]
	\centering
	\includegraphics[width=8cm]{cbpf_prog.jpg}
	\caption{Execution of a BPF filter.}
	\label{fig:cbpf_prog}
\end{figure}


\subsection{BPF bytecode instruction format}
In order to implement the CFG to be run at the BPF VM, BPF filter programs are made up of BPF bytecode, which is defined by a new BPF instruction set. Therefore, a BPF filter program is an array of BPF bytecode instructions \cite{bpf_organicprogrammer_analysis}.


\begin{table}[htbp]
\begin{tabular}{|c|c|c|c|c|}
\hline
& OPCODE & JT & JF & K\\
\hline
BITS & 16 & 8 & 8 & 32\\
\hline
\end{tabular}
\caption{BPF instruction format.}
\label{table:bpf_inst_format}
\end{table}

Table \ref{table:bpf_inst_format} shows the format of a BPF bytecode instruction. As it can be observed, it is a fixed-length 64-bit instruction composed of:
\begin{itemize}
\item An \textbf{opcode}, similar to assembly opcode, it indicates the operation to be executed.
\item Field \textbf{jt} indicates the offset to the next instruction to jump in case a condition is evaluated as \textit{true}.
\item Field \textbf{jf} indicates the offset to the next instruction to jump in case a condition is evaluated as \textit{false}.
\item Field \textbf{k} is miscellaneous and its contents vary depending on the instruction opcode.
\end{itemize}

Figure \ref{fig:bpf_instructions} shows how BPF instructions are defined according to the BPF instruction set. As we mentioned, similarly to assembly, instructions include an opcode which indicates the operation to execute, and the multiple arguments defining the arguments of the operation. The table shows, in order by rows, the following instruction types \cite{bpf_bsd_origin_bpf_page8}:
\begin{itemize}
\item Rows 1-4 are \textbf{load instructions}, copying the addressed value into the index or accumulator register.
\item Rows 4-6 are \textbf{store instructions}, copying the accumulator or index register into the scratch memory store.
\item Rows 7-11 are \textbf{jump instructions}, changing the program counter register. These are usually present on each node of the CFG and evaluate whether the condition to be evaluated is true or not.
\item Rows 12-19 and 21-22 are \textbf{arithmetic and miscellaneous instructions}, performing operations usually needed during the program execution.
\item Row 20 is a \textbf{return instruction}, it is positioned in the final end of the CFG, and indicate whether the filter accepts the packet (returning true) or otherwise rejects it (return false).
\end{itemize}

\begin{figure}[htbp]
	\centering
	\includegraphics[width=8cm]{bpf_instructions.png}
	\caption{Supported classic BPF instructions, as shown by McCanne and Jacobson \cite{bpf_bsd_origin_bpf_page7}}
	\label{fig:bpf_instructions}
\end{figure}

The column \textit{addr modes} in figure \ref{fig:bpf_instructions} describes how the parameters of a BPF instruction are referenced depending on the opcode. The address modes are detailed in figure \ref{fig:bpf_address_mode}. As it can be observed, parameters may consist of immediate values, offsets to memory positions or on the packet, the index register or combinations of the previous.

\begin{figure}[htbp]
	\centering
	\includegraphics[width=8cm]{bpf_address_mode.png}
	\caption{BPF address modes, as shown by McCanne and Jacobson \cite{bpf_bsd_origin_bpf_page8}}
	\label{fig:bpf_address_mode}
\end{figure}

\subsection{An example of BPF filter with tcpdump}
At the time, by filtering packets before they are handled by the kernel instead of using a user-level application, BPF offered a performance improvement between 10 and 150 times the state-of-the art technologies of the moment \cite{bpf_bsd_origin_bpf_page1}. Since then, multiple popular tools began to use BPF, such as the network tracing tool \textit{tcpdump} \cite{tcpdump_page}.

\textit{tcpdump} is a command-line tool that enables to capture and analyse the network traffic going through the system. It works by setting filters on a network interface, so that it shows the packets that are accepted by the filter. Still today, \textit{tcpdump} uses BPF for the filter implementation. Figure \ref{fig:bpf_tcpdump_example} shows an example of BPF code used by \textit{tcpdump} to implement a simple filter.

\begin{figure}[htbp]
	\centering
	\includegraphics[width=10cm]{tcpdump_example.png}
	\caption{BPF bytecode tcpdump needs to set a filter to display packets directed to port 80.}
	\label{fig:bpf_tcpdump_example}
\end{figure}

In figure \ref{fig:bpf_tcpdump_example} we can see how tcpdump sets a filter to display traffic directed to all interfaces (\textit{-i any}) directed to port 80. Flag \textit{-d} instructs tcpdump to display BPF bytecode.

In the example, using the \textit{jf} and \textit{jt} fields, we can label the nodes of the CFG described by the BPF filter. Figure \ref{fig:tcpdump_ex_sol} describes the shortest graph path that a true comparison will need to follow to be accepted by the filter. Note how instruction 010 is checking the value 80, the one our filter is looking for in the port.

\begin{figure}[htbp]
	\centering
	\includegraphics[width=6cm]{cBPF_prog_ex_sol.png}
	\caption{Shortest path in the CFG described in the example of figure \ref{fig:bpf_tcpdump_example} that a packet needs to follow to be accepted by the BPF filter set with \textit{tcpdump}.}
	\label{fig:tcpdump_ex_sol}
\end{figure}

\section{Modern eBPF} \label{section:modern_ebpf}
This section discusses the current state of eBPF in the Linux kernel. By building on the previous architecture described in classic BPF, we will be able to provide a comprehensive picture of the underlying infrastructure in which eBPF relies today.

The addition of classic BPF in the Linux kernel set the foundations of eBPF, but nowadays it has already extended its presence to many other components other than traffic filtering. Similarly to how BPF filters were included in the networking module of the Linux kernel, we will now study the necessary changes made in the kernel to support these new program types. Table \ref{table:ebpf_history} shows the main updates that were incorporated and shaped modern eBPF of today.

\begin{table}[htbp]
\begin{tabular}{|c|c|c|}
\hline
Description & Kernel version & Year\\
\hline
\hline
\textit{BPF}: First addition in the kernel & 2.1.75 & 1997\\
\textit{BPF+}: New JIT assembler & 3.0 & 2011\\
\textit{eBPF}: Added eBPF support & 3.15 & 2014\\
\textit New bpf() syscall & 3.18 & 2014\\
\textit Introduction of eBPF maps & 3.19 & 2015\\
\textit eBPF attached to kprobes & 4.1 & 2015\\
\textit Introduction of Traffic Control & 4.5 & 2016\\
\textit eBPF attached to tracepoints & 4.7 & 2016\\
\textit Introduction of XDP & 4.8 & 2016\\


\hline
\end{tabular}
\caption{Relevant eBPF updates. Note that only those relevant for our research objectives are shown. This is a selection of the official complete table at \cite{ebpf_funcs_by_ver}.}
\label{table:ebpf_history}
\end{table}

As it can be observed in the table above, the main breakthrough happened in the 3.15 version, where Alexei Starovoitov, along with Daniel Borkmann, decided to expand the capabilities of BPF by remodelling the BPF instruction set and overall architecture \cite{brendan_gregg_bpf_book}.

Figure \ref{fig:ebpf_architecture} offers an overview of the current eBPF architecture. During the subsequent subsections, we will proceed to explain its components in detail.

\begin{figure}[htbp]
	\centering
	\includegraphics[width=15cm]{ebpf_arch.jpg}
	\caption{eBPF architecture in the Linux kernel and the process of loading an eBPF program. Based on \cite{brendan_gregg_bpf_book} and \cite{ebpf_io_arch}.}
	\label{fig:ebpf_architecture}
\end{figure}

\subsection{eBPF instruction set} \label{subsection:ebpf_inst_set}
The eBPF update included a complete remodel of the instruction set architecture (ISA) of the BPF VM. Therefore, eBPF programs will need to follow the new architecture in order to be interpreted as valid and executed.

Table \ref{table:ebpf_inst_format} shows the new instruction format for eBPF programs \cite{ebpf_inst_set}. As it can be observed, it is a fixed-length 64-bit instruction. The new fields are similar to x86\_64 assembly, incorporating the typically found immediate and offset fields, and source and destination registers \cite{8664_inst_set_specs}. Similarly, the instruction set is extended to be similar to the one typically found on x86\_64 systems, the complete list can be consulted in the official documentation \cite{ebpf_inst_set}.
%Should I talk about assembly or this more in detail?

\begin{table}[htbp]
\begin{tabular}{|c|c|c|c|c|c|}
\hline
& IMM & OFF & SRC & DST & OPCODE \\
\hline
BITS & 32 & 16 & 4 & 4 & 8\\
\hline
\end{tabular}
\caption{eBPF instruction format.}
\label{table:ebpf_inst_format}
\end{table}

With respect to the BPF VM registers, they get extended from 32 to 64 bits of length, and the number of registers is incremented to 10, instead of the original accumulator and index registers. These registers are also adapted to be similar to those in assembly, as it is shown in table \ref{table:ebpf_regs}.

\begin{table}[htbp]
\begin{tabular}{|c|c|m{21em}|}
\hline
eBPF register & x86\_64 register & Purpose\\
\hline
r0 & rax & Return value from functions and exit value of eBPF programs\\
r1 & rdi & Function call argument 1\\
r2 & rsi & Function call argument 2\\
r3 & rdx & Function call argument 3\\
r4 & rcx & Function call argument 4\\
r5 & r8 & Function call argument 5\\
r6 & rbx & Callee saved register, value preserved between calls\\
r7 & r13 & Callee saved register, value preserved between calls\\
r8 & r14 & Callee saved register, value preserved between calls\\
r9 & r15 & Callee saved register, value preserved between calls\\
r10 & rbp & Frame pointer for stack, read only\\
\hline
\end{tabular}
\caption{eBPF registers and their purpose in the BPF VM. \cite{ebpf_inst_set} \cite{ebpf_starovo_slides}.}
\label{table:ebpf_regs}
\end{table}

\subsection{JIT compilation}
We mentioned in subsection \ref{subsection:ebpf_inst_set} that eBPF registers and instructions describe an almost one-to-one correspondence to those in x86 assembly. This is in fact not a coincidence, but rather it is with the purpose of improving a functionality that was included in Linux kernel 3.0, called Just-in-Time (JIT) compilation \cite{ebpf_JIT} \cite{ebpf_JIT_demystify_page13}.

JIT compiling is an extra step that optimizes the execution speed of eBPF programs. It consists of translating BPF bytecode into machine-specific instructions, so that they run as fast as native code in the kernel. Machine instructions are generated during runtime, written directly into executable memory and executed there \cite{ebpf_JIT_demystify_page14}.

Therefore, when using JIT compiling (a setting defined by the variable \textit{bpf\_jit\_enable} \cite{jit_enable_setting}, BPF registers are translated into machine-specific registers following their one-to-one mapping and bytecode instructions are translated into machine-specific instructions \cite{ebpf_starovo_slides_page23}. There no longer exists an interpretation step by the BPF VM, since we can execute the code directly \cite{brendan_gregg_bpf_book_bpf_vm}.

The programs developed during this project will always have JIT compiling active.


\subsection{The eBPF verifier} \label{subsection:ebpf_verifier}
We introduced in figure \ref{fig:ebpf_architecture} the presence of the so-called eBPF verifier. Provided that we will be loading programs in the kernel from user space, these programs need to be checked for safety before being valid to be executed.

The verifier performs a series of tests which every eBPF program must pass in order to be accepted. Otherwise, user programs could leak privileged data, result in kernel memory corruption, or hang the kernel in an infinite loop, between others. Therefore, the verifier limits multiple aspects of eBPF programs so that they are restricted to the intended functionality, whilst at the same time offering a reasonable amount of freedom to the developer.

The following are the most relevant checks that the verifier performs in eBPF programs \cite{ebpf_verifier_kerneldocs} \cite{ebpf_JIT_demystify_page17-22}:
\begin{itemize}
\item Tests for ensuring overall control flow safety:
	\subitem No loops allowed (bounded loops accepted since kernel version 5.3 \cite{ebpf_bounded_loops}.
	\subitem Function call and jumps safety to known, reachable functions.
	\subitem Sleep and blocking operations not allowed (to prevent hanging the kernel).
\item Tests for individual instructions:
	 \subitem Divisions by zero and invalid shift operations.
	 \subitem Invalid stack access and invalid out-of-bound access to data structures.
	 \subitem Reads from uninitialized registers and corruption of pointers.
\end{itemize}

These checks are performed by two main algorithms:
\begin{itemize}
\item Build a graph representing the eBPF instructions (similar to the one shown in section \ref{subsection:analysis_bpf_filter_prog}. Check that it is in fact a direct acyclic graph (DAG), meaning that the verifier prevents loops and unreachable instructions.
\item Simulate execution flow by starting on the first instruction and following each possible path, observing at each instruction the state of every register and of the stack.
\end{itemize}

\subsection{eBPF maps} \label{subsection:ebpf_maps}
An eBPF map is a generic storage for eBPF programs used to share data between user and kernel space, to maintain persistent data between eBPF calls and to share information between multiple eBPF programs \cite{ebpf_maps_kernel}.

A map consists of a key + value tuple. Both fields can have an arbitrary data type, the map only needs to know the length of the key and the value field at its creation \cite{bpf_syscall}. Programs can open maps by specifying their ID, and lookup or delete elements in the map by specifying its key, also insert new ones by supplying the element value and they key to store it with.

Therefore, creating a map requires a struct with the fields shown in table \ref{table:ebpf_map_struct}.

\begin{table}[htbp]
\begin{tabular}{|c|c|}
\hline
FIELD & VALUE\\
\hline
type & Type of eBPF map. Described in table \ref{table:ebpf_map_types}\\
key\_size & Size of the data structure to use as a key\\
value\_size & Size of the data structure to use as value field\\
max\_entries & Maximum number of elements in the map\\
\hline
\end{tabular}
\caption{Common fields for creating an eBPF map.}
\label{table:ebpf_map_struct}
\end{table}

Table \ref{table:ebpf_map_types} describes the main types of eBPF maps that are available for use. During the development of our rootkit, we will mainly focus on hash maps (BPF\_MAP\_TYPE\_HASH), provided that they are simple to use and we do not require of any special storage for our research purposes.

\begin{table}[htbp]
\begin{tabular}{|c|>{\centering\arraybackslash}p{10cm}|}
\hline
TYPE & DESCRIPTION\\
\hline
BPF\_MAP\_TYPE\_HASH & A hast table-like storage, elements are stored in tuples.\\
BPF\_MAP\_TYPE\_ARRAY & Elements are stored in an array.\\
BPF\_MAP\_TYPE\_RINGBUF & Map providing alerts from kernel to user space, covered in subsection \ref{subsection:bpf_ring_buf}\\
BPF\_MAP\_TYPE\_PROG\_ARRAY & Stores descriptors of eBPF programs\\
\hline
\hline
\end{tabular}
\caption{Types of eBPF maps. Only those used in our rootkit are displayed, the full list can be consulted in the man page \cite{bpf_syscall}}
\label{table:ebpf_map_types}
\end{table}

\subsection{The eBPF ring buffer} \label{subsection:bpf_ring_buf}
eBPF ring buffers are a special kind of eBPF maps, providing a one-way directional communication system, going from an eBPF program in the kernel to a user space program that subscribes to its events.

%TODO DIAGRAM OF A TYPICAL RING BUFFER

\subsection{The bpf() syscall} \label{subsection:bpf_syscall}
The bpf() syscall is used to issue commands from user space to kernel space in eBPF programs. This syscall is multiplexor, meaning that it can perform a great range of actions, changing its behaviour depending on the parameters.

The main operations that can be issued are described in table \ref{table:ebpf_syscall}:

\begin{table}[htbp]
\begin{tabular}{|c|>{\centering\arraybackslash}p{5cm}|>{\centering\arraybackslash}p{5cm}|}
\hline
COMMAND & ATTRIBUTES & DESCRIPTION\\
\hline
\hline
BPF\_MAP\_CREATE & Struct with map info as defined in table \ref{table:ebpf_map_struct} & Create a new map\\
\hline
BPF\_MAP\_LOOKUP\_ELEM & Map ID, and struct with key to search in the map & Get the element on the map with a specific key\\
\hline
BPF\_MAP\_UPDATE\_ELEM & Map ID, and struct with key and new value & Update the element of an specific key with a new value\\
\hline
BPF\_MAP\_DELETE\_ELEM & Map ID and struct with key to search in the map & Delete the element on the map with an specific key\\
\hline
BPF\_PROG\_LOAD & Struct describing the type of eBPF program to load & Load an eBPF program in the kernel\\
\hline
\end{tabular}
\caption{Types of syscall actions. Only those relevant to our research are shown the full list and attribute details can be consulted in the man page \cite{bpf_syscall}}
\label{table:ebpf_syscall}
\end{table}

With respect to the program type indicated with BPF\_PROG\_LOAD, this parameter indicates the type of eBPF program, setting the context in the kernel in which it will run, and to which modules it will have access to. The types of programs relevant for our research are described in table \ref{table:ebpf_prog_types}.

\begin{table}[htbp]
\begin{tabular}{|c|>{\centering\arraybackslash}p{5cm}|}
\hline
PROGRAM TYPE & DESCRIPTION\\
\hline
\hline
BPF\_PROG\_TYPE\_KPROBE & Program to instrument code to an attached kprobe\\
\hline
BPF\_PROG\_TYPE\_UPROBE & Program to instrument code to an attached uprobe\\
\hline
BPF\_PROG\_TYPE\_TRACEPOINT & Program to instrument code to a syscall tracepoint\\
\hline
BPF\_PROG\_TYPE\_XDP & Program to filter, redirect and monitor network events from the Xpress Data Path\\
\hline
BPF\_PROG\_TYPE\_SCHED\_CLS & Program to filter, redirect and monitor events using the Traffic Control classifier\\
\hline
\end{tabular}
\caption{Types of eBPF programs. Only those relevant to our research are shown. The full list and attribute details can be consulted in the man page \cite{bpf_syscall}.}
\label{table:ebpf_prog_types}
\end{table}

In section \ref{section:TODO}, we will proceed to analyse in detail the different program types and what capabilities` they offer.

\subsection{eBPF helpers} \label{subsection:ebpf_helpers}
Our last component to cover of the eBPF architecture are the eBPF helpers. Since eBPF programs have limited accessibility to kernel functions (which kernel modules commonly have free access to), the eBPF system offers a set of limited functions called helpers \cite{ebpf_helpers}, which are used by eBPF programs to perform certain actions and interact with the context on which they are run. The list of helpers a program can call varies between eBPF program types, since different programs run in different contexts.

It is important to highlight that, just like commands issued via the bpf() syscall can only be issued from the user space, eBPF helpers correspond to the kernel-side of eBPF program exclusively. Note that we will also find a symmetric correspondence to those functions of the bpf() syscall related to map operations (since these are accessible both from user and kernel space).

Table \ref{table:ebpf_helpers} lists the most relevant general-purpose eBPF helpers we will use during the development of our project. We will later detail those helpers exclusive to an specific eBPF program type in the sections on which they are studied.

\begin{table}[htbp]
\begin{tabular}{|c|>{\centering\arraybackslash}p{10cm}|}
\hline
eBPF helper & DESCRIPTION\\
\hline
\hline
bpf\_map\_lookup\_elem() & Query an element with a certain key in a map\\
\hline
bpf\_map\_delete\_elem() & Delete an element with a certain key in a map\\
\hline
bpf\_map\_update\_elem() & Update the value of the element with a certain key in a map\\
\hline
bpf\_probe\_read\_user() & Attempt to safely read data at an specific user address into a buffer\\
\hline
bpf\_probe\_read\_kernel() & Attempt to safely read data at an specific kernel address into a buffer\\
\hline
bpf\_trace\_printk() & Similarly to printk() in kernel modules, writes buffer in \/sys\/kernel\/debug\/tracing\/trace\_pipe\\
\hline
bpf\_get\_current\_pid\_tgid() & Get the process' Process Id (PID) and thread group id (TGID)\\
\hline
bpf\_get\_current\_comm() & Get the name of the executable\\
\hline
bpf\_probe\_write\_user() & Attempt to write data at a user memory address\\
\hline
bpf\_override\_return() & Override return value of a probed function\\
\hline
bpf\_ringbuf\_submit() & Submit data to an specific eBPF ring buffer, and notify to subscribers\\
\hline
bpf\_tail\_call() & Jump to another eBPF program preserving the current stack\\
\hline
\end{tabular}
\caption{Common eBPF helpers. Only those relevant to our research are shown. Those helpers exclusive to an specific program type are not listed. The full list and attribute details can be consulted in the man page \cite{ebpf_helpers}.}
\label{table:ebpf_helpers}
\end{table}


\section{eBPF program types} \label{section:ebpf_prog_types}
In the previous subsection \ref{subsection:bpf_syscall} we introduced the new types of eBPF programs that are supported and that we will be developing for our offensive analysis. In this section, we will analyse in greater detail how eBPF is integrated in the Linux kernel in order to support these new functionalities.

\subsection{XDP} \label{subsection:xdp}
EXpress Data Path (XDP) programs are a novel type of eBPF program that allows for the lowest-latency traffic filtering and monitoring in the whole Linux kernel. In order to load an XDP program, a bpf() syscall with the command BPF\_PROG\_LOAD and the program type BPF\_PROG\_TYPE\_XDP must be issued. 

These programs are directly attached to the Network Interface Controller (NIC) driver, and thus they can process the packet before any other module \cite{xdp_gentle_intro}.

Figure \ref{fig:xdp_diag} shows how XDP is integrated in the network processing of the Linux kernel. After receiving a raw packet (in the figure, \textit{xdp\_md}, which consists on the raw bytes plus some very basic metadata about the packet) from the incoming traffic, XDP program can perform the following actions \cite{xdp_manual}:
\begin{itemize}
\item Analyse the data between the packet buffer bounds.
\item Modify the packet contents, and modify the packet length.
\item Decide between one of the actions displayed in table \ref{table:xdp_actions_av}.
\end{itemize}

\begin{figure}[htbp]
	\centering
	\includegraphics[width=15cm]{xdp_diag.jpg}
	% Either this caption, or change the text afterwards. I still need to know whether to put the long explanation here or on the paragraph, it gets repetitive.
	\caption{XDP and TC modules integration in the network processing module of the Linux kernel.}
	\label{fig:xdp_diag}
\end{figure}

\begin{table}[htbp]
\begin{tabular}{|c|>{\centering\arraybackslash}p{10cm}|}
\hline
ACTION & DESCRIPTION\\
\hline
\hline
XDP\_PASS & Let packet proceed with operated modifications on it.\\
\hline
XDP\_TX & Return the packet at the same NIC it was received from. Packet modifications are kept.\\
\hline
XDP\_DROP & Drops the packet completely, kernel networking will not be notified.\\
\hline
\end{tabular}
\caption{Relevant XDP return values.}
\label{table:xdp_actions_av}
\end{table}

Some of the XDP-exclusive eBPF helpers we will be discussing in later sections are shown in table \ref{table:xdp_helpers}.
\begin{table}[htbp]
\begin{tabular}{|c|>{\centering\arraybackslash}p{10cm}|}
\hline
eBPF helper & DESCRIPTION\\
\hline
\hline
bpf\_xdp\_adjust\_head() & Enlarges or reduces the extension of a packet, by moving the address of its first byte.\\
\hline
bpf\_xdp\_adjust\_tail() & Enlarges or reduces the extension of a packet, by moving the address of its last byte.\\
\hline
\end{tabular}
\caption{Relevant XDP-exclusive eBPF helpers.}
\label{table:xdp_helpers}
\end{table}


\subsection{Traffic Control} \label{subsection:tc}
Traffic Control (TC) programs are also indicated for networking instrumentation. Similarly to XDP, their module is positioned before entering the overall network processing of the kernel. However, as it can be observed in figure \ref{fig:xdp_diag}, they differ in some aspects:
\begin{itemize}
\item TC programs receive a network buffer with metadata (in the figure, \textit{sk\_buff}) about the packet in it. This renders TC programs less ideal than XDP for performing large packet modifications (like new headers), but at the same time the additional metadata fields make it easier to locate and modify specific packet fields \cite{tc_differences}.
\item TC programs can be attached to the \textit{ingress} or \textit{egress} points, meaning that an eBPF program can operate not only over incoming traffic, but also over the outgoing packets.
\end{itemize}

With respect to how TC programs operate, the Traffic Control system in Linux is greatly complex and would require a complete section by itself. In fact, it was already a complete system before the appearance of eBPF. Full documentation can be found at \cite{tc_docs_complete}. For this document, we will explain the overall process needed to load a TC program \cite{tc_direct_action}:
\begin{enumerate}
\item The TC program defines a so-called queuing discipline (qdisc), a packet scheduler that issues packets in a First-In-First-Out (FIFO) order as soon as they are received. This qdisc will be attached to a specific network interface (e.g.: wlan0).
\item Our TC eBPF program is attached to the qdisc. It will work as a filter, being run for every of the packets dispatched by the qdisc.
\end{enumerate}

Similarly to XDP, the TC eBPF programs can decide an action to be executed on a packet by specifying a return value. These actions are almost analogous to the ones in XDP, as it can be observed in table \ref{table:tc_actions}.

\begin{table}[htbp]
\begin{tabular}{|c|>{\centering\arraybackslash}p{10cm}|}
\hline
ACTION & DESCRIPTION\\
\hline
\hline
TC\_ACT\_OK & Let packet proceed with operated modifications on it.\\
\hline
TC\_ACT\_RECLASSIFY & Return the packet to the back of the qdisc scheduling queue.\\
\hline
TC\_ACT\_SHOT & Drops the packet completely, kernel networking will not be notified.\\
\hline
\end{tabular}
\caption{Relevant TC return values. Full list can be consulted at \cite{tc_ret_list_complete}.}
\label{table:tc_actions}
\end{table}

Finally, as in XDP, there exists a list of useful BPF helpers that will be relevant for the creation of our rootkit. They are shown in table \ref{table:tc_helpers}.
\begin{table}[htbp]
\begin{tabular}{|c|>{\centering\arraybackslash}p{10cm}|}
\hline
eBPF helper & DESCRIPTION\\
\hline
\hline
bpf\_l3\_csum\_replace() & Recomputes the network layer 3 (e.g.: IP) checksum of the packet.\\
\hline
bpf\_l4\_csum\_replace() & Recomputes the network layer 4 (e.g.: TCP) checksum of the packet.\\
\hline
bpf\_skb\_store\_bytes() & Write a data buffer into the packet.\\
\hline
bpf\_skb\_pull\_data() & Reads a sequence of packet bytes into a buffer.\\
\hline
bpf\_skb\_change\_head() & (Only) enlarges the extension of a packet, by moving the address of its first byte.\\
\hline
bpf\_skb\_change\_tail() & Enlarges or reduces the extension of a packet, by moving the address of its last byte.\\
\hline
\hline
\end{tabular}
\caption{Relevant TC-exclusive eBPF helpers.}
\label{table:tc_helpers}
\end{table}


%TODO This section might benefit from some diagrams, maybe. It was a bit to extense already, so skipping it from now
\subsection{Tracepoints} \label{subsection:tracepoints}
Tracepoints are a technology in the Linux kernel that allows to hook functions in the kernel, connecting a 'probe': a function that is executed every time the hooked function is called \cite{tp_kernel}. These tracepoints are set statically during kernel development, meaning that for a function to be hooked, it needs to have been previously marked with a tracepoint statement indicating its traceability. At the same time, this limits the number of tracepoints available.

The list of tracepoint events available depends on the kernel version and can be visited under the directory \textit{/sys/kernel/debug/tracing/events}.

It is particularly relevant for our later research that most of the system calls incorporate a tracepoint, both when they are called (\textit{enter} tracepoint) and when they are exited (\textit{exit} tracepoints). This means that, for a system call sys\_open, both the tracepoint sys\_enter\_open and sys\_exit\_open are available. 

Also, note that the probe functions that are called when hitting a tracepoint receive some parameters related to the context on which the tracepoint is located. In the case of syscalls, these include the parameters with which the syscall was called (only for \textit{enter} syscalls, \textit{exit} ones will only have access to the return value). The exact parameters and their format which a probe function receives can be visited in the file \textit{/sys/kernel/debug/tracing/events/<subsystem>/<tracepoint>/format}. In the previous example with sys\_enter\_open, this is \textit{/sys/kernel/debug/tracing/events/syscalls/sys\_enter\_open/format}.

In eBPF, a program can issue a bpf() syscall with the command BPF\_PROG\_LOAD and the program type BPF\_PROG\_TYPE\_TRACEPOINT, specifying which is the function with the tracepoint to attach to and an arbitrary function probe to call when it is hit. This function probe is defined by the user in the eBPF program submitted to the kernel.

\subsection{Kprobes}
Kprobes are another tracing technology of the Linux kernel whose functionality has been become available to eBPF programs. Similarly to tracepoints, kprobes enable to hook functions in the kernel, with the only difference that it is dynamically attached to any arbitrary function, rather than to a set of predefined positions \cite{kprobe_manual}. It does not require that kernel developers specifically mark a function to be probed, but rather kprobes can be attached to any instruction, with a short list of blacklisted exceptions. 

As it happened with tracepoints, the probe functions have access to the parameters of the original hooked function. Also, the kernel maintains a list of kernel symbols (addresses) which are relevant for tracing and that offer us insight into which functions we can probe. It can be visited under the file \textit{/proc/kallsyms}, which exports symbols of kernel functions and loaded kernel modules \cite{kallsyms_kernel}.

Also similarly, since tracepoints could be found in their \textit{enter} and \textit{exit} variations, kprobes have their counterpart, name kretprobes, which call the hooked probe once a return instruction is reached after the hooked symbol. This means that a kretprobe hooked to a kernel function will call the probe function once it exits.

In eBPF, a program can issue a bpf() syscall with the command BPF\_PROG\_LOAD and the program type BPF\_PROG\_TYPE\_KPROBE, specifying which is the function with the kprobe to attach to and an arbitrary function probe to call when it is hit. This function probe is defined by the user in the eBPF program submitted to the kernel.

\subsection{Uprobes}
Uprobes is the last of the main tracing technologies which has been become accessible to eBPF programs. They are the counterparts of Kprobes, allowing for tracing the execution of an specific instruction in the user space, instead of in the kernel. When the execution flow reaches a hooked instruction, a probe function is run. 

For setting an uprobe on a specific instruction of a program, we need to know three components:
\begin{itemize}
\item The name of the program.
\item The address of the function where the instruction is contained.
\item The offset at which the specific instruction is placed from the start of the function.
\end{itemize}

Similarly to kprobes, uprobes have access to the parameters received by the hooked function. Also, the complementary uretprobes exist too, running the probe function once the hooked function returns.

In eBPF, programs can issue a bpf() syscall with the command BPF\_PROG\_LOAD and the program type BPF\_PROG\_TYPE\_UPROBE, specifying the function with the uprobe to attach to and an arbitrary function probe to call when it is hit. This function probe is also defined by the user in the eBPF program submitted to the kernel.

% Is this the best title?
\section{Developing eBPF programs}
In section \ref{section:modern_ebpf}, we discussed the overall architecture of the eBPF system which is now an integral part of the Linux kernel. We also studied the process which a piece of eBPF bytecode follows in order to be accepted in the kernel. However, for an eBPF developer, programming bytecode and working with bpf() calls natively is not an easy task, therefore an additional layer of abstraction was needed. 

Nowadays, there exist multiple popular alternatives for writing and running eBPF programs. We will overview which they are and proceed to analyse in further detail the option that we will use for the development of our rootkit.

\subsection{BCC}
BPF Compiler Collection (BCC) is one of the first and well-known toolkits for eBPF programming available \cite{bcc_github}. It allows to include eBPF code into user programs. These programs are developed in python, and the eBPF code is embedded as a plain string. An example of a BCC program is included in %TODO ANNEX???

Although BCC offers a wide range of tools to easy the development of eBPF programs, we found it not to be the most appropriate for our large-scale eBPF project. In particular, this was due to the feature of eBPF programs being stored as a python string, which leads to difficult scalability, poor development experience given that programming errors are detected at runtime (once the python program issues the compilation of the string), and simply better features from competing libraries.

\subsection{Bpftool}
Bpftool is not a development framework like BCC, but one of the most relevant tools for eBPF program development. Some of its functionalities include:
\begin{itemize}
\item Loading eBPF programs.
\item List running eBPF programs.
\item Dumping bytecode from live eBPF programs.
\item Extract program statistics and data from programs.
\item List and operate over eBPF maps.
\end{itemize}

Although we will not be covering bpftool during our overview on the constructed eBPF rootkit, it was used extensively during the development and became a key tool for debugging eBPF programs, particularly to peek data at eBPF maps during runtime.

\subsection{Libbpf}
Libbpf \cite{libbpf_github} is a library for loading and interacting with eBPF programs, which is currently maintained in the Linux kernel source tree \cite{libbpf_upstream}. It is one of the most popular frameworks to develop eBPF applications, both because it makes eBPF programming similar to common kernel development and because it aims at reducing kernel-version dependencies, thus increasing programs portability between systems \cite{libbpf_core}. During our research, however, we will not make use of this functionalities given that a portable program is not in our research goals.

As we discussed in section \ref{section:modern_ebpf}, eBPF programs are composed of both the eBPF code in the kernel and a user space program that can interact with it. With libbpf, the eBPF kernel program is developed in C (a real program, not a string later compiled as with BCC), while user programs are usually developed in C, Rust or GO. For our project, we will use the C version of libbpf, so both the user and kernel side of our rootkit will be developed in this language.

% Cites in the following paragraph?
When using libbpf with the C language, both the user-side and kernel eBPF program are compiled together using the Clang/LLVM compiler, translating C instructions into eBPF bytecode. As a clarification, Clang is the front-end of the compiler, translating C instructions into an intermediate form understandable by LLVM, whilst LLVM is the back end compiling the intermediate code into eBPF bytecode. As it can be observed in figure \ref{fig:libbpf}, the result of the compilation is a single program, comprising the user-side which will launch a user process, the eBPF bytecode to be run in the kernel, and other structures libbpf generates about eBPF maps and other meta data. This program is encapsulated as an ELF file (a common executable format).

\begin{figure}[htbp]
	\centering
	\includegraphics[width=12cm, keepaspectratio=true]{libbpf_prog.jpg}
	\caption{Compilation and loading process of a program developed with libbpf.}
	\label{fig:libbpf}
\end{figure}

Finally, we will overview one of the main functionalities of libbpf to simplify eBPF programming, namely the BPF skeleton. This is auto-generated code by libbpf whose aim is to simplify working with eBPF from the user-side program. As a summary, it parses the eBPF programs developed (which may be using different technologies such as XDP, kprobes, TC...) and the eBPF maps used, and as a result offers a simple set of functions for dealing with these programs from the user program. In particular, it allows for loading and unloading a specific eBPF program from user space at runtime.

Table \ref{table:libbpf_skel} describes the API offered by the BPF skeleton. Note that <name> is substituted by the name of the program being compiled.

\begin{table}[htbp]
\begin{tabular}{|c|>{\centering\arraybackslash}p{10cm}|}
\hline
Function name & Description\\
\hline
\hline
<name>\_\_open() & Parse the eBPF programs and maps.\\
\hline
<name>\_\_load() & Load the eBPF map in the kernel after its validation, create the maps. However, the programs are not active yet.\\
\hline
<name>\_\_attach() & Activate the eBPF programs, attaching them to their corresponding parts in the kernel (e.g. kprobes to kernel functions).\\
\hline
<name>\_\_destroy() & Detach and unload the eBPF programs from the kernel.\\
\hline
\end{tabular}
\caption{BPF skeleton functions.}
\label{table:libbpf_skel}
\end{table}

Note that the BPF skeleton also offers further granularity at the time of dealing with programs, so that individual programs can be loaded or attached instead of all simultaneously. This is the approach we will generally use in the development of our rootkit, as it will be explained in section \ref{subsection:ebpf_progs_config}.



\section{Security features in eBPF}
As we have shown in section \ref{section:modern_ebpf}, eBPF has been an active part of the Linux kernel from its 3.18 version. However, as with many other components of the kernel, its availability to the user depends on the parameters with which the kernel has been compiled. Specifically, eBPF is only available to kernels compiled with the flags specified in table \ref{table:ebpf_kernel_flags}.

\begin{table}[htbp]
\begin{tabular}{|c|c|>{\centering\arraybackslash}p{8cm}|}
\hline
Flag & Value & Description\\
\hline
\hline
\multicolumn{1}{|c|}{CONFIG\_BPF} & \multicolumn{1}{|c|}{y} & \multirow{2}{*}{Basic BPF compilation (mandatory)}\\
\cline{1-2}
\multicolumn{1}{|c|}{CONFIG\_BPF\_SYSCALL} & \multicolumn{1}{|c|}{m} & \\
\hline
\multicolumn{1}{|c|}{CONFIG\_NET\_ACT\_BPF} & \multicolumn{1}{|c|}{m} & \multirow{2}{*}{Traffic Control functionality}\\
\cline{1-2}
\multicolumn{1}{|c|}{CONFIG\_NET\_CLS\_BPF} & \multicolumn{1}{|c|}{y} & \\
\hline
\multicolumn{1}{|c|}{CONFIG\_BPF\_JIT} & \multicolumn{1}{|c|}{y} & \multirow{2}{*}{Enable JIT compliation}\\
\cline{1-2}
\multicolumn{1}{|c|}{CONFIG\_HAVE\_BPF\_JIT} & \multicolumn{1}{|c|}{y} & \\
\hline
\multicolumn{1}{|c|}{CONFIG\_BPF\_EVENTS} & \multicolumn{1}{|c|}{y} & \multirow{4}{*}{Enable kprobes, uprobes and tracepoints}\\
\cline{1-2}
\multicolumn{1}{|c|}{CONFIG\_KPROBE\_EVENTS} & \multicolumn{1}{|c|}{y} & \\
\cline{1-2}
\multicolumn{1}{|c|}{CONFIG\_UPROBE\_EVENTS} & \multicolumn{1}{|c|}{y} & \\
\cline{1-2}
\multicolumn{1}{|c|}{CONFIG\_TRACING} & \multicolumn{1}{|c|}{y} & \\
\hline
CONFIG\_XDP\_SOCKETS & y & Enable XDP\\
\hline
\end{tabular}
\caption{Kernel compilation flags for eBPF.}
\label{table:ebpf_kernel_flags}
\end{table}

Table \ref{table:ebpf_kernel_flags} is based on BCC's documentation, but the full list of eBPF-related flags can be extracted in a live system via bpftool, as detailed in Annex \ref{annex:bpftool_flags_kernel}. Nowadays, all mainstream Linux distributions include kernels with full support for eBPF.


\subsection{Access control} \label{subsection:access_control}
It must be noted that, similarly to kernel modules, loading an eBPF program requires privileged access in the system. In old kernel versions, this means either a user having full root permissions, or having the Linux capability \cite{ubuntu_caps} CAP\_SYS\_ADMIN. Therefore, there existed two main options:
%TODO some words about capabilities
\begin{itemize}
\item \textbf{Privileged users} can load any kind of eBPF program and use any functionality.
\item \textbf{Unprivileged users} can only load and attach eBPF programs of type BPF\_PROG\_TYPE\_SOCKET\_FILTER \cite{evil_ebpf_p9}, offering the very limited functionality of filtering packets received on a socket.
\end{itemize}

More recently, in an effort to further granulate the permissions needed for loading, attaching and running eBPF programs, CAP\_SYS\_ADMIN has been substituted by more specific capabilities \cite{ebpf_caps_intro} \cite{ebpf_caps_lwn}. The current system is therefore described in table \ref{table:ebpf_caps_current}.

\begin{table}[htbp]
\begin{tabular}{|>{\centering\arraybackslash}p{4cm}|>{\centering\arraybackslash}p{10cm}|}
\hline
Capabilities & eBPF functionality\\
\hline
\hline
No capabilities & Load and attach BPF\_PROG\_TYPE\_SOCKET\_FILTER, load BPF\_PROG\_TYPE\_CGROUP\_SKB programs.\\
\hline
CAP\_BPF & Load (but not attach) any type of program, create most types of eBPF map and access them if their id is known\\
\hline
CAP\_NET\_ADMIN & Attach networking programs (Traffic Control, XDP, ...)\\
\hline
CAP\_PERFMON & Attaching kprobes, uprobes and tracepoints. Read access to kernel memory.\\
\hline
CAP\_SYS\_ADMIN & Privileged eBPF. Includes iterating over eBPF maps, and CAP\_BPF, CAP\_NET\_ADMIN, CAP\_PERFMON functionalities.\\
\hline
\end{tabular}
\caption{Capabilities needed for eBPF.}
\label{table:ebpf_caps_current}
\end{table}

Therefore, eBPF network programs usually require both CAP\_BPF and CAP\_NET\_ADMIN, whilst tracing programs require CAP\_BPF and CAP\_PERFMON. CAP\_SYS\_ADMIN remains as the (non-preferred) capability to assign to eBPF programs with complete access in the system.

Although for a long time there have existed efforts towards enhancing unprivileged eBPF, it remains a worrying feature \cite{unprivileged_ebpf}. The main issue is that the verifier must be prepared to detect any attempt to extract kernel memory access or user memory modification by unprivileged eBPF programs, which is a complex task. In fact, there have existed numerous security vulnerabilities which allow for privilege escalation using eBPF, that is, execution of privileged eBPF programs by exploiting vulnerabilities in unprivileged eBPF \cite{cve_unpriv_ebpf}.

This influx of security vulnerabilities leads to the recent inclusion of an attribute into the kernel which allows for setting whether unprivileged eBPF is allowed in the system or not. This parameter is named \textit{kernel.unprivileged\_bpf\_disabled}, its values can be seen in table \ref{table:unpriv_ebpf_values}.

\begin{table}[htbp]
\begin{tabular}{|>{\centering\arraybackslash}p{4cm}|>{\centering\arraybackslash}p{10cm}|}
\hline
Value & Meaning\\
\hline
\hline
0 & Unprivileged eBPF is enabled.\\
\hline
1 & Unprivileged eBPF is disabled. A system reboot is needed to enable it after changing this value.\\
\hline
2 & Unprivileged eBPF is disabled. A system reboot is not needed to enable it after changing this value.\\
\hline
\end{tabular}
\caption{Values for unprivileged eBPF kernel parameter.}
\label{table:unpriv_ebpf_values}
\end{table}

Nowadays, most Linux distributions have set value 1 to this parameter, therefore disallowing unprivileged eBPF completely. These include Ubuntu \cite{unpriv_ebpf_ubuntu}, Suse Linux \cite{unpriv_ebpf_suse} or Red Hat Linux \cite{unpriv_ebpf_redhat}, between others.



\section{Memory management in Linux}
Multiple of the techniques incorporated in our rootkit require a deep understanding into how memory is managed in a Linux process. Therefore, in this section we will present all the background about memory management needed for our later discussion of the offensive capabilities of eBPF in this context.

\subsection{Memory pages and faults} \label{subsection:mem_faults}
Linux systems divide the available random-access memory (RAM) into 'pages', subsections of an specific length, usually 4 KB. The collection of all pages is called physical memory.

Likewise, individual memory sections need to be assigned to each running process in the system, but instead of assigning a set of pages from physical memory, a new address space is defined, named virtual memory, which is divided into pages as well. These virtual memory pages are related to physical memory pages via a page table, so that each virtual memory address of a process can be translated into a real, physical memory address in RAM \cite{mem_page_arch}. Figure \ref{fig:mem_arch_pages} shows a diagram of the described architecture.

\begin{figure}[htbp]
	\centering
	\includegraphics[width=13cm]{mem_arch_pages.jpg}
	\caption{Memory translation of virtual pages to physical pages.}
	\label{fig:mem_arch_pages}
\end{figure}

As we can observe in the figure, each virtual page is related to one physical page. However, RAM needs to maintain multiple processes and data simultaneously, and therefore sometimes the operating system (OS) will remove them from physical memory when it believes they are no longer being used. This leads to the occurrence of two type of memory events \cite{page_faults}:
\begin{itemize}
\item \textbf{Major page faults} occur when a process tries to access a virtual page, but the related physical page has been removed from RAM. In this case, the OS will need to request a secondary storage (such as a hard disk) for the data removed and allocate a new physical page for the virtual page. Figure \ref{fig:mem_major_page_fault} illustrates a major page fault.
\begin{figure}[htbp]
	\centering
	\includegraphics[width=11cm]{mem_major_page_fault.jpg}
	\caption{Major page fault after a page was removed from RAM.}
	\label{fig:mem_major_page_fault}
\end{figure}
\item \textbf{Minor page faults} occur when a process tries to access a virtual page, and although the related physical page exists, the connection in the page table has not been completed. A common event when these fault happen is on fork() calls, since with the purpose of making the call more efficient, the page table of the parent is not always completely copied into the child, leading into multiple minor page faults once the child tries to access the data on them. Figure \ref{fig:mem_minor_page_fault} illustrates a minor page fault after a fork.
\end{itemize}
\begin{figure}[htbp]
	\centering
	\includegraphics[width=11cm]{mem_minor_page_fault.jpg}
	\caption{Minor page fault after a fork() in which the page table was not copied completely.}
	\label{fig:mem_minor_page_fault}
\end{figure}

\subsection{Process virtual memory}
In the previous subsection we have studied that each process disposes of a virtual address space. We will now describe how this virtual memory is organized in a Linux system.

Figure \ref{fig:mem_proc_arch} describes how virtual memory is distributed within a process in the x86\_64 architecture. As we can observe, it is partitioned into multiple sections:
\begin{figure}[htbp]
	\centering
	\includegraphics[width=6cm]{memory.jpg}
	\caption{Virtual memory architecture of a process \cite{mem_arch_proc}.}
	\label{fig:mem_proc_arch}
\end{figure}
\begin{itemize}
\item Lower and upper memory addresses are reserved for the kernel.
\item A section where shared libraries code is stored.
\item A .text section, which contains the code of the program being run.
\item A .data section, containing initialized static and global variables.
\item A .bss section, which contains global and static variables which are uninitialized or initialized to zero.
\item The heap, a section which grows from lower to higher memory addresses, and which contains memory dynamically allocated by the program.
\item The stack, a section which grows from higher to lower memory addresses, towards the heap. It is a Last In First Out (LIFO) structure used to store local variables, function parameters and return addresses.
\item Right at the start of the stack we can find the arguments with which the programs has been executed.
\end{itemize}

\subsection{The process stack} \label{subsection:stack}
Between all the sections we identified in a process virtual memory, the stack will be particularly relevant during our research. We will therefore study it now in detail. 

Firstly, we will present how the stack is structured, and which operations can be executed on it. Figure \ref{fig:stack_pres} presents a stack during the execution of a program. Table \ref{table:systemv_abi_other} explains the purpose of the most relevant registers related to the stack and program execution:

\begin{figure}[htbp]
	\centering
	\includegraphics[width=14cm]{stack_pres.jpg}
	\caption{Simplified stack representation showing only stack frames.}
	\label{fig:stack_pres}
\end{figure}

\begin{table}[htbp]
\begin{tabular}{|>{\centering\arraybackslash}p{2cm}|>{\centering\arraybackslash}p{10cm}|}
\hline
Register & Purpose\\
\hline
\hline
rip & Instruction Pointer - Memory address of the next instruction to execute\\
\hline
rsp & Stack Pointer - Memory address where next stack operation takes place\\
\hline
rbp & Base/Frame Pointer - Memory address of the start of the stack frame\\
\hline
\end{tabular}
\caption{Relevant registers in x86\_64 for the stack and control flow and their purpose.}
\label{table:systemv_abi_other}
\end{table}

As it can be observed in figure \ref{fig:stack_pres}, the stack grows towards lower memory addresses, and it is organized in stack frames, delimited by the registers rsp and rbp. An stack frame is a division of the stack which contains all the data (variables, call arguments...) belonging to a single function execution. When a function is exited, its stack frame is removed, and if a function calls a nested function, then its stack frame is preserved and a new stack frame is inserted into the stack. 

As table \ref{table:systemv_abi_other} explains, the rbp and rsp registers are used for keeping track of the starting and final position of the current stack frame respectively. We can see in figure \ref{fig:stack_pres} that their value is a memory address pointing to their stack positions. On the other hand, the rip register does not point to the stack, but rather to the .text section (see figure \ref{fig:mem_proc_arch}), where it points to the next instruction to be executed. However, as we will now see, its value must also be stored in the stack frame when a nested function is called, since after the nested function exits we need to restore the execution in the same instruction of the original function.

As with any LIFO structure, the stack supports two main operations: \textit{push} and \textit{pop}. In the x86\_64 architecture, it operates with chunks of data of either 16, 32 or 64 bytes. Table \ref{fig:stack_ops} shows a representation of these operations in the stack.
\begin{itemize}
\item A \textbf{push} operation writes data in the free memory pointed by register rsp. It then moves the value of rsp to point to the new end of the stack.
\item A \textbf{pop} operation moves the value of rsp by 16, 32 or 64 bytes, and reads the data previously saved in that position.
\end{itemize}

\begin{figure}[htbp]
	\centering
	\includegraphics[width=10cm]{stack_ops.jpg}
	\caption{Representation of push and pop operations in the stack.}
	\label{fig:stack_ops}
\end{figure}


As we mentioned, the stack stores function parameters, return addresses and local variables inside a stack frame. We will now study how the processor uses the stack in order to call, execute, and exit a function. To illustrate this process, we will simulate the execution of function \lstinline{func(char* a, char* b, char* c)} \lstinline{}. Figures \ref{fig:stack_before} and \ref{fig:stack} show a representation of the stack during these operations.

\begin{figure}[htbp]
	\centering
	\includegraphics[width=14cm]{stack_before.jpg}
	\caption{Stack representation right before starting the function call process.}
	\label{fig:stack_before}
\end{figure}

\begin{figure}[htbp]
	\centering
	\includegraphics[width=14cm]{stack.jpg}
	\caption{Stack representation right after the function preamble.}
	\label{fig:stack}
\end{figure}

\begin{enumerate}
\item The function arguments are pushed into the stack. We can see them in the stack of figure \ref{fig:stack} in reverse order. 
\item The function is called:
\begin{enumerate}
	\item The value of register rip is pushed into the stack, so that it is saved for when the function exists. We can see it on figure \ref{fig:stack} as 'ret'.
	\item The value of rip changes to point to the first instruction of the called function.
\item We execute what is called as the \textit{function preamble} \cite{8664_params_abi_p18}, which prepares the stack frame for the called function:
\begin{enumerate}
	\item The value of rbp is pushed into the stack, so that we can restore the previous stack frame when the function exits. We can see it on figure \ref{fig:stack} as the 'saved frame pointer'.
	\item The value of rsp is moved into rbp. Therefore, now rbp points to the end of the previous stack frame.
	\item The value of rsp is usually decremented (since the stack needs to go to lower memory addresses) so that we allocate some space for function variables.
\end{enumerate}
\end{enumerate}
\item The function instructions are executed. The stack may be further modified, but on its end rsp must point to the same address of the beginning. Register rbp always keeps pointing to the end of the stack.
\item We execute what is called the \textit{function epilogue}, which removes the stack frame and restores the original function:
\begin{enumerate}
	\item The value of rbp is moved into rsp, so that rsp points to the start of the previous stack frame. All data allocated in the previous stack frame is considered to be free.
	\item The value of the saved frame pointer is popped and stored into rbp, so that rbp now points to the start of the previous stack frame.
	\item The value of the saved rip value is popped into register rip, so that the next instruction to execute is the instruction right after the function call.
\end{enumerate}
\item Since the function arguments where pushed into the stack, they are popped now.
\end{enumerate}


\section{Attacks at the stack} \label{section:attacks_stack}
In section \ref{subsection:stack}, we studied how the stack works and which is the process that a program follows in order to call a function. As we saw in figure \ref{fig:stack}, the processor pushes into the stack several data which is used to restore the context of the original function once the called function exits. These pushed arguments included:
\begin{itemize}
\item The arguments with which the function is being called (if they need to be passed in the stack, such as byte arrays).
\item The original value of the rip register (ret), to restore the execution on the original function.
\item The original value of the rbp register (sfp), to restore the frame pointer of the original stack frame.
\end{itemize}

Although this process is simple enough, it opens the possibility for an attacker to easily hijack the flow of execution if it can modify the value of ret, as it is shown in figure \ref{fig:stack_ret_hij_simple}.
\begin{figure}[htbp]
	\centering
	\includegraphics[width=15cm]{stack_ret_hij_simple.jpg}
	\caption{Execution hijack overwriting saved rip value.}
	\label{fig:stack_ret_hij_simple}
\end{figure}

In the figure, we can observe how, during the execution of the called function, the attacker overwrites the value of ret in the stack. Once the function exists, as we explained in section \ref{subsection:stack}, during the function epilogue the value of ret will be popped and moved into rip, so that the execution is directed to the original next instruction. However, because the value was modified, the attacker controls which instructions are executed next.

Attackers have historically used multiple techniques to overwrite the ret value in the stack. In this section, we will present two of the most popular techniques, which will be used as a basis for designing our own attacks using eBPF.

\subsection{Buffer overflow} \label{subsection: buf_overflow}
The stack buffer overflow is one of the most popular exploitation techniques to overwrite data at the stack. In this technique, an attacker takes advantage of a program receiving a user value stored in a buffer whose capacity is smaller of that of the supplied value. Code snippet \ref{code:vuln_overflow} shows an example of a vulnerable program:

\begin{lstlisting}[language=C, caption={Program vulnerable to buffer overflow.}, label={code:vuln_overflow}]
void foo(char *bar){ // bar may be larger than 12 characters
   char buffer[12];
   strcpy(buffer, bar); //no bounds checking 
}

int main(int argc, char *argv[]){
   foo(argv[1]);
   return 0;
}
\end{lstlisting}

During the execution of the above program, since the char array \textit{buffer} is a buffer of length 12 stored in the stack, then if the value of \textit{bar} is larger than 12 bytes it will overflow the allocated space in the stack. This is usually the case of using unsafe functions for processing user input such as strcpy(), which does not check whether the array fits in the buffer. Figure \ref{fig:buffer_overflow} shows how the overflow happens in the stack.

\begin{figure}[htbp]
	\centering
	\includegraphics[width=15cm]{buffer_overflow.jpg}
	\caption{Stack buffer overflow overwriting ret value.}
	\label{fig:buffer_overflow}
\end{figure}

As we can observe in the figure, the new data written into the buffer has also overwritten other fields which were pushed into the stack, such as sfp and ret, resulting in changing the flow of execution once the function exists.

Usually, an attacker exploiting a program vulnerable to stack buffer overflow is interested in running arbitrary (malicious) code. For this, the attacker follows the process shown in figure \ref{fig:buffer_overflow_shellcode}:

\begin{figure}[htbp]
	\centering
	\includegraphics[width=15cm]{buffer_overflow_shellcode.jpg}
	\caption{Executing arbitrary code exploiting a buffer overflow vulnerability.}
	\label{fig:buffer_overflow_shellcode}
\end{figure}

As we can observe in the figure, the attacker will take advantage of the buffer overflow to overwrite not only ret, but also the rest of the current stack frame and sfp with malicious code. This code is known as shellcode, consisting of instruction opcodes (machine assembly instructions translated to their representation in hexadecimal values) which the processor will execute. We will briefly explain how to write shellcode in section \ref{TODO probably an Annex}. Therefore, in this technique the attacker will:
\begin{itemize}
\item Introduce a byte array that overflows the buffer, consisting on SHELLCODE + the address of the buffer.
\begin{itemize}
	\item The shellcode overwrites the buffer and all data until ret.
	\item ret is overwritten by the value of the address where the buffer starts.
\end{itemize}
\item When the function exits and ret is popped from the stack, the register rip will now point to the address of the buffer at the stack, processing the stack data as instructions part of a program. The malicious code will be executed.
\end{itemize}

Although the classic buffer overflow is one of the best-known techniques in binary exploitation, it is also one of the oldest and thus numerous protections have historically been incorporated to mitigate this type of exploits. This is why the attack presented here does not work work in a modern system anymore. 

The reason is that one of the protections consists of the prohibition of executing code from the stack. By marking the stack as non-executable, in the case of rip pointing to an address in the stack any malicious code will not be run, even if an application was vulnerable to a buffer overflow. We will explain more in detail the main protections that nowadays are incorporated in modern systems in section \ref{subsection:hardening_elf}.

\subsection{Return oriented programming attacks} \label{subsection:rop}
After the stack was marked non-executable, a new refined technique was invented to circumvent this restriction and adapt the classic buffer overflow to modern systems. In the end, attackers still maintained the ability to overflow the buffer in the stack of vulnerable applications, writing shellcode and overwriting ret, the only issue was that the shellcode could not be executed.

Return Oriented Programming (ROP) is an exploitation technique that takes advantage of the fact that, even if malicious code in the stack cannot be executed, the attacker can still redirect the flow of execution by modifying ret to any other piece of executable code. The challenge for the attacker is executing malicious code, since any available executable instructions are either at the .text section (which will correspond to the normal functioning of the program) or at shared libraries, but none are useful for malware. 

ROP tackles this challenge by designing a method of reconstructing malicious code from parts of already-existing code, as in a 'collage'. Assembly instructions are selected from multiple places, so that, when put together and executed sequentially, they recreate the shellcode which the attacker wants to execute. These pieces of code are called ROP gadgets, and consist of a set of arbitrary instructions followed by a final \textit{ret} instruction, which triggers the function exit and pops the value of ret. These gadgets may belong to any code in the process memory, usually selected between the code of the shared libraries (see figure \ref{fig:stack}) to which the process is linked.

Finding ROP gadgets and writing ROP-compatible payloads manually is hard, thus multiple programs exist that automatically scan the system libraries and construct provide the gadgets given the shellcode to execute \cite{rop_prog_finder}.

However, we will now illustrate how ROP works with an example. Suppose that an attacker has discovered a buffer overflow vulnerability, but the stack is marked as not executable. The attacker wants to execute the assembly code shown in snippet \ref{code:rop_ex}:

\begin{lstlisting}[language=C, caption={Sample program to run using ROP.}, label={code:rop_ex}]
mov rdx, 10
mov rax, [rsp]
\end{lstlisting}

After finding the address of the ROP gadgets manually or using an automated tool, the attacker takes advantage of a buffer overflow (or, in our case, a direct write using eBPF's bpf\_probe\_write\_user()) to overwrite the value of ret with the address of the first ROP gadget, and also additional data in the stack. Figure \ref{fig:rop_compund} shows how we can execute the original program using ROP:

\begin{figure}[htbp]
	\centering
	\includegraphics[width=16cm]{ROPcompound.jpg}
	\caption{Steps for executing code sample using ROP.}
	\label{fig:rop_compund}
\end{figure}

The steps described in the figure are the following:
\begin{enumerate}
\item First step shows the two gadgets located and their addresses, and the overwritten data in the stack. The function has already exited and, because ret was overwritten with the address of the first gadget, register rip now points to that location, and thus it is the next instruction to execute. Register rsp, in turn, now points to the bottom address of the current stack frame, which is right next to the old ret (see section \ref{subsection:stack} for stack frames functioning).
\item The first instruction of the gadget is executed, popping the value from the stack (which also moves register rsp, see stack push and pop operations in section \ref{subsection:stack}). As we can observe, the value "10" was specifically put in that position by the attacker, so that, according to the instruction to execute \lstinline{mov rdx, 10} \lstinline{}, we now have loaded that data into register rdx.
\item The return instruction is executed, which pops from the stack what is supposed to be the value of the saved rip, but in turn the attacker has placed the address of the next gadget there. Now, rip has jumped to the address of the second gadget. By continuing with this process, we can chain an infinite number of gadgets.
\item Finally, we repeated the same process as before, using a pop instruction to load a value from the stack. This illustrates that push and pop instructions, commonly used on most programs, are also possible to be using ROP.

After this step, the return instruction will be executed. Note that, at this point, if the attacker wants to be stealthy and avoid crashing the program (since we overwrote the original data in the stack), the original stack must be restored, together with the value of the registers before the malicious code execution. We will see an example of a technique for reconstructing the original state during our explanation of the library injection in section \ref{TODO}.
\end{enumerate}



\section{Networking fundamentals in Linux} \label{section:networking_fundamentals}
This section presents an overview on the most relevant aspects of the network system in Linux, which will be needed to tackle multiple of the techniques discussed during the design of the network capabilities of our rootkit. In particular, we will be focusing on the Ethernet, IP and TCP protocols.

\subsection{An overview on the network layer}
Firstly, we will describe the data structure we will be dealing with in networking programs. This will be Ethernet frames containing TCP/IP packets. Figure \ref{fig:frame} shows the frame in its completeness:

\begin{figure}[htbp]
	\centering
	\includegraphics[width=14cm]{frame.jpg}
	\caption{Ethernet frame with TCP/IP packet.}
	\label{fig:frame}
\end{figure}

As we can observe, we can distinguish five different network layers in the frame. This division is made according to the OSI model \cite{network_layers}:
\begin{itemize}
\item Layer 1 corresponds to the physical layer, and it is processed by the NIC hardware, even before it reaches the XDP module (see figure \ref{fig:xdp_diag}). Therefore, this layer is discarded and completely invisible to the kernel. Note that it does not only include a header, but also a trailer (a Frame Check Sequence, a redundancy check included to check frame integrity).
\item Layer 2 is the data layer, it is in charge of transporting the frame via physical media, in our case an Ethernet connection. Most relevant fields are the MAC destination and source, used for performing physical addressing.
\item Layer 3 is the network layer, in charge of packet forwarding and routing. In our case, packets will be using the IP protocol. Most relevant fields are the source and destination IP, used to indicate the host that sent the packet and who is the receiver.
\item Layer 4 is the transport layer, in charge of providing end-to-end connection services to applications in a host. We will be focusing on TCP during our research. Relevant fields include the source and destination port, which indicate the ports involved in the communication on which the applications on each host are listening and sending packets.
\item The last layer is the payload of the TCP packet, which contains, according to the OSI model, all layers belong to application data.
\end{itemize}


\subsection{Introduction to the TCP protocol} \label{subsection:tcp}
We will now focus our view on the transport layer, specifically on the TCP protocol, since it will be a major concern at the time of designing the network capabilities of our rootkit.

Firstly, since TCP aims to offer a reliable and ordered packet transmission \cite{tcp_reliable}, it includes sequence numbers (see table \ref{fig:frame}) which mark the order in which they are transmitted. However, since the physical medium may corrupt or lose packets during the transmission, TCP must incorporate mechanisms for ensuring the order and delivery of all packets:
\begin{itemize}
\item Mechanism for opening and establishing a reliable connection between two parties.
\item Mechanism for ensuring that packets are retransmitted in case of an error during the connection.
\end{itemize}

With respect to the establishment of a reliable connection, this is achieved via a 3-way handshake, in which certain TCP flags will be set in a series of interchanged packets (see in figure \ref{fig:frame} the field TCP flags). Most relevant TCP flags are described in table \ref{table:tcp_flags}.

\begin{table}[htbp]
\begin{tabular}{|>{\centering\arraybackslash}p{4cm}|>{\centering\arraybackslash}p{10cm}|}
\hline
Flag & Purpose\\
\hline
\hline
ACK & Acknowledges that a packet has been successfully received. In the acknowledgment number (see figure \ref{fig:frame}), it is stored the sequence number of the packet being acknowledged + 1. \\
\hline
SYN & Used during the 3-way handshake, indicates request for establishing a connection.\\
\hline
FIN & Used to request a connection termination.\\
\hline
RST & Abruptly terminates the connection, usually sent when a host receives an unexpected or unrecognized packet.\\
\hline
\end{tabular}
\caption{Relevant TCP flags and their purpose.}
\label{table:tcp_flags}
\end{table}

Taking the above into account, figure \ref{fig:tcp_conn} shows a depiction of the 3-way handshake \cite{tcp_handshake}:
\begin{figure}[htbp]
	\centering
	\includegraphics[width=12cm]{tcp_conn.jpg}
	\caption{TCP 3-way handshake.}
	\label{fig:tcp_conn}
\end{figure}

As we can observe in the figure, the hosts interchange a sequence of SYN, SYN+ACK, ACK packets, after which the communication starts. During this communication, the sender transmits packets with data (and no flags set), to which it expects an ACK packet acknowledging having received it.

With respect to maintaining the integrity of the connection once it starts, TCP works using timers, as it is illustrated in figure \ref{fig:tcp_retransmission}:
\begin{enumerate}
\item A data packet with sequence number X is sent. The timer starts.
\item The destination host receives the packet and returns an ACK packet with acknowledgment number X+1.
\item The sender receives the ACK packet and stops the timer. If, for any reason, the ACK packet is not received before the timer ends, then the same packet is retransmitted.
\end{enumerate}

\begin{figure}[htbp]
	\centering
	\includegraphics[width=12cm]{tcp_retransmission.jpg}
	\caption{TCP packet retransmission on timeout.}
	\label{fig:tcp_retransmission}
\end{figure}

\section{ELF binaries} \label{section:elf}
This section details the Executable and Linkable Format (ELF) \cite{elf}, the format in which we find executable files (between other file types) in Linux systems. We will perform an analysis from a security standpoint, that is, mainly oriented to describe the most relevant sections and the permissions incorporated into them. We will also focus on several of these sections which will be relevant for designing our attack.

After that, we will overview the security hardening techniques that have been historically incorporated into Linux to mitigate possible exploitation techniques when running ELF executables (such as the stack buffer overflow we explained in section \ref{subsection: buf_overflow}). During the design of our rootkit, we will attempt to bypass these techniques using multiple workarounds.

\subsection{The ELF format and Lazy Binding} \label{subsection:elf_lazy_binding}
Linux supports multiple tools that enable a deep inspection of ELF binaries and its sections. Table \ref{table:elf_tools} shows the main tools we will use during this analysis:

\begin{table}[htbp]
\begin{tabular}{|>{\centering\arraybackslash}p{3cm}|>{\centering\arraybackslash}p{10cm}|}
\hline
Tool & Purposes\\
\hline
\hline
Readelf & Display information about ELF files\\
\hline
Objdump & Display information about object files, mainly used for decompiling programs\\
\hline
GDB & The GNU Project Debugger, allows for debugging programs during runtime\\
\hline
GDB-peda & The Python Exploit Development Assistance for GDB, allows for multiple advanced operations that ease exploit development, such as showing register values, the stack state or memory information. It works as a plugin for GDB.\\
\hline
\end{tabular}
\caption{Tools used for analysis of ELF programs.}
\label{table:elf_tools}
\end{table}

Firstly, we will analyse the main sections we can find in an ELF executable. We will approach this study using a sample program that has been compiled using Clang/LLVM: TODO %TODO How do I explain which progrm it is? It is an example I developed, src/helpers/simple_timer.c. Shoud I write the code somewhere? Seems excesive

The commands used and complete list of headers can be found in Annex \ref{annexsec:readelf_sec_headers}. The most relevant sections are described in table \ref{table:elf_sec_headers}:

\begin{table}[htbp]
\begin{tabular}{|>{\centering\arraybackslash}p{1cm}|>{\centering\arraybackslash}p{9cm}|>{\centering\arraybackslash}p{2cm}|}
\hline
Tool & Purpose & Permissions\\
\hline
\hline
.init & Contains instructions executed before the \textit{main} function of the program & Alloc, Executable\\
\hline
.plt & Procedure Linkage Table (PLT), contains code stubs that use the addresses at .got.plt for jumping to position-independent code & Alloc, Executable\\
\hline
.got & Global Offset Table (GOT), it contains addresses of global variables and functions once the linker resolves them at runtime & Alloc, Writable\\
\hline
.got.plt & A subset of .got section separated from .got with some compilers, it contains only the target addresses of position-independent code once the linker loads them at runtime, used by .plt section. & Alloc, Writable\\
\hline
.plt.got & Generated depending on compiler options, it is a PLT section which does not use lazy binding. & Alloc, Executable\\
\hline
.text & Stores executable instructions. & Alloc, Executable\\
\hline
.data & Contains initialized static and global variables. & Alloc, Writable\\
\hline
.bss & Contains global and static variables which are uninitialized or initialized to zero. & Alloc, Writable\\
\hline 
\end{tabular}
\caption{Tools used for analysis of ELF programs.}
\label{table:elf_sec_headers}
\end{table}

As it can be observed in table \ref{table:elf_sec_headers}, we can find that all sections have the Alloc flag, meaning they will be loaded into process memory during runtime (see table \ref{TODO}, they have not been shown in previous diagrams for simpleness). 

Apart from those we have already discussed previously, we can find the GOT and PLT sections, whose purpose is to support Position Independent Code (PIC), that is, instructions whose address in virtual memory is not hardcoded by the compiler into the executable, but rather they are not known until resolved at runtime. This is usually the case of shared libraries, which can be loaded into virtual memory starting at any address \cite{plt_got_overlord}.

Therefore, in order to call a function of a shared library, the dynamic linker follows a process called 'Lazy binding' \cite{plt_got_technovelty}:
\begin{enumerate}
\item From the .text section, instead of calling a direct absolute address as usual, a PLT stub (in the .plt section) is called. Snippet \ref{code:lazy_bind_1} shows a call to the function timerfd\_settime, implemented by the shared library glibc and thus using a PLT.
\begin{lstlisting}[language=C, caption={Call to PLT stub seen from objdump.}, label={code:lazy_bind_1}]
$ objdump -d simple_timer
4014cb:	b9 00 00 00 00       	mov    $0x0,%ecx
4014d0:	be 01 00 00 00       	mov    $0x1,%esi
4014d5:	89 c7                	mov    %eax,%edi
4014d7:	e8 44 fc ff ff       	call   401120 <timerfd_settime@plt>
\end{lstlisting}

\item In the PLT stub, the flow of execution jumps to an address which is stored in the GOT section, which is the absolute address of the function at glibc. This address must be written there by the dynamic linker but, according to lazy binding, the first time to call this function the linker has not calculated that address yet. 

\begin{figure}[htbp]
	\centering
	\includegraphics[width=15.5cm]{sch_gdb_plt.png}
	\caption{PLT stub for timerfd\_settime, seen from gdb-peda.}
	\label{fig:lazy_bind_2}
\end{figure}

\begin{figure}[htbp]
	\centering
	\includegraphics[width=15.5cm]{sch_gdb_got_prev.png}
	\caption{Inspecting address stored in GOT section before dynamic linking, seen from gdb-peda.}
	\label{fig:lazy_bind_3}
\end{figure}

\item As we can see in figures \ref{fig:lazy_bind_2} and \ref{fig:lazy_bind_3}, the PLT stub calls address 0x4010a0, which leads to a dynamic linking routine, which proceeds to write the address into the GOT section and jump back to the start of the PLT stub. This time, the memory address at GOT to which the PLT jumps is already loaded with the address to the function at the shared library, as shown by figure \ref{fig:lazy_bind_4}.

\begin{figure}[htbp]
	\centering
	\includegraphics[width=15.5cm]{sch_gdb_got_after.png}
	\caption{Inspecting address stored in GOT section after dynamic linking, seen from gdb-peda.}
	\label{fig:lazy_bind_4}
\end{figure}

\begin{figure}[htbp]
	\centering
	\includegraphics[width=15.5cm]{sch_glibc_func.png}
	\caption{Glibc function to which PLT jumps using address stored at GOT, seen from gdb-peda.}
	\label{fig:lazy_bind_5}
\end{figure}

\end{enumerate}

Therefore, in essence, when using lazy binding the dynamic linker will individually load into GOT the addresses of the functions at the shared libraries, during the first time they are called in the program. After that, the address will remain in the GOT section and will be used by the PLT for all subsequent calls.

The reason lazy binding matters to us is because, as we will explain section \ref{subsection:got_attack}, the GOT section is actually writable from an eBPF program. This is because this section specifically must be writeable at runtime for the dynamic linker to store the address once they are resolved. Therefore, we would be able to modify the GOT section from eBPF, redirecting the address at which the PLT jumps, and thus controlling the flow of execution in the program. 

\subsection{Hardening ELF binaries} \label{subsection:hardening_elf}
During section \ref{section:attacks_stack}, we presented multiple of the classic attacks at the stack such as buffer overflow and ROP. However, as we mentioned, during the years multiple hardening measures have been introduced into modern compilers, which attempt to mitigate these and other techniques. We will now present them so that, during the design of our rootkit, we can attempt to bypass all of these.

Table \ref{table:compilers} shows the compilers that we will be considering during this study. We will be exclusively looking at those security features that are included by default.

\begin{table}[htbp]
\begin{tabular}{|>{\centering\arraybackslash}p{5cm}|>{\centering\arraybackslash}p{9cm}|}
\hline
Compiler & Security features by default\\
\hline
\hline
Clang/LLVM 12.0.0 (2021) & Stack canaries, DEP/NX, ASLR\\
\hline
GCC 10.3.0 (2021) & Stack canaries, DEP/NX, ASLR, PIE, Full RELRO\\
\hline 
\end{tabular}
\caption{Security features in C compilers used in the study.}
\label{table:compilers}
\end{table}

\textbf{Stack canaries}\\
Stack canaries are random data that is pushed into the stack before calling potentially vulnerable functions (such as strcpy()) that attempts to prevent attacks at the stack by ensuring that their value is the same before and after the execution of the called function. It is particularly useful at detecting buffer overflow attacks.

If a stack canary is present and a buffer overflow happened, it would potentially overwrite the value of the canary, therefore alerting of the attack, in which case the processor halts the execution of the program.

\textbf{DEP/NX}\\
Data Execution Prevention, also known as No Execute, is the option of marking the stack as non-executable. This prevents, as we explained in section \ref{subsection: buf_overflow}, the possibility of executing injected shellcode in the stack after modifying the value of the saved rip.

The creation of advanced techniques like ROP is one reaction to this mitigation, that circumvents this protection.

\textbf{ASLR}\\
Address Space Layout Randomization is a technique that randomizes the position of memory sections in a process virtual memory, including the heap, stack and libraries, so that an attacker cannot rely on known addresses during exploitation (e.g.: libraries are loaded at a different memory address each time the program is run, so ROP gadgets change their position) \cite{aslr_pie_intro}.

In the context of a stack buffer overflow attack, the memory position of the stack is random, and therefore even if shellcode is injected into the stack by an attacker, the address at which it resides cannot be written into the saved value of rip in order to hijack the flow of execution.

\textbf{PIE}\\
Position Independent Executable is a mitigation introduced to reduce the ability of an attacker to locate symbols in virtual memory by randomizing the base address at which the program itself (including the .text section) is loaded. This base address determines an offset which is added to all memory addresses in the code, so that each instruction is located at an address + this offset. Therefore, all jumps are made using relative addresses \cite{aslr_pie_intro}.

\textbf{RELRO}\\
Relocation Read-Only is a hardening technique that mitigates the possibility of an attacker overwriting the GOT section, as we explained at section \ref{subsection:elf_lazy_binding}. In order to achieve the lazy binding process is substituted by the linker resolving all entries in the GOT section right after the beginning of the execution, and then marking the .got section as read-only. 

Two settings for RELRO are the most widespread, either Partial RELRO (which only marks sections of the .got section not related to the PLT as read-only, leaving .got.plt writeable) or Full RELRO (which marks the .got section as read-only completely). Binaries with only Partial RELRO are still non-secure, as the address at which the PLT section jumps can still be overwriten (including from eBPF, as we will explain) \cite{relro_redhat}.

\textbf{Intel CET}\\
Intel Control-flow Enforcement Technology is a hardening feature fully incorporated in Windows 10 systems \cite{cet_windows} and a work in progress in Linux \cite{cet_linux}. Its purpose is to defeat ROP attacks and other derivates (e.g: Jump-oriented programming, JOP), by adding a strict kernel-supported control of the return addresses and strong restrictions over jump and call instructions.

In Linux, the kernel will support a hidden 'shadow stack' that will save the return addresses for each call. This prevents modifying the saved value of rip in the stack, since the kernel would realise that the flow of execution has been modified. We can also find that modern compilers (such as GCC 10.3.0) already generate Intel CET-related instructions such as \textit{endbr64}, whose purpose is to be placed at the start of functions, marking that as the only address to which an indirect jump can land (otherwise, jumps will be rejected if not landing at \textit{endbr64}).

As mentioned, we will not consider this feature since it is not active in the Linux kernel.

\section{The proc filesystem} \label{section:proc_filesystem}
The proc filesystem is a virtual filesystem which provides an interface to kernel data structures \cite{proc_fs}. It can be found mounted automatically at \textit{/proc}.

This filesystem offers a great range of capabilities to interact with the kernel internal structures, however, in this section, we will focus on the most relevant files and directories for our research.

Specifically, we will be studying the files under the \textit{/proc/<pid>/} directory, whose purpose is to expose information about the process with the corresponding process ID.

Note that the access control for the \textit{/proc/<pid>/} is governed by the value set at \textit{/proc/sys/kernel/yama/ptrace\_scope}. Table \ref{table:yama_values} show its possible values.

\begin{table}[htbp]
\begin{tabular}{|>{\centering\arraybackslash}p{3cm}|>{\centering\arraybackslash}p{11cm}|}
\hline
Value & Description\\
\hline
\hline
0 & Unprivileged processes may access any file or subdirectory\\
\hline
1 & Only privileged processes or those belonging to that PID may access the any file. Unprivileged process can still list the directories at \textit{/proc}, finding the complete list of running processes.\\
\hline
2 & Only privileged processes or those belonging to that PID may access the any file. Unlike with setting '1', unprivileged users cannot list the directores at \textit{/proc} anymore.\\ 
\hline
\end{tabular}
\caption{Values for \textit{/proc/sys/kernel/yama/ptrace\_scope}.}
\label{table:yama_values}
\end{table}

In Ubuntu 21.04, the value of this setting is of '1', therefore the access is limited to users with root privileges or to unprivileged users accessing only their own or their children process information.

\subsection{/proc/<pid>/maps} \label{subsection:proc_maps}
This file provides, for the process with process ID <pid>, its mapped memory regions and their access permissions, that is, those virtual memory pages actively connected to a physical memory page (as shown in figure \ref{fig:mem_arch_pages}).

Figure \ref{fig:proc_maps_sample} shows the maps file of a simple program. As we can observe, by reading this file we can get information such as:
\begin{itemize}
\item The virtual addresses that limit each memory section.
\item The permissions over each memory section.
\item In the case of memory from a file, the offset from which the data was loaded.
\item A pathname, in the case that memory section was loaded from a file.

The ability to easily find memory sections on the virtual address space of a process with a specific set of permissions is particularly relevant for this research. Also, apart from disclosing the address of the stack (and sometimes the heap too), we can infer the address of other memory sections such as the .text section, which must be the only one marked as executable (in figure \ref{fig:proc_maps_sample}, the second entry that appears).

\end{itemize}

\begin{figure}[htbp]
	\centering
	\includegraphics[width=15.5cm]{sch_proc_maps_sample.png}
	\caption{File /proc/<pid>/maps of a sample program.}
	\label{fig:proc_maps_sample}
\end{figure}

\subsection{/proc/<pid>/mem}
This file enables a process to access the virtual memory of the process with process id <pid>. According to the documentation, "this file can be used to access the pages of a process's memory through open(2), read(2), and lseek(2)" \cite{proc_fs}, meaning that we can read any memory address from the virtual memory space of the process.

However, we found the documentation not to be complete. In our experience, not only we can read virtual memory, but also freely write into it. There existed some discussions in the Linux community, and it was considered safe enough to be set as writeable by privileged programs \cite{proc_mem_write}, although the changes were never reflected in the official documentation.

Apart from being able to write into virtual memory, this write accesses are performed without regard of the permission flags set on each memory section. Therefore, we can modify non-writeable virtual memory by writing into the \textit{/proc/<pid>/mem} file.


\chapter{Analysis of offensive capabilities} \label{chapter:analysis_offensive_capabilities}
In the previous chapter, we detailed which functionalities eBPF offers and studied its underlying architecture. As with every technology, a prior deep understanding is fundamental for discussing its security implications. 

Therefore, given the previous background, this chapter is dedicated to an analysis in detail of the security implications of a malicious use of eBPF. For this, we will firstly explore the security features incorporated in the eBPF system. Then, we will identify the fundamental pillars onto which malware can build their functionality. As we mentioned during the project goals, these main topics of research will be the following:
\begin{itemize}
\item Analysing eBPF's possibilities when hooking system calls and kernel functions.
\item Learning eBPF's potential to read/write arbitrary memory.
\item Exploring networking capabilities with eBPF packet filters.
\end{itemize}


\section{eBPF maps security}
In section \ref{subsection:access_control}, we explained that only programs with CAP\_SYS\_ADMIN are allowed to iterate over eBPF maps. The reason why this is restricted to privileged programs is because it is functionality that is a potential security vulnerability, which we will now proceed to analyse.

Also, in section \ref{subsection:ebpf_maps}, we mentioned that eBPF maps are opened by specifying an ID (which works similarly to the typical file descriptors), while in table \ref{table:ebpf_map_types} we showed that, for performing operations over eBPF maps using the bpf() syscall, the map ID must be specified too. 

Map IDs are known by a program after creating the eBPF map, however, a program can also explore all the available maps in the system by using the BPF\_MAP\_GET\_NEXT\_ID operation in the bpf() syscall, which allows for iterating through a complete hidden list of all the maps created. This means that privileged programs can find and have read and write access to any eBPF map used by any program in the system.

Therefore, a malicious privileged eBPF program can access and modify other programs' maps, which can lead to:
\begin{itemize}
\item Modify data used for the program operation. This is the case for maps which mainly store data structures, such as BPF\_MAP\_TYPE\_HASH.
\item Modify the program control flow, altering the instructions executed by an eBPF program. This can be achieved if a program is using the bpf\_tail\_call() helper (introduced in table \ref{table:ebpf_helpers}) which is taking data from a map storing eBPF programs (BPF\_MAP\_TYPE\_PROG\_ARRAY, introduced in table \ref{table:ebpf_map_types}).
\end{itemize}


\section{Abusing tracing programs}
eBPF tracing programs (kprobes, uprobes and tracepoints) are hooked to specific points in the kernel or in the user space, and call probe functions once the flow of execution reaches the instruction to which they are attached. This section details the main security concerns regarding this type of programs.

\subsection{Access to function arguments} \label{subsection:tracing_arguments}
As we saw in section \ref{section:ebpf_prog_types}, tracing programs receive as a parameter those arguments with which the hooked function originally was called. These parameters are read-only and thus, in principle, they cannot be modified inside the tracing program (we will show this is not entirely true in section \ref{section:mem_corruption}). The next code snippets show the format in which parameters are received when using libbpf (Note that libbpf also includes  some macros that offer an alternative format, but the parameters are the same).

\begin{lstlisting}[language=C, caption={Probe function for a kprobe on the kernel function vfs\_write.}, label={code:format_kprobe}]
SEC("kprobe/vfs_write")
int kprobe_vfs_write(struct pt_regs* ctx){
\end{lstlisting}

\begin{lstlisting}[language=C, caption={Probe function for an uprobe, execute\_command is defined from user space.}, label={code:format_uprobe}]
SEC("uprobe/execute_command")
int uprobe_execute_command(struct pt_regs *ctx){
\end{lstlisting}

\begin{lstlisting}[language=C, caption={Probe function for a tracepoint on the start of the syscall sys\_read.}, label={code:format_tracepoint}]
SEC("tp/syscalls/sys_enter_read") 
int tp_sys_enter_read(struct sys_read_enter_ctx *ctx) { 
\end{lstlisting}

In code snippets \ref{code:format_kprobe} and \ref{code:format_uprobe} we can identify that the parameters are passed to kprobe and uprobe programs as a pointer to a \textit{struct pt\_regs*}. This struct contains as many attributes as registers exist in the system architecture, in our case x86\_64. Therefore, on each probe function, we will receive the state of the registers at the original hooked function. This explains the format of the \textit{struct pt\_regs}, shown in code snippet \ref{code:format_ptregs}:

\begin{lstlisting}[language=C, caption={Format of struct pt\_regs.}, label={code:format_ptregs}]
struct pt_regs {
	long unsigned int r15;
	long unsigned int r14;
	long unsigned int r13;
	long unsigned int r12;
	long unsigned int bp;
	long unsigned int bx;
	long unsigned int r11;
	long unsigned int r10;
	long unsigned int r9;
	long unsigned int r8;
	long unsigned int ax;
	long unsigned int cx;
	long unsigned int dx;
	long unsigned int si;
	long unsigned int di;
	long unsigned int orig_ax;
	long unsigned int ip;
	long unsigned int cs;
	long unsigned int flags;
	long unsigned int sp;
	long unsigned int ss;
};
\end{lstlisting}

By observing the value of the registers, we are able to extract the parameters of the original hooked function. This can be done by using the System V AMD64 ABI\cite{8664_params_abi}, the calling convention used in Linux. Depending on whether we are in the kernel or in user space, the registers used to store the values of the function arguments are different. Table \ref{table:systemv_abi} summarizes these two interfaces. 

\begin{table}[H]
\begin{tabular}{|>{\centering\arraybackslash}p{2cm}|>{\centering\arraybackslash}p{3cm}|}
\hline
\multicolumn{2}{|c|}{User interface}\\
\hline
Register & Purpose\\
\hline
\hline
rdi & 1st argument\\
\hline
rsi & 2nd argument\\
\hline
rdx & 3rd argument\\
\hline
rcx & 4th argument\\
\hline
r8 & 5th argument\\
\hline
r9 & 6th argument\\
\hline
rax & Return value\\
\hline
\end{tabular}
\quad
\begin{tabular}{|>{\centering\arraybackslash}p{2cm}|>{\centering\arraybackslash}p{3cm}|}
\hline
\multicolumn{2}{|c|}{Kernel interface}\\
\hline
Register & Purpose\\
\hline
\hline
rdi & 1st argument\\
\hline
rsi & 2nd argument\\
\hline
rdx & 3rd argument\\
\hline
r10 & 4th argument\\
\hline
r8 & 5th argument\\
\hline
r9 & 6th argument\\
\hline
rax & Return value\\
\hline
\end{tabular}
\caption{Argument passing convention of registers for function calls in user and kernel space respectively.}
\label{table:systemv_abi}
\end{table}

In the case of tracepoints, we can see in code snippet \ref{code:format_tracepoint} that it receives a \textit{struct sys\_read\_enter\_ctx*}. This struct must be manually defined, as explained in \ref{subsection:tracepoints}, by looking at the file \textit{/sys/kernel/debug/tracing/events/syscalls/sys\_enter\_read/format}. Code snippet \ref{code:sys_enter_read_tp} shows the format of the struct.

\begin{lstlisting}[language=C, caption={Format for parameters in sys\_enter\_read specified at the format file.}, label={code:sys_enter_read_tp_format}]
field:unsigned short common_type; offset:0; size:2; signed:0;
field:unsigned char common_flags; offset:2; size:1; signed:0;
field:unsigned char common_preempt_count; offset:3; size:1; signed:0;
field:int common_pid; offset:4; size:4; signed:1;
field:int __syscall_nr;	offset:8; size:4; signed:1;
field:unsigned int fd; offset:16; size:8; signed:0;
field:char * buf; offset:24; size:8; signed:0;
field:size_t count; offset:32; size:8; signed:0;
\end{lstlisting}

\begin{lstlisting}[language=C, caption={Format of custom struct sys\_read\_enter\_ctx.}, label={code:sys_enter_read_tp}]
struct sys_read_enter_ctx {
    unsigned long long pt_regs;
    int __syscall_nr;
    unsigned int padding;
    unsigned long fd;
    char* buf;
    size_t count;
};
\end{lstlisting}

As we can observe, we are given a set of attributes which include the parameters with which the syscall was called. Moreover, we can still obtain an address pointing to another \textit{struct pt\_regs}, as in kprobes and uprobes, by combining the first four fields and considering it as a 32-bit long address. This means we will still be able to extract the value of the rest of the registers too. 

It must be noted that, in syscalls, in addition to use the kernel parameter passing convention specified in table \ref{table:systemv_abi}, the number specifying the syscall must be passed in register rax too.

On a final note, as we mentioned in section \ref{section:ebpf_prog_types}, there exist differences in the parameters received in probe functions depending on the two variations of tracing programs. Therefore:
\begin{itemize}
\item kprobe, uprobe and \textit{enter} tracepoints will receive the full parameters as we specified before, but not the return value of the function (since it is not executed yet).
\item kretprobes, uretprobes and \textit{exit} tracepoints will still receive the \textit{struct pt\_regs}, but without any of the parameters and with only the return value of the function.
\end{itemize}

Taking into account all the previous, the fact that tracing programs have read-only access to function arguments can be considered an useful and needed feature for tracing applications, but malicious eBPF can use this for purposes such as:
\begin{itemize}
\item Gather kernel and user data passed to a function as a parameter. In many cases this information can be potentially interesting for an attacker, such as passwords.
\item Store in eBPF maps information about system activities, to be used by other malicious eBPF programs.
\end{itemize}

Usually, since many function arguments are pointers to user or kernel addresses (such as buffers where a string or a struct with data is located), eBPF tracing programs can use two eBPF helpers that enable to read large byte arrays from both kernel and user space:
\begin{itemize}
\item bpf\_probe\_read\_user()
\item bpf\_probe\_read\_kernel()
\end{itemize}

These helpers, previously introduced in table \ref{table:ebpf_helpers}, enable to read an arbitrary number of bytes from an user or kernel address respectively, allowing us to extract the information pointed by the parameters received by eBPF programs.

\subsection{Reading memory out of bounds} \label{subsection:out_read_bounds}
As we introduced in the previous subsection, the bpf\_probe\_read\_user() and bpf\_probe\_read\_kernel() helpers can be used to access memory of pointers received as parameters in the hooked functions. 

However, although in general the eBPF verifier attempts to reject illegal memory accesses, it does not prevent a malicious program from passing an arbitrary memory address (in kernel or user space) to the above helpers. This means that an eBPF program can potentially read any address in user or kernel space, (as long as it is marked as readable in the corresponding memory pages). Furthermore, an attacker can locate specific data structures and memory sections by taking the function parameter as a reference point in memory.

A particularly relevant case (which we will later use for our rootkit) involves accessing user memory via the parameters of tracepoints attached at system calls. Provided the nature of syscalls, whose purpose is to communicate user and kernel space, all parameters received will belong to the user space, and therefore any pointer passed will be an address in user memory. This enables an eBPF program to get a foothold into the virtual address space of the process calling the syscall, which it can proceed to scan looking for data or specific instructions. This technique will be further elaborated in section \ref{subsection_bpf_probe_write_apps}.

\subsection{Overriding function return values}
A potentially dangerous functionality in eBPF tracing programs is the ability to modify the return value of kernel functions\cite{ebpf_friends_p15}\cite{ebpf_override_return}. This can be done via the eBPF helper bpf\_override\_return, and it works exclusively from kretprobes.

Apart from only working on kretprobes, additional restrictions are applied to this helper. It will only work if the kernel was compiled with the CONFIG\_BPF\_KPROBE\_OVERRIDE flag, and only if the kretprobe is attached to a function to which, during the kernel development, the macro ALLOW\_ERROR\_INJECTION() has been indicated. Currently, only a small selection of functions include this macro, but most system calls can be found to implement it. The following code snippets show how a system call like sys\_open is defined in kernel v5.11:

\begin{lstlisting}[language=C, caption={Definition of the syscall sys\_open in the kernel \cite{code_kernel_open}}, label={code:override_return_1}]
SYSCALL_DEFINE3(open, const char __user *, filename, int, flags, umode_t, mode)
{
	if (force_o_largefile())
		flags |= O_LARGEFILE;
	return do_sys_open(AT_FDCWD, filename, flags, mode);
}
\end{lstlisting}

\begin{lstlisting}[language=C, caption={Definition of the macro for creating syscalls, containing the error injection macro. Only relevant instructions included, complete macro can be found in the kernel \cite{code_kernel_syscall}}, label={code:override_return_2}]
#define SYSCALL_DEFINE3(name, ...) SYSCALL_DEFINEx(3, _##name, __VA_ARGS__)
#ifndef __SYSCALL_DEFINEx
#define __SYSCALL_DEFINEx(x, name, ...)\
	[...]
	ALLOW_ERROR_INJECTION(sys##name, ERRNO);\
	[...]
\end{lstlisting}


By looking at snippets \ref{code:override_return_1} and \ref{code:override_return_2}, we can observe that the system call sys\_open involves the inclusion of the ALLOW\_ERROR\_INJECTION macro. Therefore, any kretprobe attached to a system call function will be able to modify its return value.

In order to be able to modify the return value of functions, the aforementioned eBPF helper makes use of the fault injection framework of the Linux kernel\cite{fault_injection}, which was created before eBPF itself, and whose original purpose is to allow for generating errors in kernel programs for debugging purposes.

Taking the previous information into account, we can find that a malicious eBPF program, by tampering with the kernel-user space interface which are system calls, can mislead user programs, which trust the output of kernel code. This can lead to:
\begin{itemize}
\item A program believes a system call exited with an error, while in reality the kernel completed the operation with success, or viceversa. For instance, the result of a call to sys\_open can mislead a user program into thinking that a file does not exist.
\item A program receives incorrect data on purpose. For instance, a buffer may look empty or of a reduced size upon a sys\_read call, while in reality more data is available to be read.
\end{itemize}

\subsection{Sending signals to user programs}
Another eBPF helper that is subject to malicious purposes is bpf\_send\_signal. This helper enables to send an arbitrary signal to the thread of the process running a hooked function.

Therefore, this helper can be used to forcefully terminate running user processes, by sending the SIGKILL signal. In this way, combined with the observability into the parameters received at a function call, malicious eBPF can kill and deactivate processes to favour its malicious purposes.

\subsection{Takeaways} \label{subsection:tracing_attacks_conclusion}
As a summary, a malicious eBPF program loaded and attached as a tracing program undermines the existing trust between user programs and the kernel space. 

Its ability to access sensitive data in function parameters and reading arbitrary memory can lead to gathering extensive information on the running processes of a system, whilst the malicious use of eBPF helpers enables the modification of the data passed to the user space from the kernel, and the control over which programs are allowed to be running on the system.

\section{Memory corruption} \label{section:mem_corruption}
In the previous section we described how tracing programs can read user memory out of the bounds of function parameters via the helpers bpf\_probe\_read\_user() and bpf\_probe\_read\_kernel(). In this section, we will analyse another eBPF helper can be found to be the heart of malicious programs.

Privileged eBPF programs (or those with at least CAP\_BPF + CAP\_PERFMON capabilities) have the potential to use an experimental (it is labelled as so \cite{ebpf_helpers}) helper called bpf\_probe\_write\_user(). This helper enables to write into user memory from within an eBPF program. 

However, this helper has certain limitations that restrict its use. We will now proceed to review some background into how user memory works and, afterwards, we will analyse the restrictions and possible uses of this eBPF helper in the context of malicious applications.


\subsection{Attacks and limitations of bpf\_probe\_write\_user()} \label{subsection:bpf_probe_write_apps}
Provided the background into memory architecture and the stack operation, we will now study the offensive capabilities of the bpf\_probe\_write\_user() helper and which restrictions are imposed into its use by eBPF programs.

The bpf\_probe\_write\_user() helper, when used from a tracing eBPF program, can write into any memory address in the user space of the process responsible from calling the hooked function. However, the write operation fails has some restrictions:
\begin{itemize}
\item{The operation fails if the memory space pointed by the address is marked as non-writeable by the user space process. For instance, if we try to write into the .text section, the helpers fails because this section is only marked as readable and executable (for protection reasons).} Therefore, the process must indicate a writeable flag in the memory section for the helper to succeed.
\item{The operation fails if the memory page is served with a minor or major page fault \cite{bpf_probe_write_user_errors}. As we saw in section \ref{subsection:ebpf_verifier}, eBPF programs are restricted from executing any sleeping or blocking operations, to prevent hanging the kernel. Therefore, since during a page fault the operating system needs to block the execution and write into the page table or retrieve data from the secondary disk, bpf\_probe\_write\_user() is defined as a non-faulting helper\cite{write_helper_non_fault}, meaning that instead of issuing a page fault for accessing data, it will just return and fail.}
\item{Each time the helper is called, an alert message is written into the kernel logs, alerting that a potentially dangerous eBPF program is making use of the helper. Note that this message appears when the eBPF program is attached, and not each time the helper is called. This will be particularly relevant since we will be able to bypass this alert by taking advantage of this.}
\end{itemize}

Although we will not be able to modify kernel memory or the instructions of a program, this eBPF helper opens a range of possible attacks:
\begin{itemize}
\item Modify any of the arguments with which a system call is called (either with a tracepoint or a kprobe). Therefore, a malicious program can hijack any call to the kernel with its own arguments.
\item Modify user-provided arguments in kernel functions. When reading kernel code, we can find that data provided by the user is marked with the keyword \textit{\_\_user}. For instance, an internal kernel function in a nested call of the system call sys\_read receives an user buffer:
\begin{lstlisting}[language=C, caption={Definition of kernel function vfs\_read. \cite{code_vfs_read}}, label={code:vfs_read}]
ssize_t vfs_read(struct file *file, char __user *buf, size_t count, loff_t *pos)
\end{lstlisting}
Then, if we attach a kprobe to vfs\_read, we would be able to modify the value of the buffer.
\item Modify process memory by taking function parameters as a reference and scanning the stack. This technique, first introduced in section \ref{subsection:out_read_bounds} when we mentioned that tracing programs can read any user memory location with the bpf\_probe\_read\_user() helper, and which was publicly first used by Jeff Dileo at his talk in DEFCON 27\cite{evil_ebpf_p6974}, consists of:
\begin{enumerate}
\item Take an user-passed parameter received on a tracing program. The parameter must be a pointer to a memory location (such as a pointer to a buffer), so that we can use that memory address as the reference point in user space. According to the x86\_64 documentation, this parameter will be stored in the stack\cite{8664_params_abi_p1922}, so we will receive an stack address.
\item Locate the target data which we aim to write. There are two main methods for this:
\begin{itemize}
	\item Sequentially read the stack, using bpf\_probe\_read\_user(), until we locate the bytes we are looking for. This requires knowing which data we want to overwrite.
	\item By previously reverse engineering the user program, we can calculate the offset at which an specific data section will be stored in virtual memory with respect to the reference address we received as a parameter.
\end{itemize}
\item Overwrite the memory buffer using bpf\_probe\_write\_user().
\end{enumerate}
\end{itemize}

Figure \ref{fig:stack_scan_write_tech} illustrates a high-level overview of the stack scanning technique previously described.
%TODO i just noticed I included SFP outside the current stack frame, correct it here and everywhere
\begin{figure}[H]
	\centering
	\includegraphics[width=16cm]{stack_scan_write_tech.jpg}
	\caption{Overview of stack scanning and writing technique.}
	\label{fig:stack_scan_write_tech}
\end{figure}

The figure shows process memory executing a program similar to the following:
\begin{lstlisting}[language=C, caption={Sample program being executed on figure \ref{fig:stack_scan_write_tech}.}, label={code:stack_scan_write_tech}]
void func(char* a, char* b, char* c){
	int fd = open("FILE", 0);
	write(fd, a, 1);
}

int main(){
	char a[] = "AAA";
	char b[] = "BBB";
	char c[] = "CCC";
	func(a, b, c);
}
\end{lstlisting}

In the figure, we can clearly observe how the technique is used to overwrite an specific buffer. The attacker goal is to overwrite buffer \textit{c} with some other bytes, but the kprobe program only has direct access to buffer \textit{a}:
\begin{enumerate}
\item By reverse engineering the program (we will see how this process works in section \ref{TODO}) we notice that buffer \textit{c} is stored 8 bytes lower on the stack than buffer \textit{a}.
\item When register rip points to the write() instruction, the processor executes the instruction and a system call is issued to sys\_write().
\item The kprobe eBPF program hooked to the syscall hijacks the program execution. Since it has access to the memory address of buffer \textit{a} and it knows the relative position of buffer \textit{c}, it writes to that location whatever it wants (e.g.: "DDD") with the bpf\_probe\_write\_user() helper.
\item The eBPF program ends and the control flow goes back to the system call. It ends its execution successfully, and returns a value to the user space. The result of the program is that 1 byte has been written into file "FILE", and that buffer \textit{c} now contains "DDD".
\end{enumerate}

\subsection{Takeaways}
As a summary, the bpf\_probe\_write\_user() helper is one of the main attack vectors for malicious eBPF programs. Although it does contain some restrictions, its ability to overwrite any user parameter enables it to, in practice, execute arbitrary code by hijacking that of others. When it is combined with tracing programs' ability to read memory out of bounds, it unlocks a wide range of attacks, since any writeable section of the process memory is a possible target. 

Therefore, if on the conclusion of section \ref{subsection:tracing_attacks_conclusion} we discussed that the ability to change the return value of kernel functions and kill processes hinders the trust between the user and kernel space (since what the kernel returns may not be a correct result), then the ability to directly overwrite process data is a complete disrupt of trust in any of the data in the user space itself, since it is subject to the control of a malicious eBPF program.

Moreover, in the next sections we will discuss how we can create advanced attacks on the basis of the background and techniques previously discussed. We will research further into which sections of a process memory are writeable and whether they can lead to new attack vectors.


\section{Abusing networking programs}\label{section:abusing_networking}
The final main piece of a malicious eBPF program comes from taking advantage of the networking capabilities of TC and XDP programs. As we mentioned during sections \ref{subsection:xdp} and \ref{subsection:tc}, these type of programs have access to network traffic:
\begin{itemize}
\item Traffic Control programs can be placed either on egress or ingress traffic, and receive a struct \textit{sk\_buff}, containing the packet bytes and meta data that helps operating on it.
\item External Data Path programs can only be attached to ingress traffic, but in turn they receive the packet before any kernel processing (as a struct \textit{xdp\_md}) being able to access the raw data directly.
\end{itemize}

Networking eBPF programs not only have read access to the network packets, but also write access:
\begin{itemize}
\item XDP programs can directly modify the raw packet via memcpy() operations. They can also increment or reduce the size of the packet at any of its ends (adding bytes before the head or after the packet tail). This is done via the multiple helpers previously presented on table \ref{table:xdp_helpers}.
\item TC programs can also modify the packet via the helpers presented on table \ref{table:tc_helpers}. The packet can be expanded or reduced via these eBPF helpers too.
\end{itemize} 

Apart from write access to the packet, the other critical feature of networking programs is their ability to drop packets. As we presented in tables \ref{table:xdp_actions_av} and \ref{table:tc_actions}, this can be achieved by returning specific values.


\subsection{Attacks and limitations of networking programs}
Based on the previous background, we will now proceed to explore which limitations exist on which actions a network eBPF program can perform:
\begin{itemize}
\item Read and write access to the packet is heavily controlled by the eBPF verifier. It is not possible to read or write data out of bounds. Extreme care must also be taken before attempting to read any data inside the packet, since the verifier first requires making lots of checks beforehand. For any access to take place, the program must first classify the packet according to the network protocol it belongs, and later check that every header of every layer is well defined (e.g: Ethernet, IP and TCP). Only after that, the headers can be modified. 

If the program also wants to modify the packet payload, then it must be checked to be between the bounds of the packet and well defined according to the packet headers(using fields IHL, packet length and data offset, in figure \ref{fig:frame}). Also, after using any of the helpers that enlarge or reduce the size of the packet, all check operations must be repeated again before any subsequent operation.

Finally, note that after any modification in the packet, some network protocols (such as IP and TCP) require to recalculate their checksum fields. 

\item XDP and TC programs are not able to create packets, they can only operate over existing traffic.

\item If an XDP program modifies an incoming packet, the kernel will not know about the original data, but if an egress TC program modifies a packet being sent, the kernel will be able to notice the modification.
\end{itemize}

Having the previous restrictions in mind, we can find multiple possible malicious uses of an XDP/TC program:
\begin{itemize}
\item \textbf{Spy all network connections} in the system. An XDP or TC ingress program can read any packet from any interface, therefore achieving a comprehensive view on which are the running communications and opened ports (even if protocols with encryption are being used) and gathering transmitted data (if the connection is also in plaintext).
\item \textbf{Hide arbitrary traffic} from the host. If an XDP program drops a packet, the kernel will not be able to know any packet was received in the first place. This can be used to hide malicious incoming traffic. However, as we will mention in section{TODO}, malicious traffic may still be detected by other external devices, such as network-wide firewalls.
\item \textbf{Modify incoming traffic} with XDP programs. Every packet can be modified (as we mentioned at the beginning of section \ref{section:abusing_networking}), and any modification will be unnoticeable to the kernel, meaning that we will have complete, invisible control over the packets received by the kernel.
\item \textbf{Modify outgoing traffic} with TC egress programs. Since every packet can be modified at will, we will therefore have complete control over any packet sent by the host. This can be used to enable a malicious program to communicate over the network and exfiltrate data, since even if we cannot create a new connection from eBPF, we can still modify existing packets, writing any payload and headers on it (thus being able to, for instance, change the destination of the packet).

Notice, however, that these modifications are not transparent to the kernel as with XDP, and thus an internal firewall may detect our malicious traffic.
\end{itemize}

Although we mention the possibility of modifying outgoing traffic as an alternative to the impossibility of sending new packets from eBPF, there exists a major disadvantage by doing this, since the original packet of the application will be lost, and we will thus be disrupting the normal functioning of the system (which in a rootkit is unacceptable, as we mentioned in section \ref{section:motivation}, stealth is a priority).

There exists, however, a simple way of duplicating a packet so that the original packet is not lost but we can still send our overwritten packet. This technique, first presented by Guillaume Fournier and Sylvain Afchainthe in their DEFCON talk, consists of taking advantage of TCP retransmissions we described on section \ref{subsection:tcp}. Figure \ref{fig:tcp_exfiltrate_retrans} shows this process:

\begin{figure}[H]
	\centering
	\includegraphics[width=15cm]{tcp_exfiltrate_retrans.jpg}
	\caption{Technique to duplicate a packet for exfiltrating data.}
	\label{fig:tcp_exfiltrate_retrans}
\end{figure}

In the figure, we can observe a host infected by a malicious TC egress program. An user space application at some point needs to send a packet (in this case a simple ping), and the TC program will overwrite it (in this case, it writes a password which it has been able to find, and substitutes the destination IP address with that of a listening attacker.
After the timer runs out, the TCP protocol itself will retransmit the same packet as previously and thus the original data is delivered too.

Using this technique, we will be able to send our own packets every time an application sends outgoing traffic. And, unless the network is being monitored, this attack will go unnoticed, provided that the delay of the original packet is similar to that when a single packet lost.

\subsection{Takeaways}
As a summary, networking eBPF programs offer complete control over incoming and outgoing traffic. If tracing programs and memory corruption techniques served to disrupt the trust in the execution of both any user or kernel program, then a malicious networking program has the potential to do the same with any communication, since any packet is under the control of eBPF.

Ultimately, the capabilities discussed in this section unlock complete freedom for the design of malicious programs. As we will explain in the next chapter, one particularly relevant type of application can be built:
\begin{itemize}
\item A \textbf{backdoor}, a stealthy program which listens on the network interface and waits for secret instructions from a remote attacker-controlled client program. This backdoor can have \textbf{Command and Control (C2)} capabilities, meaning that it can process commands sent by the attacker and received at the backdoor, executing a series of actions corresponding to the request received, and (when needed) answering the attacker with the result of the command.
\end{itemize}

%TODO maybe a conclusion for this chapter?

\chapter{Design of a malicious eBPF rootkit}
In the previous chapter, we discussed the capabilities of eBPF programs from a security standpoint, detailing which helpers and program types are particularly useful for developing malicious programs, and analysing some techniques (stack scanning, overwriting packets together with TCP retransmissions) which helps us circumvent some of the limitations of eBPF.

Taking as a basis these capabilities, this chapter is now dedicated to a comprehensive description of our rootkit, including the techniques and functionalities implemented, thus showing how these capabilities can lead to the creation of a real malicious application. As we mentioned during the project objectives, our goals for our rootkit include the following:
\begin{itemize}
\item Hijacking the execution of user programs while they are running, injecting libraries and executing malicious code, without impacting their normal execution.
\item Featuring a command-and-control module powered by a network backdoor, which can be operated from a remote client. This backdoor should be controlled with stealth in mind, featuring similar mechanisms to those present in rootkits found in the wild.
\item Tampering with user data at system calls, resulting in running malware-like programs and for other malicious purposes.
\item Achieving stealth, hiding rootkit-related files from the user.
\item Achieving rootkit persistence, the rootkit should run after a complete system reboot.
\end{itemize}

We will firstly present an overview on the rootkit architecture and design. Afterwards, we will be exploring each functionality individually, offering a comprehensive view on how each of the systems work.


\section{Rootkit architecture}
Figure \ref{fig:rootkit} shows an overview of the rootkit modules and components which have been built for this research work.

\begin{figure}[htbp]
	\centering
	\includegraphics[width=15.5cm]{rootkit.png}
	\caption{Overview of the rootkit subsystems and components.}
	\label{fig:rootkit}
\end{figure}

As we can observe in the figure, we can distinguish 6 different rootkit modules, along with a rootkit client which provides remote control of the rootkit over the network from the attacker machine. Also, there exists a rootkit user space process, which is listening for commands issued from the kernel-side, transmitted through a ring buffer.
\begin{itemize}
\item The \textbf{user space process} of the rootkit is in charge of loading and attaching the eBPF rootkit in the kernel, and creating the eBPF maps needed for their operations. For this, it uses the eBPF programs configurator, an internal structure that manages the eBPF modules at runtime, being able to attach or deattach them after a command to do so is received.

The user space process also listens to any data received at the ring buffer, an special map which the eBPF program at the kernel will use to communicate with the user-side, issuing commands and triggering actions from it. Between others actions, the rootkit user space process can spawn TLS clients, execute malicious programs or use the eBPF program configurator for managing the eBPF programs.

\item The \textbf{library injection} module is in charge of hijacking the execution of target processes by injecting a malicious library. For this, it uses a set of eBPF tracepoints in the kernel side, and a code caver module in the user side in charge of scanning user processes and injecting shellcode, apart from the malicious library itself, which is prepared to communicate with the attacker's remote client.

\item The \textbf{execution hijacking} module is in charge of hijacking the execution of programs right before the process is even created, modifying the kernel function arguments in such a way that the a new malicious program is called, but the original information is not lost so that the malicious program can still create the original process. Therefore, it hijacks the creation of processes by transparently injecting the creation of one additional malicious process on top of the intended one.

\item The \textbf{privilege escalation} module is in charge of ensuring that any user process spawned by the rootkit will maintain full privilege in the system. Therefore, it hijacks any call to the sudoers file (on which privileged users are listed) so that the user on which the rootkit is loaded is always treated as root. Note that we have not listed this module as one of the main project objetives mainly because it acts as a helper to other modules, such as the execution hijacking one.

\item The \textbf{backdoor} is one of the most critical modules in the rootkit. It has full control over incoming traffic with an XDP program, and outgoing traffic with a TC egress program. As we will see, both the XDP and TC programs are loaded in different eBPF programs, so they use a shared eBPF map to communicate between them.

The backdoor maintains a Command and Control (C2) system that is prepared to listen for specially-crafted network triggers which intend to be stealthy and go unnoticed by network firewalls. These triggers transmit information and commands to the XDP program at the network border, which the backdoor is in charge of interpreting and issuing the corresponding actions, either by writing data at an eBPF map in which other eBPF programs are reading, or issuing an action request via the ring buffer. On top of that, the TC program interprets the data parsed by the XDP program and shapes the outgoint traffic, being able to inject secret messages into packets. 

\item The \textbf{rootkit stealth} module is in charge of implementing measures to hide the rootkit from the infected host. For this, it hijacks certain system calls so that rootkit-related files and directories are hidden from the system.

\item The \textbf{rootkit persistence} module is in charge of ensuring that the rootkit will stay loaded even after a complete reboot of the infected system. For this, it injects secret files at the \textit{cron} system (which will launch the rootkit after a reboot) and at the sudo system (which maintains the privileged permissions of the rootkit after the reboot).

\item The \textbf{rootkit client} is a command-line interface (CLI) program that enables the attacker to remotely control the rootkit at the infected machine. For this, it incorporates multiple operation modes that launch different commands and network triggers. These network triggers, and any other packet sent to the backdoor, are customly designed TCP packets sent over a raw socket, enabling to avoid the noisy TCP 3-way handshake and to control every detail of the packet fields. Each of the messages generated by the client (and sent by the backdoor) follow a custom rootkit protocol, that defines the format of the messages and allows both the client and the backdoor to identify those packets belonging to this malicious traffic. In order to craft these packets, the rootkit client uses a raw sockets library (RawTCP\_Lib) that we have developed for this purpose \cite{rawtcp_lib}.

The RawTCP\_Lib library incorporates packets building, raw socket packet transmissions, and a sniffer for incoming packets. This sniffer is particularly relevant since the client will need to listen for responses by the rootkit backdoor and quickly detect those that follow the rootkit protocol format.

Apart from the network triggers, upon receiving a response by the backdoor the rootkit client can start pseudo-shells connections (commands can be sent to the backdoor and the backdoor executes them, but no shell process is spawned in the client), or spawn TLS servers that establish an encrypted connection with the backdoor. This connection, internally, still uses the custom rootkit protocol to act as a pseudo-shell, enabling to execute commands remotey.
\end{itemize}


With respect to how the rootkit implementation is distributed into multiple programs, we can find that, overall, there exist 4 main components, as shown in figure \ref{fig:rootkit_files}.

\begin{figure}[htbp]
	\centering
	\includegraphics[width=15cm]{rootkit_files.jpg}
	\caption{Rootkit programs and scripts.}
	\label{fig:rootkit_files}
\end{figure}

As we can observe in the figure, the rootkit modules we have overviewed previously are distributed into different files:
\begin{itemize}
\item The program \textit{\textbf{injector}} comprises the rootkit client and the shared library RawTCP\_Lib. This program is to be launched from the attacker machine after a successful infection of a host.
\item The program \textit{\textbf{tc}} contains the TC program needed for managing the egress network traffic. The reason why it is loaded separately is because the libbpf library does not currently incorporate support for integrating TC programs easily as with XDP or tracepoints.

This program is also responsible of creating the shared map which the backdoor will use, and therefore it must be the first part of the rootkit loaded.
\item The program \textit{\textbf{kit}} contains most of the rootkit functionality, spawning the user process and the kernel-side eBPF programs and maps.
\item The \textit{\textbf{packager.sh}} and \textit{\textbf{deployer.sh}} files are scripts which an attacker, upon gaining access to a machine, can use to quickly set up the rootkit and infect the machine:
\begin{itemize}
	\item \textit{packager.sh} compiles the rootkit and prepares the \textit{injector}, \textit{kit} and \textit{tc} files in an output directory to be used (this directory is hidden by the rootkit once it is loaded).
	\item \textit{deployer.sh} uses the output directory to launch the rootkit files in order (first \textit{tc}, then \textit{kit}). It also injects the necessary files into the sudoers.d and cron.d directories (which will be later hidden by the rootkit) to maintain persistence.
\end{itemize}
\end{itemize}



\section{Library injection module}
In this section, we will discuss how to hijack an user process running in the system so that it executes arbitrary code instructed from an eBPF program. For this, we will be injecting a library which will be executed by taking advantage of the fact that the GOT section in ELFs is flagged as writable (as we introduced in section \ref{subsection:elf_lazy_binding} and using the stack scanning technique covered in section \ref{subsection:bpf_probe_write_apps}. This injection will be stealthy (it must not crash the process), and will be able to hijack privileged programs such as systemd, so that the code is executed as root.

We will also research how to circumvent the protections which modern compilers have set in order to prevent similar attacks (when performed without eBPF), as we overview in section \ref{subsection:hardening_elf}.

This technique has some advantages and disadvantages to the one described by Jeff Dileo at DEFCON 27 \cite{evil_ebpf_p6974}, which we will briefly cover before presenting ours. Both techniques will be later compared in chapter \ref{chapter:related_work}.


\subsection{ROP with eBPF} \label{subsection:rop_ebpf}
In 2019, Jeff Dileo presented in DEFCON 27 the first technique to achieve arbitrary code execution using eBPF \cite{evil_ebpf_p6974}. For this, he used the ROP technique we described in section \ref{subsection:rop} to inject malicious code into a process. We will present an overview on his technique, in order to later compare it to the one we will develop for our rootkit, and find advantages and disadvantages. Note that this is a summary and some aspects have been simplified, however we will go in full detail during the explanation of our own technique.

Figure \ref{fig:rop_evil_ebpf_1} shows an overview on the process memory and the eBPF programs loaded. For this injection, we will use the stack scanning technique (section \ref{subsection:bpf_probe_write_apps}) using the arguments of a system call whose arguments are passed using the stack (sys\_timerfd\_settime, which receives two structs utmr and otmr). Therefore, a kprobe is attached to the system call, so that it can start to scan for the return address of the system call, which we know is the original value of register rip which was pushed into the stack (ret).

%TODO This figure needs a remodel. I tried to keep it simple to explain the main concepts on the technique described afterwards, but after writing the next section I realised it gets some things wrong:
% - It does not show .got and .plt sections.
% - It shows the RBP register in an incorrect place.
\begin{figure}[htbp]
	\centering
	\includegraphics[width=15cm]{rop_evil_ebpf_1.jpg}
	\caption{Initial setup for the ROP with eBPF technique.}
	\label{fig:rop_evil_ebpf_1}
\end{figure}

%TODO I don't quite like this. Maybe the glibc bit, because of its importance, is better somewhere else
An additional aspect must be introduced now (we will cover it more in detail in section \ref{TODO}): system calls are not directly called by the instructions in the .text section, but rather user programs in C make use of the C Standard Library to delegate the actual syscall, which in this case is the GNU Standard Library (glibc) \cite{glibc}. Therefore, a program calls a function in glibc (in this case timerfd\_settime) in which the syscall is performed, and the kernel executes it.

This means that, during the stack scanning technique, if we start from struct utmr and scan forward in the stack, what we will find in ret is the return address of the PLT stub that calls the function at glibc, and not directly that of the syscall to the kernel. Therefore, our goal is, for every data in the stack while scanning forward, check whether it is the real return address of the PLT stub we are looking for. For an address to be the real return address, we will follow the next steps:
\begin{enumerate}
\item Take an address from the stack. If that is the return address (the saved rip), then the instruction that called the PLT stub that jumps to the function in glibc must be the previous instruction (rip - 1).
\item We now have a \textit{call} instruction, that directs us to the PLT stub. We take the address stored at the GOT section and jump to the function at glibc.
\item We scan forward, inside timerfd\_settime of glibc, until we find a \textit{syscall} instruction. That is the point where the flow of execution moves to the kernel, so we have checked that the return address we found in the stack truly is the one we are looking for.
\end{enumerate}

Now that we have found the return address, we save a backup of the stack (to recover the original data later) and we proceed to overwrite the stack using bpf\_probe\_write\_user(), setting it for the ROP technique. For this, some gadgets (G0, G1 ... GN) have been previously discovered in the glibc library. Figure \ref{fig:rop_evil_ebpf_2} shows process memory after this overwrite:

\begin{figure}[H]
	\centering
	\includegraphics[width=15cm]{rop_evil_ebpf_2.jpg}
	\caption{Process memory after syscall exits and ROP code overwrites the stack.}
	\label{fig:rop_evil_ebpf_2}
\end{figure}

As we can see in the figure, the function has already exited, and ret has been popped into register rip. As we explained in section \ref{subsection:rop}, the attacker places in that position the address of the first ROP gadget. After that, the attacker can execute arbitrary code. Jeff Dileo, for instance, loads a malicious library into the process (we will do the same and explain this process in the next sections).

Once the attacker has finished executing the injected code, the stack must be restored to the original position so that the program can continue without crashing. A simplified view of this procedure consists of attaching a kprobe to a random system call (in this case, sys\_close()) so that, from the ROP code, we can alert the eBPF program when it is time to remove the ROP code and restore the original stack. Figure \ref{fig:rop_evil_ebpf_3} shows this final step:

\begin{figure}[H]
	\centering
	\includegraphics[width=15cm]{rop_evil_ebpf_3.jpg}
	\caption{Stack data is restored and program continues its execution.}
	\label{fig:rop_evil_ebpf_3}
\end{figure}

As we can see, eBPF writes back the original stack and thus the execution can continue. Note that, in practice, some final gadgets must also be executed in order to restore the state of rip and rsp, the stack data for this is written in the free memory zone, so that it does not need to be removed.


%TODO Eligible to writing more. This was merged with the explanation of each feature before, so it was more extense, but now it might need some more info??
\subsection{Bypassing hardening features in ELFs} \label{subsection:hardening_bypass}
During section \ref{subsection:hardening_elf}, we presented multiple  security hardening measures that have been introduced to prevent common exploitation techniques (such as stack buffer overflows) and that nowadays can be incorporated, usually by default, in ELF binaries generated using modern compilers. We will now explore how to bypass these features, so that we can design an injection technique that can target any process in the system, independently on whether it was compiled using these mitigations.

\textbf{Stack canaries}\\
Since stack canaries will be checked after the vulnerable function returns, an attacker seeking to overwrite the stack must ensure that the value of the canary remains constant. In the context of a buffer overflow attack, this can be achieved by leaking the value of the canary and incorporating it into the overflowing data at the stack, so that the same value is written on the same address \cite{canary_exploit}.

In our rootkit, unlike in the ROP technique presented in section \ref{subsection:rop_ebpf}, we will avoid overwriting the value of the saved rip in the stack completely. Therefore, as long as our eBPF program leaves all registers and stack data in the same state as before calling the function, we will not trigger any alerts.

\textbf{DEP/NX}\\
The only alternative for an attacker upon a non-executable stack is either injecting shellcode at any other executable memory address, or the use of advanced techniques like ROP that fully circumvent this mitigation since the data at the stack is not directly executed at any step.

In our rootkit, we will choose the first option, scanning the process virtual memory for an executable page where we will inject our shellcode. This process is usually known as finding 'code caves'.

\textbf{ASLR}\\
In order to bypass ASLR, attackers must take into account that, although the address at which, for instance, a library is loaded is random, the internal structure of the library remains unchanged, with all symbols in the same relative position, as figure \ref{table:aslr_offset} shows.

%TODO Add the .data section here
\begin{figure}[htbp]
	\centering
	\includegraphics[width=13cm]{aslr_offset.jpg}
	\caption{Two runs of the same executable using ASLR, showing a library and two symbols.}
	\label{fig:alsr_offset}
\end{figure}

As we can observe in the figure, although glibc is loaded at a different base address each run, the offset between the functions it implements, malloc() and free(), remains constant. Therefore, a method for bypassing ASLR is to gather information about the absolute address of any symbol, which can then easily lead to knowing the address of any other if the attacker decompiles the executable and calculates the offset between a pair of addresses where one is known. This is the chosen method for our technique.

\textbf{PIE}\\
Similarly to ASLR, although the starting base address of each memory section is random, the internal structure of each section remains the same. Therefore, if an attacker is able to leak the address of some symbol in a section, and by knowing the offset at which it is located with respect to the base address of the section, then the address of any other symbol in the same section can be calculated \cite{pie_exploit}. This is the technique we will incorporate in our rootkit.

\textbf{RELRO}\\
If an executable was compiled using Partial RELRO, then the value of GOT can still be overwritten. If in turn it was compiled using Full RELRO, this stops any attempt of GOT hijacking, unless an attacker finds an alternative method for writing into the virtual memory of a process that bypasses the read-only flag. 

In our rootkit, we will directly write using eBPF the value of GOT if it was compiled with Partial RELRO, and use an alternative technique for writing into the virtual memory of a process whenever it was compiled using Full RELRO.


\subsection{Library injection via GOT hijacking} \label{subsection:got_attack}
Taking into account the previous background and that about stack attacks, ELF's lazy binding and hardening features for binaries we presented in section \ref{section:elf}, we will now present the exploitation technique incorporated in our rootkit to inject a malicious library into a running process. 

This attack is based on the possibility of overwriting the data at the GOT section. As we have mentioned previously, this section is marked as writeable if the program was compiled using Partial RELRO, meaning that we will be able to overwrite its value from an eBPF program using the helper bpf\_probe\_write\_user(). After modifying the value of GOT, a PLT stub will take the new value as the jump address (as we explained in section \ref{subsection:elf_lazy_binding}), effectively hijacking the flow of execution of the program. In the case that a program was compiled with Full RELRO (which will be the case of many programs running by default in a Linux system such as systemd), we will make use of the /proc filesystem for overwriting this value.

The rootkit will inject the library once an specific syscall is called by a process, but the library injection will only happen after the second syscall, since we need to wait for the GOT address to be loaded by the dynamic linker. This is a necessary step because eBPF will need to validate that it really is the GOT section to overwrite.

This technique works both in compilers with low hardening fetaures by default (Clang) and also on a compiler with all of them active (GCC), see table \ref{table:compilers}. On each of the steps, we will detail the different existing methods depending on the compiler features.

For this research work, the rootkit is prepared to perform this attack on any process that makes use of either the system call sys\_openat or sys\_timerfd\_settime, which are called by the standard library glibc.

We will now describe the multiple exploitation stages for our technique. Appendix \ref{annexsec:lib_injection} shows a flow diagram with the complete process.


\textbf{Stage 1: eBPF tracing and scan the stack}\\
We load and attach a tracepoint eBPF program at the \textit{enter} position of syscall sys\_timerfd\_settime. Firstly, we must ensure that the process calling the tracepoint is one of the processes to hijack.

We will then proceed with the stack scanning technique, as we explained in section \ref{subsection:bpf_probe_write_apps}. In this case, we will take one of the syscall parameters and scan forward in the stack. For each iteration, we must check if the data at the stack corresponds to the saved return address of the PLT stub that jumps to glibc where the syscall sys\_timerfd\_settime is called. Figure \ref{fig:lib_stage1} shows an overview of how these call instructions relate each memory section. 


\begin{figure}[htbp]
	\centering
	\includegraphics[width=13cm]{plt_got_glibc_flow.jpg}
	\caption{Overview of jump and return instructions from the program instructions to the syscall at the kernel.}
	\label{fig:lib_stage1}
\end{figure}

The following are the steps we will follow to perform check some data at the stack is the saved return address:
\begin{enumerate}
\item Check that the previous instruction is a call instruction, by checking the instruction length and opcodes (call instructions always start with e8, and the length is 5 bytes, see figure \ref{fig:firstcall}).
\begin{figure}[htbp]
	\centering
	\includegraphics[width=13cm]{sch_firstcall.png}
	\caption{Call to the glibc function, using objdump.}
	\label{fig:firstcall}
\end{figure}
\item Now that we know we localized a call instruction, we take the address at which it jumps. That should be an address in a PLT stub.
\item We analyse the instructions at the PLT stub. If the program was compiled with GCC, the first instruction will be an \textit{endbr64} instruction followed by the PLT jump instruction using the address at GOT (see figure \ref{fig:plt_gcc}), since it generates Intel CET-compatible programs. Otherwise, if using Clang, which does not generate Intel CET instructions, the first instruction is the PLT jump (see figure \ref{fig:plt_clang}).

We analyse the jump instruction and, again, take the address at which it jumps. This time, it should be the address of the function at glibc.
\begin{figure}[htbp]
	\centering
	\includegraphics[width=14cm]{sch_plt_gcc.png}
	\caption{PLT stub generated with gcc compiler, using objdump.}
	\label{fig:plt_gcc}
\end{figure}
\begin{figure}[htbp]
	\centering
	\includegraphics[width=14cm]{sch_plt_clang.png}
	\caption{PLT stub generated with clang compiler, using objdump.}
	\label{fig:plt_clang}
\end{figure}

\item We now have the address of timerfd\_settime at glibc, from where the syscall will be called. From eBPF, we continue to scan the first opcodes and compare them to those we expect to find at glibc. Specifically, the function would have to contain the instruction opcodes shown in figure \ref{fig:settime_glibc}. Note that, in our version of Ubuntu, we will find Glibc compiled with GCC.

\begin{figure}[htbp]
	\centering
	\includegraphics[width=14cm]{sch_settime_glibc.png}
	\caption{Timerfd\_settime function at glibc, using objdump.}
	\label{fig:settime_glibc}
\end{figure}

\end{enumerate}

Once we ensured we reached the correct glibc function, we are now sure that the data we found at the stack is the return address of the PLT stub that jumped to glibc and called the syscall sys\_timerfd\_settime. Most importantly, we know the address of the GOT section which we want to overwrite.

Our rootkit also incorporates an alternative scanning technique for processes calling the syscall sys\_openat(). This technique enables to scan the stack even when the system call does not incorporate any arguments from the userspace (and thus we cannot take them from our eBPF tracing program to use them as a foothold in the stack).

As we explained in section \ref{subsection:tracing_arguments}, tracepoint programs receive an struct pt\_regs pointer as an argument. We can take this struct and use the value of register rbp as our starting point for scanning the stack. As we can see on figures \ref{fig:plt_clang}, \ref{fig:plt_gcc} and \ref{fig:settime_glibc}, the  PLT does not contain any function prologue (it does not modify the value of rsp) and the function at glibc does not change this value either. Therefore, in our eBPF program, since we are hooking the syscall at the beginning of its execution, the value of rbp will be the original frame pointer before calling the PLT, and therefore we can use it as our starting address for stack scan, proceeding to scan forward until we find the saved return address.

\textbf{Stage 2: Programming shellcode}\\
Once that we have the address of the GOT section, we need to prepare our shellcode to be injected into the process memory. We will overwrite the value at GOT and redirect the flow of execution to the address at which our shellcode is stored in memory. 

Since we want our shellcode to be able to load a library, it will need to call the function \_\_libc\_dlopen\_mode, which can be found in glibc. This function expects to receive as an argument a string with the file path of the malicious library, and therefore the shellcode will also need to call \_\_libc\_malloc to allocate space for the argument. Tables \ref{table:libc_malloc} and \ref{table:libc_dlopen_mode} explain the expected arguments and return value of each function in detail.

\begin{table}[htbp]
\begin{tabular}{|>{\centering\arraybackslash}p{4cm}|>{\centering\arraybackslash}p{10cm}|}
\hline
Register & Value\\
\hline
\hline
edi & Number of bytes to allocate. \\
\hline
rax & Return value, contains the address at which the requested bytes were allocated\\
\hline
\end{tabular}
\caption{Arguments and return value of function \_\_libc\_malloc.}
\label{table:libc_malloc}
\end{table}

\begin{table}[htbp]
\begin{tabular}{|>{\centering\arraybackslash}p{4cm}|>{\centering\arraybackslash}p{10cm}|}
\hline
Register & Value\\
\hline
\hline
rsi & 0x1, indicating flag RTLD\_LAZY\\
\hline
rdi & Address where to read path of library to load\\
\hline
\end{tabular}
\caption{Arguments of function \_\_libc\_dlopen\_mode.}
\label{table:libc_dlopen_mode}
\end{table}

The programs were compiled having ASLR active, and therefore we cannot know the virtual address at which these functions are loaded into the process memory. However, since we have leaked the address of timerfd\_settime at glibc with the previous eBPF scan, we can calculate the address of the other functions, as we introduced in section \ref{subsection:hardening_bypass}. Figure \ref{fig:aslr_bypass_example} shows an example of this process.

\begin{figure}[htbp]
	\centering
	\includegraphics[width=10cm]{aslr_bypass_example.png}
	\caption{Functions at glibc with ASLR active.}
	\label{fig:aslr_bypass_example}
\end{figure}

We will use the example of the figure to illustrate how to calculate the address of the functions:
\begin{enumerate}
\item Decompile using objdump the glibc diagram and calculate the constant offset between the timerfd\_settime function (whose address we will know at runtime) and a reference function usually found in the first addresses of glibc, in this case \_\_libc\_start\_main (this step can be avoided, but it is recommended when searching for many functions and to avoid working with negative offsets). In the example, this offset is 0x30000.
\item Calculate the offset from the reference function \_\_libc\_start\_main to \_\_libc\_dlopen\_mode and \_\_libc\_malloc. In the example, this is 0x20000 and 0x5000 respectively by looking at  decompiled glibc.
\item During runtime, although the ASLR offset will be applied, it will skew all functions inside glibc by the same amount, and therefore the offsets previously calculated will be maintained. By using the previously, calculated offsets, we get that:
\begin{itemize}
	\item \_\_libc\_start\_main = timerfd\_settime - 0x30000
	\item \_\_libc\_dlopen\_mode = \_\_libc\_start\_main + 0x50000
	\item \_\_libc\_malloc = \_\_libc\_start\_main + 0x20000
\end{itemize}
\end{enumerate}

Once we know the address of the functions we want our shellcode to call, we can start to develop it. We will program an x86\_64 assembly program, from which we will extract its opcodes. The shellcode will follow the next algorithm:
\begin{enumerate}
\item Backup the value of all registers, including rbp and rsp. We must ensure that the stack frame is not modified after the shellcode ends, otherwise we may trigger a stack canary alert.
\item Allocate memory for the pathname of the library at the heap using \_\_libc\_malloc.
\item Write into the allocated memory the pathname of our library to load.
\item Call \_\_libc\_dlopen\_mode indicating the allocated memory with the library pathname. Before doing this, we found that reserving an additional stack frame reduces the chances of the process crashing, since apparently the function modifies the stack. By moving rbp and rsp, we prevent the function from modifying any pre-existing data.
\item Restore the original value of the registers, and jump back to the original system call which the glibc function intended to call.
\end{enumerate}

The complete developed shellcode and its opcodes can be found in Appendix \ref{annex:shellcode}.


\textbf{Stage 3: Injecting shellcode in a code cave}\\
Once we have developed our shellcode, and before overwriting the value of GOT, we need to find a memory section where to write our shellcode, so that we can executing the necessary instructions to inject our malicious library. This area must be large enough to fit our shellcode, and it must be marked as executable. 

Because of DEP/NX, we cannot use the stack for executing code. On top of that, as we can observe in the section header dump at Appendix \ref{annexsec:readelf_sec_headers}, for security reasons all sections are nowadays marked either writeable or executable, but never both simultaneously.

Therefore, we will use the proc filesystem which we introduced in section \ref{section:proc_filesystem}. By using the file under \textit{/proc/<pid>/maps}, we will easily identify the address range of those memory sections marked as executable, and by using the file \textit{/proc/<pid>/mem}, we will write our shellcode into that memory section, bypassing the absence of a write flag.

Although we may write freely into any virtual address using this technique, as we saw in section \ref{subsection:proc_maps} executable memory usually corresponds to the .text section. Therefore, we are at risk of overwriting critical instructions of the program. This is the reason why we must search for empty memory spaces inside the virtual memory, called code caves.

We will consider an appropiate code cave as a continuous memory space inside the .text section that consists of a series of NULL bytes (opcode 0x00). Although in principle this may seem like a rare occurence, it is a common find in most processes due to how memory access control is implemented.

In figure \ref{fig:proc_maps_sample}, we can observe how virtual memory sections have a length of 0x1000, or are a multiple of it. This is not an arbitrary number, but rather it is because memory sections must always be of length multiple of the system page length (4 KB = 0x1000 bytes). Therefore, the minimum granularity of a set of permissions over a memory section is of 0x1000 bytes.

Since sections must occupy a multiple of 1000 bytes, this leads to multiple sections which leave lots of empty, NULL bytes, unocuppied without any instructions. This is the reason why we will, quite probably, find a code cave in most processes.

Therefore the steps to find a code cave and inject our shellcode are the following:
\begin{itemize}
\item Send a command from eBPF to the rootkit user space program, indicating that we want to find a code cave in process with an specific PID.
\item Iterate over each entry of \textit{/proc/<pid>/maps}, looking for a sufficiently large code cave in an executable memory section.
\item Inject the shellcode into the code cave using \textit{/proc/<pid>/mem}.
\end{itemize}

Note that, although we used the \textit{/proc/<pid>/maps} file for finding a code cave, this can still be done using the helper bpf\_probe\_read (by taking the return address at the stack and scanning forward in the .text section) or, in the case of programs compiled without PIE, finding an static code cave at the .text section by decompiling the program (since the .text section will be loaded at the same position on every program execution). Still, we would have needed to use \textit{/proc/<pid>/mem} for bypassing the write access prevention.

\textbf{Stage 4: Overwriting GOT}\\
Once the shellcode is loaded at the code cave, eBPF can proceed to overwrite the GOT value with the address of the code cave. As we mentioned, this address is writable using the helper bpf\_probe\_write\_user() if the program was compiled using Partial RELRO, but it cannot be modified if Full RELRO was used. 

Therefore, our rootkit will modify GOT using bpf\_probe\_write\_user() with the address of an static code cave for those programs compiled with Clang (Partial RELRO, no PIE), and use \textit{/proc/<pid>/mem} for modifying GOT with the value of code cave found using \textit{/proc/<pid>/maps} for those programs compiled using GCC (Full RELRO, PIE active).

\textbf{Stage 5: Second syscall, execution of the library}\\
Once we have overwriten GOT with the address of our code cave, the next time the same syscall is called, the PLT stub will jump to our code cave and execute our shellcode. As instructed by it, the malicious library will be loaded and afterwards the flow of execution jumps back to the original glibc function.

%Explain reverse shell?
With respect to the malicious library, it forks the process (to keep the malicious execution in the background) and spawns a simple reverse shell which the attacker can use to execute remote commands.


\section{Privilege escalation module}
In this section we will discuss how the rootkit tampers with the access control permissions in the system, so that unprivileged programs gain root access. Although it is based on a simple technique, it will be used to support other modules launching malicious programs with full privilege (such as the execution hijacking module).

Therefore, the purpose of this section is that, without having to introduce any password, programs executed by an unprivileged user can enjoy privileged access in a infected system.

\subsection{Sudoers file}
Sometimes, unprivileged users need to run a program requiring privileged access. For this, Linux systems incorporate the sudo security policy module, which sets a 'sudo' privilege on users and user groups, allowing them to run a program as root. 

The most widespread and default sudo security policy module is the 'sudoers' policy module, which sets the available sudo permissions of users and groups in the \textit{/etc/sudoers} file \cite{sudoers_man}. In this file, the system administrator can determine the specific permissions of each entity and set different options, including whether they need to introduce the user password when using the 'sudo' command, which is particularly relevant for us. Figure \ref{fig:sudoers} shows the \textit{/etc/sudoers} file of the host we will infect with our rootkit.

\begin{figure}[htbp]
	\centering
	\includegraphics[width=10cm]{sch_sudoers.png}
	\caption{/etc/sudoers file of infected host.}
	\label{fig:sudoers}
\end{figure}

As we can observe in the figure, members of the sudo group are allowed to execute any command as root. Figure \ref{fig:groupfile} shows the users which belong to this group.

\begin{figure}[htbp]
	\centering
	\includegraphics[width=10cm]{sch_groupfile.png}
	\caption{/etc/group file in the infected host.}
	\label{fig:groupfile}
\end{figure}

As we can appreciate, the user osboxes (the default user in the host) is included in this group, and therefore this user is allowed to use sudo and run commands as root.

Any user can check its current sudo privileges by running the command \lstinline{sudo -l} \lstinline{}. Figure \ref{fig:sudol} shows this for the osboxes user.

\begin{figure}[htbp]
	\centering
	\includegraphics[width=10cm]{sch_sudol.png}
	\caption{Sudo privileges of user osboxes, with sudo -l.}
	\label{fig:sudol}
\end{figure}

The value of these entries is taken from the parameters set in figure \ref{fig:sudoers}, where each of the ALL values mean:
\begin{itemize}
\item First ALL: Any user of the group
\item Second ALL: Any host
\item Third ALL: As any user
\item Fourth ALL: Any command
\end{itemize}

Therefore, user osboxes, as part of the sudo group, may run any command as any user in any host as sudo. The host part is not relevant for our us, since it is used when a single sudoers file is distributed betweem multiple machines, but we still have to follow the appropiate format when writing an entry in the \textit{/etc/sudoers} file.

Each time we execute a command with sudo, a process named 'sudo' will open and read the \textit{/etc/sudoers} file, interpreting the contents and allowing or rejecting the action. Note that, although once an user introduces the sudo password it may not be asked again for a period of time, the sudo process will still open and read the \textit{/etc/sudoers} file for each time sudo is used. This aspect is particularly relevant for our technique.


\subsection{Hijacking sudoers read accesses}
We will now discuss how our rootkit tampers with the sudoers policy module. The technique we will present is based on modifying the content that the sudo process reads from the \textit{/etc/sudoers} file, so that what the user process receives is different than that contained in the file. By crafting some special entries in the file, we can grant automatic password-less access to any process we want.

In order to read the contents from the \textit{/etc/sudoers} file, the sudo process will need to perform the following actions:
\begin{itemize}
\item Open the file, using the syscall sys\_openat.
\item Read the file, using the syscall sys\_read.
\end{itemize}

Note that some intermediate or additional syscalls such as sys\_newfstatat, sys\_lseek or sys\_close are also called, but we are not considering them for simplicity.

Table \ref{table:sudoers_syscall} shows the parameters expected by these system calls, based on \cite{syscall_reference}.

\begin{table}[htbp]
\begin{tabular}{|c|>{\centering\arraybackslash}p{8cm}|}
\hline
System call & Arguments\\
\hline
\hline
\multirow{4}{*}{sys\_openat} & \multicolumn{1}{c|}{int dfd}\\
\cline{2-2}
& \multicolumn{1}{c|}{const char \_\_user *filename}\\
\cline{2-2}
& \multicolumn{1}{c|}{inf flags} \\
\cline{2-2}
& \multicolumn{1}{c|}{umode\_t umode} \\
\hline
\multirow{3}{*}{sys\_read} & \multicolumn{1}{c|}{unsigned int fd}\\
\cline{2-2}
& \multicolumn{1}{c|}{char \_\_user *buf} \\
\cline{2-2}
& \multicolumn{1}{c|}{size\_t count} \\
\hline
\end{tabular}
\caption{Arguments of syscalls used by sudo process.}
\label{table:sudoers_syscall}
\end{table}

The table shows that there exist two arguments marked as \textit{\_\_user}, which, as we explained in section \ref{subsection:bpf_probe_write_apps}, can be overwritten from an eBPF tracing program using the helper bpf\_probe\_write\_user(). Therefore, there exist two different attack vectors:
\begin{itemize}
\item Modify the argument \textit{filename}, so that the sudo process opens a fake, crafted sudoers file. In this file we would write the entries needed for our user to have sudo privilege without a password. Since the sys\_open syscall returns a file descriptor, which is later used by sys\_read, that is the only argument needed to be modified.
\item Modify the buffer \textit{buf} in the sys\_read syscall so that it returns specially crafted data to the sudo program.
\end{itemize}

Although the first option is easier, the second technique can not only apply to reading files, but also to any system calls that loads data into an user buffer. Therefore, the privilege escalation module will incorporate the second technique to show the potential of eBPF in this area.

Figure \ref{fig:privilege_esc_module} shows the complete process of the technique we will use.
\begin{figure}[htbp]
	\centering
	\includegraphics[width=15cm]{privilege_esc_module.png}
	\caption{Buffer overwrite technique for the privilege escalation module.}
	\label{fig:privilege_esc_module}
\end{figure}

As we can observe in the figure, we will use three eBPF tracepoints. The reason for this is that, although we are able to write into the user buffer at any tracepoint attached to sys\_read, we would lack information with only one tracepoint:
\begin{itemize}
\item An \textit{enter} tracepoint at sys\_openat knows the file being opened, but it does not have access to the user buffer.
\item An \textit{enter} tracepoint at sys\_read has access to the user buffer, but does not know the name of the file (it only has a file descriptor). Also, if it writes into the buffer now, it will be overwritten later when the kernel reads the \textit{/etc/sudoers} file.
\item An \textit{exit} tracepoint at sys\_read only receives the return value as a parameter (as we explained in section \ref{subsection:tracing_arguments}), but it can freely write to the user buffer if it had access to it, since the kernel already finished writing on it.
\end{itemize}

Taking the above into account, we designed the privilege escalation technique as follows:
\begin{enumerate}
\item We load and attach three eBPF tracepoint programs, and an eBPF map:
\begin{itemize}
	\item An \textit{enter} tracepoint attached to sys\_openat (sys\_enter\_openat).
	\item An \textit{enter} tracepoint attached to sys\_read (sys\_enter\_read).
	\item An \textit{exit} tracepoint attached to sys\_read (sys\_exit\_read).
	\item An eBPF map (fs\_open) that stores fs\_open\_data structs, composed of:
	\begin{itemize}
	\item A process name.
	\item A filename.
	\end{itemize}
	The key of the map fs\_open is the PID of the user process from which the call to an eBPF program originated, this can be obtained using the bpf\_get\_current\_pid\_tgid() helper (see section \ref{subsection:ebpf_helpers}).
\end{itemize}
\item A malicious program we executed from user "osboxes" requests sudo privileges. Our goal is to let it run with privileged permissions without having to introduce a password. Note that, although in the system we are using osboxes is an user in the \textit{/etc/sudoers} file already (although requiring a password for running as sudo), this process also works if we used an user not included on it in the first place.

The sudo process opens the \textit{/etc/sudoers} file. The syscall is called and the sys\_enter\_openat tracepoint is called before the syscall is executed. We check that the syscall was called by the sudo process using the helper bpf\_get\_current\_comm() (see section \ref{subsection:ebpf_helpers}) and, if it is, write the filename into the fs\_open map. After that, the tracepoint exists and the syscall is executed.

\item The sudo process now reads from the file descriptor of the file \textit{/etc/sudoers}. The sys\_enter\_read tracepoint is executed right before the syscall is called. In the tracepoint, we check if we can find an entry with a filename in the fs\_open map using the process PID as key (which is the same for all tracepoints, since they originated from the same sudo process). We now write address of the buffer supplied by the sudo process into the map.

\item The sys\_read syscall is executed and, when it is about to exit, our tracepoint sys\_exit\_read is executed. We take the filename and the address of the user buffer from the fs\_open map, and overwrite the data at the user buffer which contained the bytes read from \textit{/etc/sudoers} using bpf\_probe\_write\_user(). The data we will write resembles a real entry of the \textit{/etc/sudoers} file:
\begin{verbatim}
osboxes ALL=(ALL:ALL) NOPASSWD:ALL #
\end{verbatim}

Injecting that string into the read file will grant us with password-less sudo privileges. There are two particularly relevant details on it:
\begin{itemize}
\item The NOPASSWD option instructs sudo not to request a password.
\item A \# symbol is included at the end so that any data not overwritten at that line is considered a comment (see figure \ref{fig:sudoers}).
\end{itemize}

Although the previous is sufficient for tricking the sudo process into believing we have sudo privileges, it can happen that an user (in this case, osboxes) already has an entry in the \textit{/etc/sudoers} file. When this happens, the sudo process usually chooses the last entry that appears on the file or fails. 

Although not the most elegant solution, the solution for this issue incorporated in our rootkit is that the tracepoint program will continue writing \# symbols until an error happens (thus indicating we reached the end of the file).

\end{enumerate}


\section{Execution hijacking module}
This section describes how the rootkit can hijack the execution of programs. Although in principle eBPF in the kernel cannot start the execution of a program by itself, this module shows how a malicious rootkit may take advantage of benign programs in order to execute malicious code in the user space. Therefore, we aim to achieve two main goals:
\begin{itemize}
\item Execute a malicious user program taking advantage of other program's execution.
\item Be transparent to the user space, that is, if we hijack the execution of a program so that another is run, the original program should be executed too with the least delay.
\end{itemize}

This technique is based on the modification of the arguments of the system call sys\_execve, used to execute programs. When it is called, it causes the program that is currently being run to be completely replaced by the new executed program \cite{execve_man}. Its arguments are listed in table \ref{table:execve_args}

\begin{table}[htbp]
\begin{tabular}{|c|>{\centering\arraybackslash}p{7cm}|}
\hline
Argument & Description\\
\hline
\hline
const char \_\_user *filename & Path and filename of the file to execute\\
\hline
const char \_\_user *const \_\_user *argv & NULL-terminated array with arguments passed to the program\\
\hline
const char \_\_user *const \_\_user *envp & NULL-terminated array with the environment variables associated to the executed program \cite{environ}\\
\hline
\end{tabular}
\caption{Arguments of system call sys\_execve.}
\label{table:execve_args}
\end{table}

As we can observe in the table, all of the arguments of the syscall are marked with the keyword \_\_user, and therefore as we explain in section \ref{subsection:bpf_probe_write_apps} these arguments can be overwritten using the eBPF helper bpf\_probe\_write\_user(). This opens for us the possibility of modifying these arguments so that another file is modified.

Figure \ref{fig:summ_execve_hijack} summarizes the results of an attack using this rootkit module. As we can observe in the figure, we will hijack the execution of sys\_execve to run our own program, but as we mentioned we must execute the original program too in order not to raise concerns in the user space. Therefore, the malicious program must be able to access the original arguments of the sys\_execve call to execute the original program.

\begin{figure}[htbp]
	\centering
	\includegraphics[width=14cm]{summ_execve_hijack.png}
	\caption{Overview of execution hijacking attack.}
	\label{fig:summ_execve_hijack}
\end{figure}

As we will discuss, apart from running the original program, the malicious program will run itself as sudo (taking advantage of the privilege escalation module) and then connecting to the rootkit client.


\subsection{Overwriting sys\_execve}
We have mentioned the possibility of overwriting the parameters of the sys\_execve syscall. However, after loading an eBPF \textit{enter} tracepoint attached to sys\_execve and writing into any of this buffers, we found three scenarios:
\begin{itemize}
\item The helper successfully overwrites the user buffers.
\item The helper fails to overwrite all or some of the buffers.
\item The helper successfully overwrites a buffer but, with a single write operation, it has also modified the value of some other user buffer.
\end{itemize}

The reason for this is that, as we covered in section \ref{subsection:bpf_probe_write_apps}, the bpf\_probe\_write\_user() helper fails to write any data in the occurence of a page fault. As we explained in section \ref{subsection:mem_faults}, minor memory faults are particularly common when executing a fork() of a process, since the child process will not get its page table completely copied from the parent, but will request the mapping once it is attempted to be read.

Because of the fact that programs calling sys\_execve will be completely replaced by the new program, we can find this function used commonly in two contexts:
\begin{itemize}
\item User programs which execute a new program as a child, but they do not want to be terminated themselves. For this, they call a fork() and then execute execve() (which calls the sys\_execve syscall) in the child process.
\item Programs that are run by the user in the command-line interface. Once a command is introduced, the program corresponding to the command is searched, and the bash process (or any other shell being used) will fork() itself and execute the new program.
\end{itemize}

Therefore, when modifying the arguments of sys\_execve, we will find that most calls are from programs which had executed fork() previously, thus having a high probability of failing. Note that the exact reason why writing one buffer with bpf\_probe\_write\_user() modifies multiple buffers simultaneouslly is unknown, but it is a situation we must account for, since we cannot trust in the helper not returning an error, we must check the result of this write accesses.

\subsection{Hiding data in a system call}
Apart from having to take into account that the bpf\_probe\_write\_user helper may fail in unexpected manners as we described, we also need to give special attention to how we will preserve the original information of the program being executed via sys\_execve after we modify the arguments of this call. As we showed in figure \ref{fig:summ_execve_hijack}, the malicious program executed using the hijacked syscall must be able to execute the original program. For this, the program will fork() and create a child process, on which execve() will be called with the original program arguments. Therefore, the main issue would be how to recover the original arguments once they were overwritten by eBPF.

In order to achieve this, we will hide the original arguments in those passed to the malicious program. Table \ref{fig:execve_args_hide} shows how this process works with a sample sys\_execve call. Environment variables have been omitted for simpleness, but we can usually find a large array of them.

\begin{table}[H]
\begin{tabular}{|>{\centering\arraybackslash}p{2cm}|>{\centering\arraybackslash}p{3cm}|}
\hline
\multicolumn{2}{|c|}{Original arguments}\\
\hline
\hline
filename & "/bin/ls"\\
\hline
argv[0] & "ls"\\
\hline
argv[1] & "-l"\\
\hline
argv[2] & NULL\\
\hline
envp[0] & NULL\\
\hline
\end{tabular}
\quad
\begin{tabular}{|>{\centering\arraybackslash}p{2cm}|>{\centering\arraybackslash}p{3cm}|}
\hline
\multicolumn{2}{|c|}{Modified arguments}\\
\hline
\hline
filename & "/home/osboxes/execve\_hijack"\\
\hline
argv[0] & "/bin/ls"\\
\hline
argv[1] & "-l"\\
\hline
argv[2] & NULL\\
\hline
envp[0] & NULL\\
\hline
\end{tabular}
\caption{Hiding data in sys\_execve arguments.}
\label{table:execve_args_hide}
\end{table}

As we can observe in the table, we will modify the value of \tetxit{filename} with the malicious program filename, and save the original filename into argv[0]. Performing this substitution means losing little information since the argv[0] argument contains the name of the program \cite{c_standard_main}, information that can also be taken from the filename (thus it can be recovered later). Only in very specific use cases the argv[0] argument is different from the file included in the filename argument (like in Busybox \cite{busybox_argv}). 

After the above substitution, the malicious program (in the table, "execve\_hijack") will be called, whose main function receives the following arguments:

\begin{verbatim}
int main (int argc, char *argv[], char *envp[]){}
\end{verbatim}

Hence, the malicious program will use the argv[] and envp[] arrays to make another sys\_execve call with the original arguments, running the original program. 

\subsection{Hijacking a program execution}
Once we have analysed the two fundamental issues regarding this module (bpf\_probe\_write\_user fails and hiding information in the syscall arguments) we will now analyze the execution hijacking module in detail using a sample program execution.

Figure










\chapter{Evaluation} \label{chapter:evaluation}
This chapter evaluates the malicious capabilities developed in our rootkit by comparing them to the original objectives we presented at the beginning of our research in Section \ref{section:project_objectives}. For this, we will analyse whether our rootkit meets the expected functionality by simulating a machine infection in a virtualized environment. A rootkit functionality will be considered fulfilled in the case it can be reproduced successfully in the experimental environment.

As we mentioned, the following are the functionalities we seeked to implement in our rootkit:
\begin{itemize}
\item Hijacking the execution of user programs while they are running, injecting libraries and executing malicious code, without impacting their normal execution.
\item Featuring a command-and-control module powered by a network backdoor, which can be operated from a remote client. This backdoor should be controlled with stealth in mind, featuring similar mechanisms to those present in rootkits found in the wild.
\item Tampering with user data at system calls, resulting in running malware-like programs and for other malicious purposes.
\item Achieving stealth, hiding rootkit-related files from the user.
\item Achieving rootkit persistence, the rootkit should run after a complete system reboot.
\end{itemize}

\section{Experimental setting}
The test environment that will be used to showcase the rootkit functionalities consists on two virtual machines running under Oracle VM VirtualBox \cite{virtualbox_page}. One of them will be the host infected with the rootkit, while the other will be used as the attacker machine from which to operate the rootkit client. 

Both virtual machines will be connected via a bridged adapter, as Figure \ref{fig:vm_setting_bridged} shows. With this virtual networking setting, the virtual machines connect to a device driver of the host system which injects the data received from the physical network \cite{bridged_networking}. From the virtual machine point of view, both the attacker and the infected machine appear to be physically connected (via a network cable) to the same network interface, each with a different assigned IP address. The name of this interface will be "enp0s3".

\begin{figure}[htbp]
	\centering
	\includegraphics[width=13cm]{vm_setting_bridged.png}
	\caption{Network settings for both of the VMs on the test environment.}
	\label{fig:vm_setting_bridged}
\end{figure}

Table \ref{table:vm_config_test_environment} shows the role and charactersitics of the two machines. The overall test environment configuration with the described settings is illustrated in Figure \ref{fig:test_env}.

\begin{table}[H]
\begin{tabular}{|>{\centering\arraybackslash}p{3cm}|>{\centering\arraybackslash}p{3cm}|}
\hline
\multicolumn{2}{|c|}{\textbf{INFECTED MACHINE}}\\
\hline
\textbf{Attribute} & \textbf{Value}\\
\hline
\hline
User & osboxes\\
\hline
Operating System & GNU/Linux\\
\hline
Distribution & Ubuntu 21.04\\
\hline
Kernel version & 5.11.0-49\\
\hline
IP address & 192.168.1.124\\
\hline
\end{tabular}
\quad
\begin{tabular}{|>{\centering\arraybackslash}p{3cm}|>{\centering\arraybackslash}p{3cm}|}
\hline
\multicolumn{2}{|c|}{\textbf{ATTACKER MACHINE}}\\
\hline
\textbf{Attribute} & \textbf{Value}\\
\hline
\hline
User & RED\\
\hline
Operating System & GNU/Linux\\
\hline
Distribution & Ubuntu 18.04\\
\hline
Kernel version & 5.4.0-96\\
\hline
IP address & 192.168.1.127\\
\hline
\end{tabular}
\caption{Configuration of virtual machines in the test environment.}
\label{table:vm_config_test_environment}
\end{table}

\begin{figure}[htbp]
	\centering
	\includegraphics[width=13cm]{test_env.png}
	\caption{Network topology of test environment.}
	\label{fig:test_env}
\end{figure}



\section{Rootkit compilation and installation} \label{section:compile_install}
This section details the steps for a successful compilation and installation fo the rootkit in the target machine. Note that there also exist two scripts \textit{packager.sh} and \textit{deployer.sh} which automatize this process, but these are best indicated for an attacker which wants to quickly assemble the rootkit system, as we will explain in Section \ref{section:attack_scenario}.

\subsection{Compilation}
The rootkit source code incorporates two Makefile files that automatize the compilation process with the command \textit{make}. Table \ref{table:makefiles} details the location of the multiple Makefiles that must be executed to compile the different modules of the rootkit (note that in Section \ref{section:rootkit_arch} we described the rootkit files and their purpose in detail).

\begin{table}[htbp]
\begin{tabular}{|>{\centering\arraybackslash}p{2.2cm}|>{\centering\arraybackslash}p{2.2cm}|>{\centering\arraybackslash}p{4cm}|>{\centering\arraybackslash}p{4.5cm}|}
\hline
\textbf{MAKEFILE} & \textbf{COMMAND} &\textbf{DESCRIPTION}&\textbf{RESULT}\\
\hline
\hline
src/client/ Makefile & make & Compilation of the rootkit client & src/client/ injector\\
\hline
src/Makefile & make help & Compilation of programs for testing rootkit functionalities, and the malicious program and library of the execution hijacking and library injection modules respectively & src/helpers/simple\_timer, src/helpers/simple\_open, src/helpers/simple\_execve, src/helpers/lib\_injection.so, src/helpers/execve\_hijack\\
\hline
src/Makefile & make kit & Compilation of the rootkit using the libbpf library & src/bin/kit\\
\hline
src/Makefile & make tckit & Compilation of the rootkit TC egress program & src/bin/tc.o\\
\hline
\end{tabular}
\caption{Rootkit compilation Makefiles.}
\label{table:makefiles}
\end{table}

As we can observe in the table, there are two Makefiles:
\begin{itemize}
\item A Makefile under src/client to compile only the rootkit client.
\item A Makefile under src to compile all rootkit files.
\end{itemize}

Therefore, the complete compilation process would consist on the commands shown in Code snippet \ref{code:compilation}.
\begin{lstlisting}[language=C, caption={Rootkit and rootkit client compilation.}, label={code:compilation}]
//Rootkit files
cd src
make
//Rootkit client
cd client
make
\end{lstlisting}

The output programs corresponding to the rootkit will be stored under a directory \textit{src/bin}, while those belonging to helper and client programs will be stored together with the corresponding source code.

It must also be noted that, although the rootkit backdoor and C2 capabilities work out of the box, the rest of the rootkit modules need further configuration. This configuration is set via the src/common/constants.h file, and during the rest of this evaluation we will detail the relevant settings for each individual module.

\subsection{Installation}
Once the rootkit programs are compiled, the \textit{tc.o} and \textit{kit} programs must be loaded orderly. Code snippet \ref{code:installation} shows the commands to execute for installing the rootkit.

\begin{lstlisting}[language=C, caption={Rootkit installation steps.}, label={code:installation}]
//TC egress program
sudo tc qdisc add dev enp0s3 clsact
sudo tc filter add dev enp0s3 egress bpf direct-action obj bin/tc.o sec classifier/egress
//Libbpf-powered rootkit
sudo ./bin/kit -t enp0s3
\end{lstlisting}

Note that the network interface enp0s3 may be substituted with any other interface on which the attacker desires the backdoor to be operating.

Finally, we should create the files that guarantee the rootkit persistence, as shown in Code snippet \ref{code:persist}.
\begin{lstlisting}[language=C, caption={Creation of rootkit persistence files.}, label={code:persist}]
echo "* * * * * osboxes /bin/sudo /home/osboxes/TFG/src/helpers/deployer.sh" > /etc/cron.d/ebpfbackdoor
echo "osboxes ALL=(ALL:ALL) NOPASSWD:ALL #" > /etc/sudoers.d/ebpfbackdoor
\end{lstlisting}

The name of the user "osboxes" should be changed by that of the user of the machine to infect, together with the path on which the \textit{deployer.sh} script will be hidden.


\section{Attack scenario} \label{section:attack_scenario}
Although the steps presented in Section \ref{section:compile_install} were followed during the rootkit development, an attacker which has compromised a machine and wants to install the rootkit may benefit from a more automated process that quickly prepares all files and installs them in the target machine.

This section presents an hypothetical attack scenario, covering each of the steps the attacker must follow in order to prepare the rootkit and infect a machine:

A security researcher called 'RED' has managed to exploit a high-severity RCE vulnerability in a critical system controlled by an adversary which was found exposed to the Internet (e.g.: not behind a NAT \cite{nat_comptia}). After this exploitation, RED has now spawned a reverse shell connection with the privileged user 'osboxes', but he knows that the system is often rebooted and that he may lose access soon. Furthermore, the vulnerability he exploited is already well-known and may get patched in the near future, so he needs to persist his access. RED decides to load a classic rootkit consisting of a malicious kernel module, but he finds out that this capability is restricted in the system (e.g.: kernel.modules\_disabled=1 \cite{kernel_modules_restrict}), so he must find an alternative approach. Also, it is very possible that the system has an EDR logging events such as loading a kernel module (which almost assuredly will be considered by the EDR given that it is a very relevant event), so he needed to find a more stealthy path anyway. At some point, RED realises that even if kernel modules could not be used, the system administrator did not block eBPF, so he decides to use TripleCross.

Firstly, RED creates a secret directory where to hide the rootkit, and downloads it, as shown in Figure \ref{fig:post_exp}.

\begin{figure}[htbp]
	\centering
	\includegraphics[width=13cm]{sch_post_exp.png}
	\caption{Creation of hidden directory and downloading rootkit.}
	\label{fig:post_exp}
\end{figure}

Once it is downloaded, RED executes the \textit{packager.sh} script, that will compile the rootkit. Alternatively, an attacker could have compiled it locally and sent it to the remote machine afterwards.

After the script execution finishes, a folder \textit{apps} has been generated with all the rootkit files. This directory contains all the files and scripts needed for the rootkit installation. RED now executes the \textit{deployer.sh} script, which installs the rootkit and writes the persistence files, as shown in Figure \ref{fig:deploy_root}

\begin{figure}[htbp]
	\centering
	\includegraphics[width=13cm]{sch_deploy_root.png}
	\caption{Files created by packager.sh and execution of deployer.sh.}
	\label{fig:deploy_root}
\end{figure}

Once the script has been executed, all rootkit modules are loaded and the backdoor is already waiting for commands. RED can now close the reverse shell and open the rootkit client. He now has persistent privileged access to the infected machine.

\section{Hijacking execution of running processes}
Following the infection process described in Section \ref{section:attack_scenario}, The rootkit can hijack the execution of running processes by means of the library injection module. This module incorporates two sample programs (\textit{src/helpers/simple\_timer.c and src/helpers/simple\_open.c}), both containing the execution of one of the hijacked syscalls (sys\_timerfd\_settime and sys\_openat respectively). Additionally, the functionality can be tested in any process of the infected machine by changing its settings. Table \ref{table:lib_injection_config} shows how to customize the functionality of the library injection module.

\begin{table}[htbp]
\begin{tabular}{|>{\centering\arraybackslash}p{2.8cm}|>{\centering\arraybackslash}p{5.2cm}|>{\centering\arraybackslash}p{5.2cm}|}
\hline
\textbf{FILENAME} & \textbf{CONSTANT} & \textbf{DESCRIPTION}\\
\hline
\hline
src/common/ constants.h & TASK\_COMM\_NAME\_INJECTION\_TARGET\_TIMERFD\_SETTIME & Name of process to hijack at syscall sys\_timerfd\_settime.\\
\hline
src/common/ constants.h & TASK\_COMM\_NAME\_INJECTION\_TARGET\_OPEN & Name of process to hijack at syscall sys\_openat.\\
\hline
src/helpers/ injection\_lib.c & ATTACKER\_IP \& ATTACKER\_PORT & IP address and port of attacker machine\\
\hline
\end{tabular}
\caption{Library injection module configuration.}
\label{table:lib_injection_config}
\end{table}

After a successful injection the malicious library will run a reverse shell against the attacker machine. Also, it will print a message for us to check it locally. Therefore, from the attacker machine, we will listen to the specified IP and port, considering the injection successful if a connection is opened.


\subsection{Test program simple\_timer}
Table \ref{table:lib_injection_config_simple_timer} shows the module configuration for running this attack.

\begin{table}[htbp]
\begin{tabular}{|>{\centering\arraybackslash}p{3cm}|>{\centering\arraybackslash}p{5.5cm}|>{\centering\arraybackslash}p{4cm}|}
\hline
\textbf{FILENAME} & \textbf{CONSTANT} & \textbf{VALUE}\\
\hline
\hline
src/common/ constants.h & TASK\_COMM\_NAME\_INJECTION\_TARGET\_TIMERFD\_SETTIME & "simple\_timer"\\
\hline
src/helpers/ injection\_lib.c & ATTACKER\_IP \& ATTACKER\_PORT & 192.168.1.127 \& 5555 \\
\hline
\end{tabular}
\caption{Library injection module configuration for attacking simple\_timer.c.}
\label{table:lib_injection_config_simple_timer}
\end{table}

Figure \ref{fig:sc_lib_inj_simple_timer_off} shows the execution of the simple\_timer process without the rootkit installed.

\begin{figure}[htbp]
	\centering
	\includegraphics[width=13cm]{sch_sc_lib_inj_simple_timer_off.png}
	\caption{Normal execution of simple\_timer program.}
	\label{fig:sc_lib_inj_simple_timer_off}
\end{figure}

Once the rootkit is installed it starts the module automatically, looking for system calls from the simple\_timer process.
The attacker must in the mean time start a listener (e.g.: with netcat), as shown in Figure \ref{fig:sc_lib_inj_nc}.

\begin{figure}[htbp]
	\centering
	\includegraphics[width=12cm]{sch_sc_lib_inj_nc.png}
	\caption{Attacker waiting for a connection with netcat.}
	\label{fig:sc_lib_inj_nc}
\end{figure}

Then, the simple\_timer program gets executed on the infected machine. As we can observe in Figure \ref{fig:sc_lib_inj_simple_timer_exec}, the injection suceeds and a message is printed from the library. 

\begin{figure}[htbp]
	\centering
	\includegraphics[width=12cm]{sch_sc_lib_inj_simple_timer_exec.png}
	\caption{Execution of simple\_timer.c with rootkit active.}
	\label{fig:sc_lib_inj_simple_timer_exec}
\end{figure}


Figure \ref{fig:sc_lib_inj_nc_success} shows the attacker connected to the reverse shell launched from the library.

\begin{figure}[htbp]
	\centering
	\includegraphics[width=12cm]{sch_sc_lib_inj_nc_success.png}
	\caption{Reverse shell received after library injection attack.}
	\label{fig:sc_lib_inj_nc_success}
\end{figure}


\subsection{Test program simple\_open}
The library injection module can also be tested with the simple\_timer program, which opens multiple files with sys\_openat. The rootkit configuration for this is shown in Table \ref{table:lib_injection_config_simple_open}.

\begin{table}[htbp]
\begin{tabular}{|>{\centering\arraybackslash}p{3cm}|>{\centering\arraybackslash}p{5.2cm}|>{\centering\arraybackslash}p{4cm}|}
\hline
\textbf{FILENAME} & \textbf{CONSTANT} & \textbf{VALUE}\\
\hline
\hline
src/common/ constants.h & TASK\_COMM\_NAME\_INJECTION\_TARGET\_OPEN & "simple\_open"\\
\hline
src/helpers/ injection\_lib.c & ATTACKER\_IP \& ATTACKER\_PORT & 192.168.1.127 \& 5555 \\
\hline
\end{tabular}
\caption{Library injection module configuration for attacking simple\_open.c.}
\label{table:lib_injection_config_simple_open}
\end{table}

As we can observe in figure \ref{fig:sc_lib_inj_simple_open}, when the injection suceeds, a message is printed on screen. Also, the attacker receives a shell, like we showed in Figure \ref{fig:sc_lib_inj_nc_success}.

\begin{figure}[htbp]
	\centering
	\includegraphics[width=12cm]{sch_sc_lib_inj_simple_open.png}
	\caption{Execution of simple\_open with rootkit active.}
	\label{fig:sc_lib_inj_simple_open}
\end{figure}


\subsection{Hijacking systemd}
Apart from the test programs, the library injection module can also inject the malicious library on any process of the system that makes use of either sys\_openat or sys\_timerfd\_settime. By hijacking privileged system programs such as systemd, the malicious library can achieve automatic root permissions once it is run (although these are anyways automatically granted via the privilege escalation module). Table \ref{table:lib_injection_config_systemd} shows the module configuration for running an attack against this process.

\begin{table}[htbp]
\begin{tabular}{|>{\centering\arraybackslash}p{3cm}|>{\centering\arraybackslash}p{5.5cm}|>{\centering\arraybackslash}p{4cm}|}
\hline
\textbf{FILENAME} & \textbf{CONSTANT} & \textbf{VALUE}\\
\hline
\hline
src/common/ constants.h & TASK\_COMM\_NAME\_INJECTION\_TARGET\_TIMERFD\_SETTIME & "systemd"\\
\hline
src/common/ constants.h & TASK\_COMM\_NAME\_INJECTION\_TARGET\_OPEN & "systemd"\\
\hline
src/helpers/ injection\_lib.c & ATTACKER\_IP \& ATTACKER\_PORT & 192.168.1.127 \& 5555 \\
\hline
\end{tabular}
\caption{Library injection module configuration for attacking the systemd process.}
\label{table:lib_injection_config_systemd}
\end{table}

With these configurations, we can run the rootkit and wait for systemd to call one of these syscalls. Eventually this call occurs, and using the debug messages of the rootkit we can get information on what happened, as shown in Figure \ref{fig:sc_lib_inj_systemd_debug}.

\begin{figure}[htbp]
	\centering
	\includegraphics[width=12cm]{sch_sc_lib_inj_systemd_debug.png}
	\caption{Rootkit debug messages showing library injection.}
	\label{fig:sc_lib_inj_systemd_debug}
\end{figure}

As we can observe in the figure, the rootkit finds the relevant addresses via the technique we described on Section \ref{section:lib_injection} and proceeds to overwrite the GOT address. The library is loaded and executed, and since systemd is executed by the root user, the attacker receives a root shell as shown in Figure \ref{fig:lib_inj_success_root}. Most importantly, the systemd process does not crash after this attack.

\begin{figure}[htbp]
	\centering
	\includegraphics[width=12cm]{sch_lib_inj_success_root.png}
	\caption{Reverse shell received with root user after systemd library injection.}
	\label{fig:lib_inj_success_root}
\end{figure}


\section{Backdoor and C2}
The backdoor module works out of the box without any additional configurations needed. It includes the C2 capabilities and the rootkit client used to communicate with the backdoor. As we described in Section \ref{subsection:rootkit_manual}, the client allows for the operations listed on Table \ref{table:rootkit_client_actions}.

\begin{table}[htbp]
\begin{tabular}{|>{\centering\arraybackslash}p{5cm}|>{\centering\arraybackslash}p{8.5cm}|}
\hline
\textbf{PROGRAM ARGUMENTS} & \textbf{ACTION DESCRIPTION}\\
\hline
\hline
./injector -c <Victim IP> & Spawns a plaintext pseudo-shell by using the execution hijacking module.\\
\hline
./injector -e <Victim IP> & Spawns an encrypted pseudo-shell by commanding the backdoor with a pattern-based trigger.\\
\hline
./injector -s <Victim IP> & Spawns an encrypted pseudo-shell by commanding the backdoor with a multi-packet trigger (of both types).\\
\hline
./injector -p <Victim IP> & Spawns a phantom shell by commanding the backdoor with a pattern-based trigger.\\
\hline
./injector -a <Victim IP> & Orders the rootkit to activate all eBPF programs.\\
\hline
./injector -a <Victim IP> & Orders the rootkit to detach all of its eBPF programs.\\
\hline
./injector -S <Victim IP> & Showcases how the backdoor can hide a message from the kernel.\\
\hline
./injector -h & Displays help.\\
\hline
\end{tabular}
\caption{Rootkit client options.}
\label{table:rootkit_client_actions}
\end{table}

Once the rootkit is installed, the backdoor is launched automatically and will wait for backdoor triggers ready to launch the corresponding requested actions.

\subsection{Spawning encrypted pseudo-shells}
Encrypted pseudo-shells can be spawned using the rootkit client either with pattern-based or multi-packet backdoor triggers.

\textbf{Pattern-based triggers}\\
When using a pattern-based trigger, the attacker must indicate the following information:
\begin{itemize}
\item The IP address of the infected machine.
\item The network interface to use for sending the trigger.
\end{itemize}

As Figure \ref{fig:sc_eps_rc} shows, the backdoor executes the requested action and starts an encrypted pseudo-shell connection with privileged permissions in which the attacker can introduce commands to be executed. Whenever the connection shall be closed, the attacker introduces the "EXIT" global command (as we explained in Section \ref{subsection:rootkit_manual}), which ends the transmission gracefully.

\begin{figure}[htbp]
	\centering
	\includegraphics[width=12cm]{sch_sc_eps_rc.png}
	\caption{Encrypted pseudo-shell with rootkit client using pattern-based trigger.}
	\label{fig:sc_eps_rc}
\end{figure}

\textbf{Multi-packet triggers}\\
The rootkit client offers multiple options when using the multi-packet backdoor triggers. In particular, the attacker must specify the following fields:
\begin{itemize}
\item The IP address of the infected machine.
\item The network interface to use for sending the trigger.
\item Whether to hide the payload at the TCP sequence numbers or at the TCP source port.
\end{itemize}

Figure \ref{fig:sc_eps_seqnum} shows how the rootkit client asks for this data and spawns an encrypted pseudo-shell with the client when hiding the payload at the TCP sequence number. As we can observe in the figure, the payload is divided in 3 different chunks and injected to a stream of packets, which are sent in an orderly manner.

\begin{figure}[htbp]
	\centering
	\includegraphics[width=12cm]{sch_sc_eps_seqnum.png}
	\caption{Encrypted pseudo-shell with rootkit client using multi-packet trigger with payload hidden in TCP sequence number.}
	\label{fig:sc_eps_seqnum}
\end{figure}

Figure \ref{fig:sc_eps_srcport} shows the same process but using the TCP source port as a means for hiding the data payload. As we can observe in the figure, in this case the paylaod is divided in 6 chunks.

\begin{figure}[htbp]
	\centering
	\includegraphics[width=12cm]{sch_sc_eps_srcport.png}
	\caption{Encrypted pseudo-shell with rootkit client using multi-packet trigger with payload hidden in TCP source port.}
	\label{fig:sc_eps_srcport}
\end{figure}

\subsection{Spawning phantom shells}
A phantom shell can be spawned using the rootkit client by sending pattern-based backdoor triggers. As we explained in Section \ref{subsection:c2}, the response to a client command will only be received once a TCP packet is sent from the infected machine to some location. Therefore, we need to wait until any application sends a TCP packet.

For requesting a phantom shell, the attacker must introduce the following arguments:
\begin{itemize}
\item The IP address of the infected machine.
\item The network interface to use for sending the trigger.
\end{itemize}

Once the request is sent by the rootkit client, it will scan the network for the response. As Figure \ref{fig:sc_phantom_1} shows, this rootkit client displays an alert whenever a packet is received.

\begin{figure}[htbp]
	\centering
	\includegraphics[width=12cm]{sch_sc_phantom_1.png}
	\caption{Requesting a phantom shell with the rootkit client.}
	\label{fig:sc_phantom_1}
\end{figure}

At some point, the infected machine will send a TCP packet to any host. We can speed up this process by, for instance, launching a web broswer and visiting any page. When this happens, one TCP packet will be hijacked and sent to the rootkit client, which will show the attacker that the phantom shell is now ready to introduce commands, as shown in figure \ref{fig:sc_phantom_2}.

\begin{figure}[htbp]
	\centering
	\includegraphics[width=11cm]{sch_sc_phantom_2.png}
	\caption{Rootkit client after phantom shell response is received.}
	\label{fig:sc_phantom_2}
\end{figure}



\subsection{eBPF programs control}
The rootkit client incorporates two commands to operate the state of the rootkit eBPF programs using the backdoor, enabling to activate or deactivate them as a group.

Figure \ref{fig:sc_unload_rc} shows how the attacker can detach all eBPF programs (except the backdoor, which as we mentioned in Section \ref{subsection:c2} must stay attached to receive further commands).

\begin{figure}[htbp]
	\centering
	\includegraphics[width=12cm]{sch_sc_unload_rc.png}
	\caption{Requesting to detach all eBPF programs using rootkit client.}
	\label{fig:sc_unload_rc}
\end{figure}

Once the command is executed, we can check that, for instance, the privilege execution module is unloaded, as shown in Figure \ref{fig:sc_unload_res}.

\begin{figure}[htbp]
	\centering
	\includegraphics[width=12cm]{sch_sc_unload_res.png}
	\caption{User osboxes permissions after eBPF programs are detached.}
	\label{fig:sc_unload_res}
\end{figure}

Since the backdoor will be still running, the attacker can now request to attach all eBPF programs again, as shown in Figure \ref{fig:sc_attach_rc}

\begin{figure}[htbp]
	\centering
	\includegraphics[width=12cm]{sch_sc_attach_rc.png}
	\caption{Requesting to attach all eBPF programs using rootkit client.}
	\label{fig:sc_attach_rc}
\end{figure}

After the command is executed, all rootkit modules will be loaded again. We can check it by observing the permissions of the user osboxes, as shown in Figure \ref{fig:sc_attach}.

\begin{figure}[htbp]
	\centering
	\includegraphics[width=12cm]{sch_sc_attach.png}
	\caption{User osboxes permissions after eBPF programs are attached.}
	\label{fig:sc_attach}
\end{figure}


\subsection{Modifying incoming traffic (PoC)} \label{subsection:poc_evaluation}
The backdoor incorporates a simple proof of concept to show how the rootkit may modify incoming network traffic. Although this feature has not been integrated in any of the C2 modules, we considered this functionality to be relevant enough to implement it individually.

This PoC shows the rootkit client sending a packet with a payload \textit{"XDP\_PoC\_0"} sent to the infected machine port 9000. Upon inspection of this packet, the machine will read the content as \textit{"The previous message has been hidden"}. Figure \ref{fig:sc_poc_rc} shows how the rootkit client can send this packet.

\begin{figure}[htbp]
	\centering
	\includegraphics[width=12cm]{sch_sc_poc_rc.png}
	\caption{Sending packet for traffic modification PoC with rootkit client.}
	\label{fig:sc_poc_rc}
\end{figure}

To perform this PoC we will use tcpdump (which we explained in Section \ref{subsection:tcpdump}) to inspect the received packets. Figure \ref{fig:sc_tcpdump_before} shows the packet and payload received when the rootkit is not installed.

\begin{figure}[htbp]
	\centering
	\includegraphics[width=12cm]{sch_tcpdump_before.png}
	\caption{Packet captured with tcpdump in traffic modification PoC with rootkit not installed.}
	\label{fig:sc_tcpdump_before}
\end{figure}

Once the rootkit is installed, it will modify the length and contents, as shown in Figure \ref{fig:sc_tcpdump_after}.

\begin{figure}[htbp]
	\centering
	\includegraphics[width=12cm]{sch_tcpdump_after.png}
	\caption{Packet captured with tcpdump in traffic modification PoC with rootkit installed.}
	\label{fig:sc_tcpdump_after}
\end{figure}


\section{Tampering with system calls}
This functionality has been incorporated in multiple rootkit modules, but it is particularly relevant in the execution hijacking and privilege escalation modules.

\subsection{Hijacking programs execution}
Once the rootkit is installed, it will attempt to hijack any new program that is executed. As we explained in Section \ref{section:execution_hijack}, once the rootkit suceeds a malicious program will be run, which will listen for commands from the rootkit client, enabling the attacker to open a plaintext pseudo-shell. 

In this evaluation, we will attempt to test the hijacking process with a test program \textit{src/helpers/simple\_execve} and another with any process of the machine. Table \ref{table:execution_hijack_config} shows some of the configuration options that must be selected before running this module.

\begin{table}[htbp]
\begin{tabular}{|>{\centering\arraybackslash}p{3cm}|>{\centering\arraybackslash}p{4.5cm}|>{\centering\arraybackslash}p{6cm}|}
\hline
\textbf{FILENAME} & \textbf{CONSTANT} & \textbf{DESCRIPTION}\\
\hline
\hline
src/common/ constants.h & PATH\_EXECUTION\_HIJACK\_PROGRAM & Location of the malicious program to be executed upon succeeding to execute a sys\_execve call.\\
\hline
src/common/ constants.h & EXEC\_HIJACK\_ACTIVE & Deactivate (0) or activate (1) the execution hijacking module.\\
\hline
src/common/ constants.h & TASK\_COMM\_RESTRICT\_HIJACK\_ACTIVE  & Hijack any sys\_execve call (0) or only those indicated in TASK\_COMM\_NAME\_RESTRICT\_HIJACK (1).\\
\hline
src/common/ constants.h & TASK\_COMM\_NAME\_RESTRICT\_HIJACK & Name of the program from which to hijack sys\_execve calls.\\
\hline
\end{tabular}
\caption{Execution hijacking module configuration.}
\label{table:execution_hijack_config}
\end{table}


\textbf{Test program simple\_execve}\\
This program contains a simple sys\_execve call that runs the bash command "pwd", which displays the current directory. As we can observe in Table \ref{table:execution_hijack_config_simple_execve}, for this test we will set the PATH\_EXECUTION\_HIJACK\_PROGRAM setting to the path where we have hidden the malicious program, and set the TASK\_COMM\_NAME\_RESTRICT\_HIJACK setting to indicate that we want to hijack calls executed from the simple\_execve program.

\begin{table}[htbp]
\begin{tabular}{|>{\centering\arraybackslash}p{3cm}|>{\centering\arraybackslash}p{4.5cm}|>{\centering\arraybackslash}p{4cm}|}
\hline
\textbf{FILENAME} & \textbf{CONSTANT} & \textbf{VALUE}\\
\hline
\hline
src/common/ constants.h & PATH\_EXECUTION\_HIJACK\_PROGRAM & "/home/osboxes/ SECRETDIR/ src/helpers/ execve\_hijack"\\
\hline
src/common/ constants.h & EXEC\_HIJACK\_ACTIVE & 1\\
\hline
src/common/ constants.h & TASK\_COMM\_RESTRICT\_HIJACK\_ACTIVE & 1\\
\hline
src/common/ constants.h & TASK\_COMM\_NAME\_RESTRICT\_HIJACK & "simple\_execve"\\
\hline
\end{tabular}
\caption{Execution hijacking module configuration for attacking test program simple\_execve.}
\label{table:execution_hijack_config_simple_execve}
\end{table}

Figure \ref{fig:sc_execve_hijack_before_simple_execve} shows the normal execution of the simple\_execve program. As we can observe, it prints the current directory, as expected.

\begin{figure}[htbp]
	\centering
	\includegraphics[width=12cm]{sch_sc_execve_hijack_before.png}
	\caption{Execution of test program simple\_execve with rootkit not installed.}
	\label{fig:sc_execve_hijack_before_simple_execve}
\end{figure}

Once the rootkit is installed, we will open a shell in the infected machine and execute again the simple\_execve program. The result is shown in Figure \ref{fig:sc_execve_hijack_simple_execve}.

\begin{figure}[htbp]
	\centering
	\includegraphics[width=12cm]{sch_sc_execve_hijack.png}
	\caption{Execution of test program simple\_execve with rootkit installed.}
	\label{fig:sc_execve_hijack_simple_execve}
\end{figure}

As we can observe in the figure, the rootkit hijacked the call and executed the malicious program instead. Each time the malicious program is executed, it alerts us with a message (this would be hidden in a non-experimental case). We can see that it is executed twice (since it needs to run itself as sudo, as we explained in Section \ref{subsection:hijack_program_exec}) and then it forks() itself and executes the original program (we can see the output of pwd) and then starts to listen for the rootkit client connections. Figure \ref{fig:sc_execution_hijack_simple_execve_rc} shows how the rootkit client spawns a plaintext pseudo-shell with the malicious program and runs a command.

\begin{figure}[htbp]
	\centering
	\includegraphics[width=13cm]{sch_sc_execution_hijack_simple_execve_rc.png}
	\caption{Spawning plaintext pseudo-shell with rootkit client.}
	\label{fig:sc_execution_hijack_simple_execve_rc}
\end{figure}

As we can observe in the figure, the rootkit client will connect to the malicious program, enabling the attacker to send any command. Once it is received by the malicious program, it will execute it and answer back to the rootkit client with the output according to the plaintext pseudo-shell network protocol. As shown in Figure \ref{fig:sc_execution_hijack_im}, the malicious program shows information about the actions that have been executed (which would be hidden in a real scenario).

\begin{figure}[htbp]
	\centering
	\includegraphics[width=13cm]{sch_sc_execution_hijack_im.png}
	\caption{Execition of command requested from rootkit client in the infected machine.}
	\label{fig:sc_execution_hijack_im}
\end{figure}


\textbf{Hijacking the execution of any program}\\
As we mentioned in Section \ref{section:execution_hijack}, it is possible that programs fail to be hijacked due to page faults. Because of this, it can take a long time for an specific program (such as bash) to trigger the execution of the malicious program so that the attacker can connect via the plaintext pseudo-shell. This is the reason why the rootkit can also be set to attempt hijacking any program execution from the system instead of restricting the operation to a single process. In this mode, the rootkit will attempt to hijack any sys\_execve call until it succeeds once, afterwards the execution hijacking module will be deactivated. Table \ref{table:execution_hijack_config_any} shows the configuration for this mode.

\begin{table}[htbp]
\begin{tabular}{|>{\centering\arraybackslash}p{3cm}|>{\centering\arraybackslash}p{4.5cm}|>{\centering\arraybackslash}p{4cm}|}
\hline
\textbf{FILENAME} & \textbf{CONSTANT} & \textbf{VALUE}\\
\hline
\hline
src/common/ constants.h & PATH\_EXECUTION\_HIJACK\_PROGRAM & "/home/osboxes/ SECRETDIR/ src/helpers/execve\_hijack"\\
\hline
src/common/ constants.h & EXEC\_HIJACK\_ACTIVE & 1\\
\hline
src/common/ constants.h & TASK\_COMM\_RESTRICT\_HIJACK\_ACTIVE & 0\\
\hline
src/common/ constants.h & TASK\_COMM\_NAME\_RESTRICT\_HIJACK & ""\\
\hline
\end{tabular}
\caption{Execution hijacking module configuration for attempting to hijack any sys\_execve call.}
\label{table:execution_hijack_config_any}
\end{table}

The process will be identical to that shown with the test program simple\_execve. Once a sys\_execve call is hijacked, the malicious program will listen for comamnds sent from the rootkit client, as we showed previously in Figure \ref{fig:sc_execution_hijack_simple_execve_rc}.


\subsection{Privilege escalation}
As we showed in Section \ref{section:privesc}, the privilege escalation module tampers with system calls buffers to modify the contents read from the \textit{/etc/sudoers} file by the sudo process. Figure \ref{fig:sc_sudo_prev} shows the sudo permissions of user osboxes previously to the installation of the rootkit. As we can observe, it has sudo privileges, but requires a password.

\begin{figure}[htbp]
	\centering
	\includegraphics[width=12cm]{sch_sc_sudo_prev.png}
	\caption{Sudo privileges of user osboxes before rootkit installation.}
	\label{fig:sc_sudo_prev}
\end{figure}

Once the rootkit is installed, every time the sudo process requests to read the \textit{/etc/sudoers} file, the contents will be modified. Figure \ref{fig:sc_sudo_after} shows that now the user osboxes appears to have sudo privileges without requiring a password.

\begin{figure}[htbp]
	\centering
	\includegraphics[width=12cm]{sch_sc_sudo_after.png}
	\caption{Sudo privileges of user osboxes after rootkit installation.}
	\label{fig:sc_sudo_after}
\end{figure}

Note that this modification only applies to the sudo process. For instance, if any user wants to read the \textit{/etc/sudoers} file, it appears intact as shown in Figure \ref{fig:sc_sudoers}.

\begin{figure}[htbp]
	\centering
	\includegraphics[width=12cm]{sch_sc_sudoers.png}
	\caption{Reading sudoers file after rootkit installation.}
	\label{fig:sc_sudoers}
\end{figure}

\subsection{Rootkit stealth}
As we presented in Section \ref{section:rootkti_stealth}, the following files and directories will be hidden by the rootkit:
\begin{itemize}
\item Files named "ebpfbackdoor", to hide those corresponding to the rootkit persistence.
\item Entire directories named "SECRETDIR", to hide the rootkit files.
\end{itemize}

The files and directories being hidden can be modified by using the settings shown in Table \ref{table:rootkit_stealth_config}.

\begin{table}[htbp]
\begin{tabular}{|>{\centering\arraybackslash}p{3cm}|>{\centering\arraybackslash}p{4.5cm}|>{\centering\arraybackslash}p{6cm}|}
\hline
\textbf{FILENAME} & \textbf{CONSTANT} & \textbf{DESCRIPTION}\\
\hline
\hline
src/common/ constants.h & SECRET\_DIRECTORY\_NAME\_HIDE & Name of directory to hide.\\
\hline
src/common/ constants.h & SECRET\_FILE\_PERSISTENCE\_NAME & Name of the file to hide.\\
\hline
\end{tabular}
\caption{Rootkit stealth module configuration.}
\label{table:rootkit_stealth_config}
\end{table}

We will now test this module in the infected machine.

\textbf{Hiding rootkit directory}\\
In the attack scenario we described in Section \ref{section:attack_scenario}, the SECRETDIR directory was created under \textit{/home/osboxes} and it was set as the root directory where to hide the rootkit files. Table \ref{table:rootkit_stealth_config_dir} details the rootkit configuration needed to hide this directory.

\begin{table}[htbp]
\begin{tabular}{|>{\centering\arraybackslash}p{3cm}|>{\centering\arraybackslash}p{4.5cm}|>{\centering\arraybackslash}p{6cm}|}
\hline
\textbf{FILENAME} & \textbf{CONSTANT} & \textbf{VALUE}\\
\hline
\hline
src/common/ constants.h & SECRET\_DIRECTORY\_NAME\_HIDE & "SECRETDIR"\\
\hline
\end{tabular}
\caption{Rootkit configuration for hiding directory "SECRETDIR".}
\label{table:rootkit_stealth_config_dir}
\end{table}

Listing the files and directories under the command \textit{ls} yields the results shown in Figure \ref{fig:sc_stealth_dir_before}.

\begin{figure}[htbp]
	\centering
	\includegraphics[width=13cm]{sch_sc_stealth_dir_before.png}
	\caption{Listing files and directories at the home directory before rootkit installation.}
	\label{fig:sc_stealth_dir_before}
\end{figure}

After the rootkit is loaded, we can observe in Figure \ref{fig:sc_stealth_dir_after} that the directory SECRETDIR is not visible anymore.

\begin{figure}[htbp]
	\centering
	\includegraphics[width=13cm]{sch_sc_stealth_dir_after.png}
	\caption{Listing files and directories at the home directory after rootkit installation.}
	\label{fig:sc_stealth_dir_after}
\end{figure}

\textbf{Hiding persistence files}\\
Hiding the \textit{ebpfbackdoor} files can be achieved using the configuration shown in Table \ref{table:rootkit_stealth_config_file}.

\begin{table}[htbp]
\begin{tabular}{|>{\centering\arraybackslash}p{3cm}|>{\centering\arraybackslash}p{4.5cm}|>{\centering\arraybackslash}p{6cm}|}
\hline
\textbf{FILENAME} & \textbf{CONSTANT} & \textbf{VALUE}\\
\hline
\hline
src/common/ constants.h & SECRET\_FILE\_PERSISTENCE\_NAME & "ebpfbackdoor"\\
\hline
\end{tabular}
\caption{Rootkit configuration for hiding file "ebpfbackdoor".}
\label{table:rootkit_stealth_config_file}
\end{table}

As we can observe in Figure \ref{fig:sc_stealth_file_before}, this file is visible before installing the backdoor.

\begin{figure}[htbp]
	\centering
	\includegraphics[width=12cm]{sch_sc_stealth_file_before.png}
	\caption{Listing files and directories at the cron.d directory before rootkit installation.}
	\label{fig:sc_stealth_file_before}
\end{figure}

However, once the rootkit is installed, the file will not be listed under the directory (or any other), as shown in Figure \ref{fig:sc_stealth_file_after}.

\begin{figure}[htbp]
	\centering
	\includegraphics[width=12cm]{sch_sc_stealth_file_after.png}
	\caption{Listing files and directories at the cron.d directory after rootkit installation.}
	\label{fig:sc_stealth_file_after}
\end{figure}


\section{Rootkit persistence}
The files at \textit{/etc/cron.d} and \textit{/etc/sudoers.d} ensure the persistence of the rootkit in the infected system. As we explained in Section \ref{section:persistence}, these files are created by the \textit{deployer.sh} script before loading the rootkit. In this script, two constants define the contents of the entry written in these directories, as shown in Table \ref{table:rootkit_persistence_config}.

\begin{table}[htbp]
\begin{tabular}{|>{\centering\arraybackslash}p{3cm}|>{\centering\arraybackslash}p{4.5cm}|>{\centering\arraybackslash}p{6cm}|}
\hline
\textbf{FILENAME} & \textbf{CONSTANT} & \textbf{DESCRIPTION}\\
\hline
\hline
src/helpers/ deployer.sh & CRON\_PERSIST & Cron job to execute after reboot.\\
\hline
src/helpers/ deployer.sh & SUDO\_PERSIST & Sudo entry to grant password-less privileges.\\
\hline
\end{tabular}
\caption{Rootkit persistence module configuration.}
\label{table:rootkit_persistence_config}
\end{table}

Once the \textit{deployer.sh} script is excuted, the files are created  and, from that point onwards, the cron system will install the rootkit if it is not installed already once every minute. Table \ref{table:rootkit_persistence_config_defaults} shows the values of the configuration that must be set for user "osboxes". If the user of the infected system was another, or the script was located in a different location, the name of this user shall be changed.

\begin{table}[htbp]
\begin{tabular}{|>{\centering\arraybackslash}p{3cm}|>{\centering\arraybackslash}p{4.5cm}|>{\centering\arraybackslash}p{6cm}|}
\hline
\textbf{FILENAME} & \textbf{CONSTANT} & \textbf{VALUE}\\
\hline
\hline
src/helpers/ deployer.sh & CRON\_PERSIST & "* * * * * osboxes /bin/sudo /home/osboxes/TFG/apps/deployer.sh"\\
\hline
src/helpers/ deployer.sh & SUDO\_PERSIST & "osboxes ALL=(ALL:ALL) NOPASSWD:ALL \#"\\
\hline
\end{tabular}
\caption{Configuration for rootkit persistence module with the user "osboxes".}
\label{table:rootkit_persistence_config_defaults}
\end{table}



\section{Takeaways}
In the previous sections, we have explained the steps needed for using the different rootkit modules and displayed its functionalities in a test environment. As we saw, we were able to build at least one rootkit-like functionality using each of the capabilities we proposed at the beginning of this research work for our rootkit. As a summary, for each of these capabilities, we achieved the following:
\begin{itemize}
\item For hijacking running programs, we built a library injection mechanism that does not crash the process and thus allows for stealthy execution of code. We also incorporated a remote control capability for the malicious injected library so that we could execute commands remotely from the rootkit client.
\item With respect to backdoor and C2 capabilities we seeked for the rootkit, we built a comprehensive C2 system supporting multiple stealthy backdoor triggers and encrypted communication systems that allow for executing commands using the rootkit client, apart from an advanced method for exfiltrating data by modifying the outgoing traffic. The multiple stealthy features, as we explained in Section \ref{subsection:triggers}, allow for hiding data from network monitoring software using multiple techniques. Also, we demonstrated the backdoor capabilities for receiving and transmitting actions that manipulate the state of eBPF programs.
\item In the context of manipulating system calls, this was a key capability used in multiple of the rootkit modules. We were able to hijack the execution of programs or modify the contents of critical files in the system, such as \textit{/etc/sudoers}, which granted any rootkit user program privileged permissions.
\item With respect to rootkit persistence, we built a system that allows for surviving reboots, not only ensuring that the rootkit will be installed after one of these events, but also that the root permissions that were once granted to the rootkit the first time it was installed are maintained across reboots.
\item The stealth module we incorporated allows for hiding the directory where the rootkit is stored form the user, along with those files responsible from ensuring the rootkit persistence.
\end{itemize}

Taking into account all the above, we can confidently claim that we fulfilled the project objectives of our rootkit development.

\chapter{Related work}
% Comparison of the rootkit with other eBPF and non eBPF rootkits.

%Move here part of the rootkit section at the intro.
\chapter{Project budget}
This chapter describes the budget associated to the development of this
research project. For this, we will take into account the costs of the time
invested on research, development and documentation writing, along with
other indirect costs associated to the project activities.

\section{Gantt chart}
Figure \ref{fig:gantt_chart} shows a Gantt presenting the different stages of the project and the distribution of time between them. As we can observe in the figure, the project can be divided into three main sections:
\begin{itemize}
\item Preliminary research on previous work.
\item Development of each rootkit module.
\item Documentation.
\end{itemize}

It is relevant to note that in this research work, because of the complexity and variety of functionalities of the eBPF system, each of the offensive capabilities of eBPF has been discovered and implemented as a rootkit module individually. Therefore, there has not existed a single iteration of analysis, design and implementation, but rather multiple iterations have been made to develop each module. This is the reason why, if we focus our view in the development stages, each consists on at least one analysis and multiple design and implementation activities.

\newgeometry{hmargin=3cm,vmargin=2cm}
\thispagestyle{lscape}
\begin{landscape}
\begin{figure}[htbp]
	\centering
	\includegraphics[width=21cm]{gantt_chart.jpg}
	\caption{Gantt chart of the project.}
	\label{fig:gantt_chart}
\end{figure}
\end{landscape}
\restoregeometry

\section{Estimated costs}
This section presents an estimation of the costs associated with the  personnel conducting the activities described in the Gantt chart in addition to all costs derived from the development of this work.

\subsection{Personnel costs}
Although this project has been developed individually under the supervisor guidance, we can identify three different roles:
\begin{itemize}
\item A \textbf{cyber security analyst}: a role requiring expertise and knowledge about multiple aspects of Linux systems (such as ELFs, memory architecture and attacks at process memory), needed for identifying possible offensive capabilities of eBPF. Therefore this role is responsible of research and analysis of the offensive capabilities of eBPF. It will also write the corresponding documentation with the gathered knowledge.
\item A \textbf{programmer}: a role requiring knowledge about C programming and, preferably, eBPF developing experience (which requires a different skillset than normal C, being more similar to the development of programs for the Linux kernel).
\item A \textbf{project manager}: a role which administers the tasks and objectives to complete, contributing leadership and guidance to the team.
\end{itemize}

We will now consider the wages assigned to each role. The monthly and hourly salaries are displayed on Table \ref{table:salary_personnel}, and have been obtained using the salaries shown by Glassdor for each role in the city of Madrid \cite{glass_analyst} \cite{glass_manager} \cite{glass_programmer}. We have also assumed that these roles correspond to full-time positions consisting of 40 hours a week, 8 hours a day, with no vacations.

\begin{table}[htbp]
\begin{tabular}{|c|c|c|}
\hline
\textbf{ROLE} & \textbf{MONTHLY RATE} & \textbf{HOURLY RATE}\\
\hline
\hline
Cyber security analyst & 26,424 € & 12.70 € \\
\hline
Programmer & 27,018 € & 13.00 € \\
\hline
Project manager & 40,000 € & 19.23 € \\ 
\hline
\end{tabular}
\caption{Average monthly and hourly salary for project staff.}
\label{table:salary_personnel}
\end{table}

Given the different responsabilities of the team members on the project, Table \ref{table:hours_personnel} shows the number of hours which each person dedicates daily  to the project in average when perfoming each of the tasks (that is, the length of a working day when assigned to each task). 

Also, note that our own RawTCP\_Lib library is a relevant part of this project but it has been developed outside of the scope of this research. Therefore, we will consider it as an estimated 20-days long 4 hours/day development by the programmer.

\begin{table}[htbp]
\begin{tabular}{|c|c|c|}
\hline
\textbf{ROLE} & \textbf{TASK} & \textbf{HOURS/DAY}\\
\hline
\hline
\multirow{2}{*}{Cyber security analyst} & \multicolumn{1}{c|}{Research and analysis} & \multicolumn{1}{c|}{5}\\
\cline{2-3}
& \multicolumn{1}{c|}{Documentation writing} & \multicolumn{1}{c|}{10} \\
\hline
\multirow{2}{*}{Programmer} & \multicolumn{1}{c|}{Rootkit implementation} & \multicolumn{1}{c|}{7} \\
\cline{2-3}
& \multicolumn{1}{c|}{RawTCP\_Lib development} & \multicolumn{1}{c|}{4} \\
\hline
Project manager & Supervision and guidance & 1.16 \\ 
\hline
\end{tabular}
\caption{Daily dedication, in hours, that each personnel member needs to dedicate to each of their tasks.}
\label{table:hours_personnel}
\end{table}

With respect to the project manager, whose supervision task was not shown in the Gantt chart, we have considered an estimate of a total of 250 hours worked over the 215 days long project, dedicating an average of 8.18 hours once every week, or 1.16 hours daily.

With these salaries and work hours in mind, the tasks described on the Gantt chart are then distributed among these roles, as shown in Table \ref{table:personnel_total}. The total salary is calculated by taking into account the hourly salary of each role and the number of hours worked on each task (the product between hours in a working day and the total number of days).

\begin{table}[htbp]
\begin{tabular}{|>{\centering\arraybackslash}p{3cm}|c|>{\centering\arraybackslash}p{3cm}|c|}
\hline
\textbf{ROLE} & \textbf{TASK} & \textbf{DEDICATION} & \textbf{TOTAL}\\
\hline
\hline
\multirow{2}{*}{\shortstack{Cyber security\\ analyst}} &
     \multicolumn{1}{c|}{Research and analysis} & \multicolumn{1}{c|}{27
     days} & \multirow{1}{*}{1,714.50 €}\\
\cline{2-4}
& \multicolumn{1}{c|}{Documentation writing} & \multicolumn{1}{c|}{35 days}
     & \multicolumn{1}{c|}{4,445 €}\\
\hline
\multirow{2}{*}{Programmer} & \multicolumn{1}{c|}{Rootkit implementation} &
     \multicolumn{1}{c|}{84 days} & \multicolumn{1}{c|}{7,644 €}\\
\cline{2-4}
& \multicolumn{1}{c|}{RawTCP\_Lib development} & \multicolumn{1}{c|}{20
     days} & \multicolumn{1}{c|}{1,040 €}\\
\hline
Project manager & Supervision and guidance & 215 days & 4,807.50€ \\ 
\hline
\multicolumn{1}{c}{} & & \textbf{TOTAL} & 19,641 €\\
\cline{3-4}
\end{tabular}
\caption{Total costs associated to personnel.}
\label{table:personnel_total}
\end{table}


\subsection{Hardware costs}
There exists an additional cost associated to the purchase of hardware equipment needed. Table \ref{table:hardware_costs} details this cost.

\begin{table}[htbp]
\begin{tabular}{|c|c|}
\hline
\textbf{COMPONENT} & \textbf{PRICE}\\
\hline
\hline
HP OMEN 16-c0050ns & 1,300 € \\
\hline
\textbf{TOTAL} & 1,300 €\\
\hline
\end{tabular}
\caption{Estimated cost of hardware systems.}
\label{table:hardware_costs}
\end{table}

\subsection{Software costs}
All software used during this research work is open source and thus it has no additional cost. This can be observed in Table \ref{table:software_costs}.
%Ill add the version here
\begin{table}[htbp]
\begin{tabular}{|c|c|}
\hline
\textbf{COMPONENT} & \textbf{PRICE}\\
\hline
\hline
Ubuntu 21.04 & 0 € \\
\hline
libbpf & 0 € \\
\hline
Oracle VM Virtualbox & 0 € \\
\hline
\textbf{TOTAL} & 0 €\\
\hline
\end{tabular}
\caption{Cost of software components.}
\label{table:software_costs}
\end{table}

\subsection{Total costs}
The computation of the total costs involves considering the costs of hardware, software and personnel systems, together with an additive indirect cost related to minor expenses such as Internet connection or electricity consumption. We will consider these costs to be a 10\% of the total. Additionaly, note that this is a research project and, as such, it would usually be funded, so we would not have any benefits. Table \ref{table:total_costs} shows the total costs of this project.

%TODO improve the look of this table
\begin{table}[htbp]
\begin{tabular}{|c|c|}
\hline
\textbf{COST TYPE} & \textbf{PRICE}\\
\hline
\hline
Personnel costs & 19,641 € \\
\hline
Hardware costs & 1,300 € \\
\hline
Software costs & 0 € \\
\hline
\textbf{SUBTOTAL} & 20,941 €\\
\hline
Indirect costs & 10\% €\\
\hline
\textbf{TOTAL} & 23,035.10 €\\
\hline
\end{tabular}
\caption{Total cost of the project.}
\label{table:total_costs}
\end{table}

\chapter{Conclusions and future work}
This chapter revisits the project objectives, discusses the work presented
in this document, and describes possible future research lines.

\section{Conclusions}
At the beginning of this project, we proposed to study the offensive
capabilities of eBPF at the network level and both user- and kernel-space.
Our research shows that a malicious eBPF program can drop any network
packet and have read and write access over both incoming and outgoing
network traffic using XDP and TC programs. We also discuss how it can
read and write any memory at the user-space using kprobes and tracepoints,
and that it can tamper with user data passed to the kernel at system calls,
although kernel memory cannot be written. In the end, these capabilities
result in a complete disrupt of trust between the user and kernel space
since eBPF may modify data passed to system calls and thus change the
outcome of the execution, a disrupt of trust among the user space programs
themselves since eBPF may redirect the flow of execution or overwrite any
data by writing to specific sections at processes virtual memory, and
finally total control over the data sent or receieved at the network.

With these capabilities in mind, we have developed an eBPF-based rootkit
that uses these offensive capabilities to showcase multiple malicious use
cases. Our rootkit, named TripleCross, incorporates (1) a library injection
module to execute malicious code by writing at processes virtual memory;
(2) an execution hijacking module that modifies data passed to the kernel
to execute malicious programs; (3) a local privilege escalation module that
allows for running malicious programs with root privileges; (4) a
backdoor with C2 capabilities that can monitor the network and execute
commands sent from a remote rootkit client which incorporates multiple
backdoor triggers so that these actions are transmitted to the backdoor
with stealth in mind; (5) a rootkit client program that allows the attacker
to establish 3 different types of shell-like connections for sending
commands and multiple other actions that control the rootkit state
remotely; (6) a persistence module that uses a combination of scheduled
jobs and malicious configuration files at the sudo system to ensure the
rootkit remains installed with full privileges even after a reboot event;
and (7) a stealth module that hides rootkit-related files and directories
from the user.

TripleCross demonstrates the existing danger when running eBPF programs, a
technology also available by default in most distributions. On the other
hand, it must be noted that there exist some defense measures against these
rootkits:
\begin{itemize}
\item Monitor the loaded eBPF programns and the data stored at eBPF maps using tools like \textit{bpftool} or \textit{ebpfkit-monitor} \cite{ebpfkit_monitor_github} (a tool released by Fournier and Afchain that monitors the loaded eBPF programs and maps).
\item Monitor the use of the bpf() syscall in the system. The \textit{ebpfkit-monitor} tool also incorporates this capability.
\item Wait until eBPF signing is implemented in the kernel. Although this capability is not currently available, there exist some efforts towards its incorporation in the kernel \cite{bpf_signing}. Similarly to how LKMs can be signed with a private key so that the kernel only trust modules signed by the entity with the corresponding public key \cite{lkm_signing}, eBPF programs may require a similar signing process before being loaded into the BPF VM.

Note that, even if this capability is included in the future, it may be
left off by default, as it has happened with signed LKMs. Signing modules is governed by the parameter CONFIG\_MODULE\_SIG\_FORCE, which is left deactivated in some kernel compilations for backwards compatibility \cite{arch_linux_sign}.
\item Assign the lowest privilege possible to eBPF programs according to their expected functionality, as described in Section \ref{subsection:access_control}.
\item Monitor the network using IDSs and network-wide firewalls, detecting suspicious communications. Firewalls installed on the endpoints may detect ongoing malicious traffic too (but incoming traffic would be masked by XDP before it reaches the firewall).
\end{itemize}

Nevertheless, with the exception of signing eBPF programs, a sufficiently
advanced rootkit built for an specific targeted attack will be able to
bypass any monitoring actions taken at the infected host. This rootkit
could hide itself from the \textit{bpftool} tool, block access to its eBPF
maps and, ultimately, hide its activities from any monitoring tool or log
traces. This is the conclusion at which Fournier and Afchain also arrive
\cite{ebpf_friends_54}.

\section{Future work}
Although in this project we identified several offensive capabilities
using the current functionality supported by eBPF, this technology is
currently being extended and, therefore, the incorporation of new eBPF
helpers and program types could result in new offensive uses.
%
In addition, there also exist multiple capabilities that have not been
researched in depth and that can result in other attacks. Namely, the use
of uprobes, which hooks functions from specific programs, could be used to
modify the data of user space programs in the benefit of the rootkit. For
instance, an attacker could overwrite the data gathered by a firewall
installed in the system so that malicious outgoing traffic appears as
benign. Therefore, further research on uprobe programs with eBPF could
result in new attacks against specific user programs that could be
incorporated into a rootkit.

Another relevant line of work would be the modification of buffers passed
by the user which, instead of being received at system calls, are received
and operated at internal kernel functions. A rootkit overwriting this data
could alter the execution of the kernel itself outside of syscalls. 

Other lines of research include building rootkit modules using eBPF helpers
that we did not incorporate in our rootkit, such as bpf\_override\_return
and  bpf\_send\_signal, or the XDP packet modification capabilities that we
only showed as a PoC. TripleCross could then incorporate techniques such as
hiding itself from the kernel logs and find new uses for modifying incoming
network packets.

A final but very relevant research line consists of exploring the
capabilities of eBPF in Windows and Android. Since it is a novel
incorporation, there currently exists little knowledge about the limits of
eBPF in these systems, and thus it is of great interest to research which
actions a malicious program could perform in these platforms.

In summary, future work in offensive eBPF could be aimed at finding new
attack vectors for the capabilities used to develop our rootkit, and
building more complex techniques combining those we did not explore in this
work. Moreover, since the eBPF system keeps being expanded not only in
Linux but in other platforms too, it is relevant to analyze the offensive
uses for the newer functionalities of eBPF incorporated in the future.









%----------
%	BIBLIOGRAPHY
%----------	

%\nocite{*} % Si quieres que aparezcan en la bibliografía todos los documentos que la componen (también los que no estén citados en el texto) descomenta está lína

\clearpage
\addcontentsline{toc}{chapter}{Bibliography}

\printbibliography


%----------
%	ANEX
%----------	

%M-> Mentioned putting some demos and PoCs here...
%

%Including bpftool commands here to be referenced. Is it a good idea?


\chapter* {Appendix A - Bpftool commands} \label{annex:bpftool_flags_kernel}
\pagenumbering{gobble} % Las páginas de los anexos no se numeran
\section*{eBPF-related kernel compilation flags} 
\begin{lstlisting}[language=bash]
$ bpftool feature
\end{lstlisting}

\begin{verbatim}
CONFIG_BPF is set to y
CONFIG_BPF_SYSCALL is set to y
CONFIG_HAVE_EBPF_JIT is set to y
CONFIG_BPF_JIT is set to y
CONFIG_BPF_JIT_ALWAYS_ON is set to y
CONFIG_CGROUPS is set to y
CONFIG_CGROUP_BPF is set to y
CONFIG_CGROUP_NET_CLASSID is set to y
CONFIG_SOCK_CGROUP_DATA is set to y
CONFIG_BPF_EVENTS is set to y
CONFIG_KPROBE_EVENTS is set to y
CONFIG_UPROBE_EVENTS is set to y
CONFIG_TRACING is set to y
CONFIG_FTRACE_SYSCALLS is set to y
CONFIG_FUNCTION_ERROR_INJECTION is set to y
CONFIG_BPF_KPROBE_OVERRIDE is set to y
CONFIG_NET is set to y
CONFIG_XDP_SOCKETS is set to y
CONFIG_LWTUNNEL_BPF is set to y
CONFIG_NET_ACT_BPF is set to m
CONFIG_NET_CLS_BPF is set to m
CONFIG_NET_CLS_ACT is set to y
CONFIG_NET_SCH_INGRESS is set to m
CONFIG_XFRM is set to y
CONFIG_IP_ROUTE_CLASSID is set to y
CONFIG_IPV6_SEG6_BPF is set to y
CONFIG_BPF_LIRC_MODE2 is not set
CONFIG_BPF_STREAM_PARSER is set to y
CONFIG_NETFILTER_XT_MATCH_BPF is set to m
CONFIG_BPFILTER is set to y
CONFIG_BPFILTER_UMH is set to m
CONFIG_TEST_BPF is set to m
CONFIG_HZ is set to 250
\end{verbatim}


\chapter* {Appendix B - Readelf commands} \label{annex:readelf_commands}
\pagenumbering{gobble} % Las páginas de los anexos no se numeran
\section*{Section headers in ELF file} \label{annexsec:readelf_sec_headers}
\begin{lstlisting}[language=bash, caption={List of ELF section headers with readelf tool of a program compiled with GCC.}, label={code:elf_sections}]
$ readelf -S simple_timer
There are 36 section headers, starting at offset 0x4120:

Section Headers:
  [Nr] Name              Type             Address           Offset
       Size              EntSize          Flags  Link  Info  Align
  [ 0]                   NULL             0000000000000000  00000000
       0000000000000000  0000000000000000           0     0     0
  [ 1] .interp           PROGBITS         0000000000400318  00000318
       000000000000001c  0000000000000000   A       0     0     1
  [ 2] .note.gnu.pr[...] NOTE             0000000000400338  00000338
       0000000000000030  0000000000000000   A       0     0     8
  [ 3] .note.gnu.bu[...] NOTE             0000000000400368  00000368
       0000000000000024  0000000000000000   A       0     0     4
  [ 4] .note.ABI-tag     NOTE             000000000040038c  0000038c
       0000000000000020  0000000000000000   A       0     0     4
  [ 5] .gnu.hash         GNU_HASH         00000000004003b0  000003b0
       000000000000001c  0000000000000000   A       6     0     8
  [ 6] .dynsym           DYNSYM           00000000004003d0  000003d0
       0000000000000108  0000000000000018   A       7     1     8
  [ 7] .dynstr           STRTAB           00000000004004d8  000004d8
       00000000000000ad  0000000000000000   A       0     0     1
  [ 8] .gnu.version      VERSYM           0000000000400586  00000586
       0000000000000016  0000000000000002   A       6     0     2
  [ 9] .gnu.version_r    VERNEED          00000000004005a0  000005a0
       0000000000000050  0000000000000000   A       7     1     8
  [10] .rela.dyn         RELA             00000000004005f0  000005f0
       0000000000000030  0000000000000018   A       6     0     8
  [11] .rela.plt         RELA             0000000000400620  00000620
       00000000000000c0  0000000000000018  AI       6    24     8
  [12] .init             PROGBITS         0000000000401000  00001000
       000000000000001b  0000000000000000  AX       0     0     4
  [13] .plt              PROGBITS         0000000000401020  00001020
       0000000000000090  0000000000000010  AX       0     0     16
  [14] .plt.sec          PROGBITS         00000000004010b0  000010b0
       0000000000000080  0000000000000010  AX       0     0     16
  [15] .text             PROGBITS         0000000000401130  00001130
       00000000000004c5  0000000000000000  AX       0     0     16
  [16] .fini             PROGBITS         00000000004015f8  000015f8
       000000000000000d  0000000000000000  AX       0     0     4
  [17] .rodata           PROGBITS         0000000000402000  00002000
       00000000000000a5  0000000000000000   A       0     0     8
  [18] .eh_frame_hdr     PROGBITS         00000000004020a8  000020a8
       000000000000004c  0000000000000000   A       0     0     4
  [19] .eh_frame         PROGBITS         00000000004020f8  000020f8
       0000000000000120  0000000000000000   A       0     0     8
  [20] .init_array       INIT_ARRAY       0000000000403e10  00002e10
       0000000000000008  0000000000000008  WA       0     0     8
  [21] .fini_array       FINI_ARRAY       0000000000403e18  00002e18
       0000000000000008  0000000000000008  WA       0     0     8
  [22] .dynamic          DYNAMIC          0000000000403e20  00002e20
       00000000000001d0  0000000000000010  WA       7     0     8
  [23] .got              PROGBITS         0000000000403ff0  00002ff0
       0000000000000010  0000000000000008  WA       0     0     8
  [24] .got.plt          PROGBITS         0000000000404000  00003000
       0000000000000058  0000000000000008  WA       0     0     8
  [25] .data             PROGBITS         0000000000404058  00003058
       0000000000000014  0000000000000000  WA       0     0     8
  [26] .bss              NOBITS           0000000000404070  0000306c
       0000000000000020  0000000000000000  WA       0     0     16
  [27] .comment          PROGBITS         0000000000000000  0000306c
       0000000000000025  0000000000000001  MS       0     0     1
  [28] .debug_aranges    PROGBITS         0000000000000000  00003091
       0000000000000030  0000000000000000           0     0     1
  [29] .debug_info       PROGBITS         0000000000000000  000030c1
       0000000000000295  0000000000000000           0     0     1
  [30] .debug_abbrev     PROGBITS         0000000000000000  00003356
       00000000000000fd  0000000000000000           0     0     1
  [31] .debug_line       PROGBITS         0000000000000000  00003453
       000000000000024d  0000000000000000           0     0     1
  [32] .debug_str        PROGBITS         0000000000000000  000036a0
       00000000000001f5  0000000000000001  MS       0     0     1
  [33] .symtab           SYMTAB           0000000000000000  00003898
       0000000000000480  0000000000000018          34    22     8
  [34] .strtab           STRTAB           0000000000000000  00003d18
       00000000000002a2  0000000000000000           0     0     1
  [35] .shstrtab         STRTAB           0000000000000000  00003fba
       000000000000015f  0000000000000000           0     0     1
Key to Flags:
  W (write), A (alloc), X (execute), M (merge), S (strings), I (info),
  L (link order), O (extra OS processing required), G (group), T (TLS),
  C (compressed), x (unknown), o (OS specific), E (exclude),
  l (large), p (processor specific)
\end{lstlisting}





\end{document}
